\chapter{Resultados}\label{cap:capitulo_4}
Este capítulo apresenta e discute os resultados obtidos a partir das simulações descritas no capítulo anterior, considerando sistemas fotovoltaicos de diferentes portes e duas configurações de instalação: estrutura fixa e rastreamento solar de um eixo (\textit{single-axis tracker}). As análises contemplam potências de $10$ kW, $50$ kW, $100$ kW, $300$ kW, $500$ kW, $800$ kW e $1$ MW, permitindo avaliar o desempenho técnico e econômico em diferentes escalas. Para cada cenário, foram avaliados a geração anual de energia, o investimento inicial (\ac{CAPEX}), os custos operacionais (\ac{OPEX}), a \ac{TIR} e o tempo de retorno do investimento (\textit{payback}). Em seguida, são apresentadas análises comparativas visando identificar em quais faixas de potência a adoção do rastreamento solar se torna economicamente mais atrativa, fornecendo subsídios técnicos e financeiros para a tomada de decisão em projetos fotovoltaicos.
%Neste capítulo são apresentados e discutidos os resultados obtidos a partir das simulações realizadas no capitulo anterior, considerando sistemas fotovoltaicos de diferentes portes e suas configurações distintas: sistemas fixos e sistemas com rastreamento solar de um eixo. As análises contemplam potências de $10$ kW, $50$ kW, $100$ kW, $300$ kW, $500$ kW, $800$ kW e $1$MW, permitindo avaliar o desempenho técnico e econômico em diferentes escalas de projeto. Para cada cenário, foram observados parâmetros como a geração anual de energia, o investimento inicial \ac{CAPEX}, os custos operacionais \ac{OPEX}, o \ac{TIR}, o Tempo de Retorno do Investimento \textit{(Payback)} . Após a apresentação dos resultados, são realizadas análises comparativas com o objetivo de identificar em quais faixas de potência a adoção do rastreamento solar se torna mais vantajosa em relação ao sistema fixo, fornecendo subsídios técnicos e financeiros para a tomada de decisão em projetos de usinas fotovoltaicas.

\section{Usina de $10$kW}
\subsection{Resultados técnicos}
Nesta seção será feita um resumo com os principais resultados técnicos obtidos através da simulação das usinas de $10$kW de eixo fixo e móvel. A \autoref{tab:rt10} mostra os resultados da simulação:

\begin{table}[h!]
	\centering
	\caption{Resultados técnicos obtidos no PV*SOL para a usina de $10$ kW em Belo Horizonte--MG, comparando estrutura fixa e rastreamento solar de um eixo, com energia anual gerada, rendimento específico e eficiência global do sistema.}
	\begin{tabular}{C{3cm}C{3cm}C{3cm}C{3cm}}
	\hline
	\textbf{Sistema} & \textbf{Energia anual [kWh]} & \textbf{Rendimento [kWh/kWp]} & \textbf{Eficiência [\%]} \\ \hline
	Fixo & $14.523$ &  $1.376,15$ & $67,60$\% \\
	\textit{Tracker} & $17.746$ &  $1.698,48$ & $69,73$\% \\ \hline
	\end{tabular}
	\fonteautor
	\label{tab:rt10}
\end{table}

Nas usinas de $10$ kW, observa-se que a utilização de \textit{tracker} proporciona um aumento significativo na produção de energia anual, passando de $14.523$ kWh no sistema fixo para $17.746$ kWh no sistema rastreador. Esse incremento representa aproximadamente $22$\% a mais de geração. O rendimento específico também acompanha essa evolução, variando de $1.376$ kWh/kWp no fixo para $1.698$ kWh/kWp no \textit{tracker}. Em termos de eficiência global do sistema, a diferença é modesta, mas perceptível: o fixo apresenta $67,6$\%, enquanto o \textit{tracker} alcança $69,7$\%.

\subsection{Resultados Financeiros}
Nesta seção será feita um resumo com os principais resultados financeiros obtidos através da simulação das usinas de $10$kW de eixo fixo e móvel. A \autoref{tab:rf10} mostra os resultados da simulação:

\begin{table}[h!]
	\centering
	\caption{Indicadores econômico-financeiros estimados no PV*SOL para a usina de $10$ kW em Belo Horizonte--MG, comparando estrutura fixa e rastreamento solar de um eixo: custo de geração, \textit{payback}, \ac{TIR} e \ac{ROI}.}
	\begin{tabular}{C{3cm}C{3cm}C{3cm}C{2.5cm}C{2.5cm}} \hline
	\textbf{Sistema} & \textbf{Custo de geração [R\$/kWh]} & \textbf{\textit{PayBack} [anos]} & \textbf{\ac{TIR} [\%]} & \textbf{\ac{ROI} [\%]} \\ \hline
	Fixo & $0,2198$ & $7$ anos, $10$ meses & $11,91$ & $14,26$ \\
	\textit{Tracker} & $0,2331$ & $8$ anos  & $11,94$ & $14,52$ \\ \hline
	\end{tabular}
	\fonteautor
	\label{tab:rf10}
\end{table}

Nas usinas de $10$ kW, o sistema de estrutura fixa apresenta um menor \textit{payback}, de aproximadamente $7$ anos e $10$ meses, enquanto o sistema com \textit{tracker} atinge o retorno em cerca de $8$ anos, indicando uma recuperação do investimento mais rápida para a configuração fixa. Em termos de custo de geração de energia, o sistema fixo também se mostra mais vantajoso, com valor médio de R\$ $0,2198$/kWh, frente a R\$$ 0,2331$/kWh no sistema com \textit{tracker}.

Por outro lado, os indicadores de rentabilidade apresentam valores próximos entre as duas configurações. O sistema fixo registra uma \ac{TIR} de $11,91$\% e \ac{ROI} de $14,26$\%, enquanto o sistema com \textit{tracker} alcança uma \ac{TIR} ligeiramente superior, de $11,94$\%, e \ac{ROI} de $14,52$\%. Apesar dessa pequena vantagem do \textit{tracker} nos indicadores de retorno percentual, o maior custo de geração e o \textit{payback} mais longo fazem com que, para a escala de $10$ kW, o sistema fixo permaneça como a alternativa economicamente mais atrativa.

\section{Usina de $50$kW}
\subsection{Resultados técnicos}
Nesta seção será feita um resumo com os principais resultados técnicos obtidos através da simulação das usinas de $50$kW de eixo fixo e móvel. A \autoref{tab:rt50} mostra os resultados da simulação:

\begin{table}[h!]
	\centering
	\caption{Resultados técnicos obtidos no PV*SOL para a usina de $50$ kW em Belo Horizonte--MG, comparando estrutura fixa e rastreamento solar de um eixo, com energia anual gerada, rendimento específico e eficiência global do sistema.}
	\begin{tabular}{C{3cm}C{3cm}C{3cm}C{3cm}} \hline
	\textbf{Sistema} & \textbf{Energia anual [kWh]} & \textbf{Rendimento [kWh/kWp]} & \textbf{Eficiência [\%]} \\ \hline
	Fixo & $72.251$ & $1.430,18$ &  $70,25$ \\
	\textbf{Tracker} & $88.295$ &  $1.751,08$ & $71,89$  \\ \hline
	\end{tabular}
	\fonteautor
	\label{tab:rt50}
\end{table}

Nos \ac{SFCR} de $50$ kW, a diferença técnica mantém-se proporcional. O sistema fixo injeta cerca de $72.251$ kWh/ano, ao passo que o \textit{tracker} atinge $88.295$ kWh/ano, representando também um ganho de aproximadamente $22$\%. O rendimento específico sobe de $1.430$ kWh/kWp para $1.751$ kWh/kWp, evidenciando melhor aproveitamento do recurso solar. Já a eficiência, embora próxima, é ligeiramente superior no \textit{tracker} ($71,9$\%) em comparação ao fixo (70,3\%).

\subsection{Resultados Financeiros}
Nesta seção será feita um resumo com os principais resultados financeiros obtidos através da simulação das usinas de $50$kW de eixo fixo e móvel. A \autoref{tab:rf50} mostra os resultados da simulação:
\begin{table}[h!]
	\centering
	\caption{Indicadores econômico-financeiros estimados no PV*SOL para a usina de $10$ kW em Belo Horizonte--MG, comparando estrutura fixa e rastreamento solar de um eixo: custo de geração, \textit{payback}, \ac{TIR} e \ac{ROI}.}
	\begin{tabular}{C{3cm}C{3cm}C{3cm}C{2.5cm}C{2.5cm}} \hline
		\textbf{Sistema} & \textbf{Custo de geração [R\$/kWh]} & \textbf{\textit{PayBack} [anos]} & \textbf{\ac{TIR} [\%]} & \textbf{\ac{ROI} [\%]} \\ \hline
	Fixo & $0,1961$ & $7$ anos, $3$ meses & $13,07$ & $15,30$ \\ 
	\textbf{Tracker} & $0,234$ &  $8$ anos, $4$ meses & $11,00$ & $13,95$ \\ \hline
	\end{tabular}
	\fonteautor
	\label{tab:rf50}
\end{table}

Nas usinas de $50$ kW, o sistema de estrutura fixa apresenta desempenho econômico superior em relação ao sistema com \textit{tracker}. O \textit{payback} do sistema fixo é de aproximadamente $7$ anos e $3$ meses, enquanto o sistema com \textit{tracker} apresenta um retorno mais longo, de cerca de $8$ anos e $4$ meses. O custo de geração de energia também é significativamente menor no sistema fixo, com valor médio de R\$ $0,1961$/kWh, frente a R\$ $0,234$/kWh no sistema com \textit{tracker}.

No que se refere aos indicadores de rentabilidade, o sistema fixo apresenta \ac{TIR} de $13,07$\% e \ac{ROI} de $15,30$\%, ambos superiores aos obtidos pelo sistema com \textit{tracker}, que registra \ac{TIR} de $11,00$\% e \ac{ROI} de $13,95$\%. Dessa forma, para a escala de $50$ kW, os resultados indicam que o sistema de estrutura fixa é economicamente mais atrativo, apresentando menor custo de geração, retorno do investimento mais rápido e melhores indicadores financeiros globais.

\section{Usina de $100$kW}
\subsection{Resultados técnicos}
Nesta seção será feita um resumo com os principais resultados técnicos obtidos através da simulação das usinas de $100$kW de eixo fixo e móvel. A \autoref{tab:rt100} mostra os resultados da simulação:
\begin{table}[h!]
	\centering
	\caption{Resultados técnicos obtidos no PV*SOL para a usina de $100$ kW em Belo Horizonte--MG, comparando estrutura fixa e rastreamento solar de um eixo, com energia anual gerada, rendimento específico e eficiência global do sistema.}
	\begin{tabular}{C{3cm}C{3cm}C{3cm}C{3cm}} \hline
		\textbf{Sistema} & \textbf{Energia anual [kWh]} & \textbf{Rendimento [kWh/kWp]} & \textbf{Eficiência [\%]} \\ \hline
Fixo & $144.501$ & $1.430,18$ & $70,30$ \\ 
\textit{Tracker} & $176.590$ & $1.751,08$  & $71,89$  \\ \hline
	\end{tabular}
	\fonteautor
	\label{tab:rt100}
\end{table}

Para as usinas $100$ kW, o cenário se repete: o sistema fixo gera 144.501 kWh/ano, enquanto o \textit{tracker} alcança $176.590$ kWh/ano, mantendo a tendência de 22\% de ganho. O rendimento específico passa de $1.430$ para $1.751$ kWh/kWp, e a eficiência do sistema se mantém superior no rastreador ($71,89$\% contra $70,3$\%).

\subsection{Resultados Financeiros}
Nesta seção será feita um resumo com os principais resultados financeiros obtidos através da simulação das usinas de $100$kW de eixo fixo e móvel. A \autoref{tab:rf100} mostra os resultados da simulação:
\begin{table}[h!]
	\centering
	\caption{Indicadores econômico-financeiros estimados no PV*SOL para a usina de $100$ kW em Belo Horizonte--MG, comparando estrutura fixa e rastreamento solar de um eixo: custo de geração, \textit{payback}, \ac{TIR} e \ac{ROI}.}
	\begin{tabular}{C{3cm}C{3cm}C{3cm}C{2.5cm}C{2.5cm}} \hline
		\textbf{Sistema} & \textbf{Custo de geração [R\$/kWh]} & \textbf{\textit{PayBack} [anos]} & \textbf{\ac{TIR} [\%]} & \textbf{\ac{ROI} [\%]} \\ \hline
Fixo & $0,1846$ & $6$ anos, $9$ meses & $14,13$ & $16,26$ \\ \hline
\textit{Tracker} & $0,22$ & $7$ anos, $10$ meses  & $11,99$ & $14,83$ \\ \hline
	\end{tabular}
	\fonteautor
	\label{tab:rf100}
\end{table}

Nas usinas de $100$ kW, o sistema de estrutura fixa mantém desempenho econômico superior quando comparado ao sistema com \textit{tracker}. O \textit{payback} do sistema fixo é de aproximadamente $6$ anos e $9$ meses, enquanto o sistema com \textit{tracker} apresenta um retorno mais longo, de cerca de $7$ anos e $10$ meses. O custo de geração de energia do sistema fixo também é inferior, com valor médio de R\$ $0,1846$/kWh, frente a R\$ $0,22$/kWh no sistema com \textit{tracker}.

Em relação aos indicadores de rentabilidade, o sistema fixo apresenta \ac{TIR} de $14,13$\% e \ac{ROI} de $16,26$\%, valores superiores aos observados no sistema com \textit{tracker}, que registra \ac{TIR} de $11,99$\% e \ac{ROI} de $14,83$\%. Dessa forma, para a escala de $100$ kW, os resultados indicam que a configuração com estrutura fixa é economicamente mais atrativa, combinando menor custo de geração, retorno mais rápido do investimento e melhores indicadores financeiros.

\section{Usina de $300$kW}

\subsection{Resultados técnicos}
Nesta seção será feita um resumo com os principais resultados técnicos obtidos através da simulação das usinas de $300$kW de eixo fixo e móvel. A \autoref{tab:rt300} mostra os resultados da simulação:
\begin{table}[h!]
	\centering
	\caption{Resultados técnicos obtidos no PV*SOL para a usina de $300$ kW em Belo Horizonte--MG, comparando estrutura fixa e rastreamento solar de um eixo, com energia anual gerada, rendimento específico e eficiência global do sistema.}
	\begin{tabular}{C{3cm}C{3cm}C{3cm}C{3cm}} \hline
		\textbf{Sistema} & \textbf{Energia anual [kWh]} & \textbf{Rendimento [kWh/kWp]} & \textbf{Eficiência [\%]} \\ \hline
Fixo & $457.813$ & $1.525,68$ & $74,90$ \\
\textit{Tracker} & $557.047$ & $1.856,46$  & $76,20$  \\ \hline
	\end{tabular}
	\fonteautor
	\label{tab:rt300}
\end{table}

Nas usinas de $300$ kW, a produção sobe de 457.813 kWh/ano no fixo para $557.047$ kWh/ano no \textit{tracker}. O ganho percentual em relação à geração é de aproximadamente $22$\%, mas aqui a melhoria no rendimento específico é ainda mais destacada: $1.525$ kWh/kWp no fixo contra $1.856$ kWh/kWp no \textit{tracker}. Além disso, a eficiência do sistema aumenta, passando de $74,9$\% para $76,2$\%.

\subsection{Resultados Financeiros}
Nesta seção será feita um resumo com os principais resultados financeiros obtidos através da simulação das usinas de $300$kW de eixo fixo e móvel. A \autoref{tab:rf300} mostra os resultados da simulação:
\begin{table}[h!]
	\centering
	\caption{Indicadores econômico-financeiros estimados no PV*SOL para a usina de $300$ kW em Belo Horizonte--MG, comparando estrutura fixa e rastreamento solar de um eixo: custo de geração, \textit{payback}, \ac{TIR} e \ac{ROI}.}
	\begin{tabular}{C{3cm}C{3cm}C{3cm}C{2.5cm}C{2.5cm}} \hline
		\textbf{Sistema} & \textbf{Custo de geração [R\$/kWh]} & \textbf{\textit{PayBack} [anos]} & \textbf{\ac{TIR} [\%]} & \textbf{\ac{ROI} [\%]} \\ \hline
Fixo & $0,1622$ & $5$ anos, $11$ meses & $18,31$ & $18,31$ \\
\textit{Tracker} & $0,2133$ & $7$ anos, $8$ meses & $12,24$ & $15,06$ \\ \hline
	\end{tabular}
	\fonteautor
	\label{tab:rf300}
\end{table}

Nas usinas de $300$ kW, o sistema de estrutura fixa apresenta vantagem econômica significativa em relação ao sistema com \textit{tracker}. O \textit{payback} do sistema fixo é de aproximadamente $5$ anos e $11$ meses, enquanto o sistema com \textit{tracker} apresenta um retorno mais prolongado, de cerca de $7$ anos e $8$ meses. O custo de geração de energia também se mostra inferior no sistema fixo, com valor médio de R\$ $0,1622$/kWh, frente a R\$ $0,2133$/kWh no sistema com \textit{tracker}.

Quanto aos indicadores de rentabilidade, o sistema fixo registra \ac{TIR} de $18,31$\% e \ac{ROI} de $18,31$\%, ambos substancialmente superiores aos obtidos pelo sistema com \textit{tracker}, que apresenta \ac{TIR} de $12,24$\% e \ac{ROI} de $15,06$\%. Assim, para a escala de $300$ kW, os resultados indicam de forma clara que o sistema de estrutura fixa é economicamente mais atrativo, apresentando menor custo de geração, retorno do investimento mais rápido e melhores indicadores financeiros globais.

\section{Usina de $500$kW}
\subsection{Resultados técnicos}
Nesta seção será feita um resumo com os principais resultados técnicos obtidos através da simulação das usinas de $500$kW de eixo fixo e móvel. A \autoref{tab:rt500} mostra os resultados da simulação:
\begin{table}[h!]
	\centering
	\caption{Resultados técnicos obtidos no PV*SOL para a usina de $500$ kW em Belo Horizonte--MG, comparando estrutura fixa e rastreamento solar de um eixo, com energia anual gerada, rendimento específico e eficiência global do sistema.}
	\begin{tabular}{C{3cm}C{3cm}C{3cm}C{3cm}} \hline
		\textbf{Sistema} & \textbf{Energia anual [kWh]} & \textbf{Rendimento [kWh/kWp]} & \textbf{Eficiência [\%]} \\ \hline
Fixo & $769.662$ & $1.539,20$ & $75,60$ \\ 
\textit{Tracker} & 935.237 & $1.870,36$  & $76,80$  \\ \hline
	\end{tabular}
	\fonteautor
	\label{tab:rt500}
\end{table}

Nas usinas de $500$ kW, observa-se que o sistema fixo atinge $769.662$ kWh/ano, enquanto o \textit{tracker} chega a $935.237$ kWh/ano, consolidando um incremento de $22$\%. O rendimento também cresce de $1.539$ kWh/kWp para $1.870$ kWh/kWp, com melhoria na eficiência, que passa de $75,6$\% para $76,8$\%.

\subsection{Resultados Financeiros}
Nesta seção será feita um resumo com os principais resultados financeiros obtidos através da simulação das usinas de $500k$W de eixo fixo e móvel. A \autoref{tab:rf500} mostra os resultados da simulação:
\begin{table}[h!]
	\centering
	\caption{Indicadores econômico-financeiros estimados no PV*SOL para a usina de $500$ kW em Belo Horizonte--MG, comparando estrutura fixa e rastreamento solar de um eixo: custo de geração, \textit{payback}, \ac{TIR} e \ac{ROI}.}
	\begin{tabular}{C{3cm}C{3cm}C{3cm}C{2.5cm}C{2.5cm}} \hline
		\textbf{Sistema} & \textbf{Custo de geração [R\$/kWh]} & \textbf{\textit{PayBack} [anos]} & \textbf{\ac{TIR} [\%]} & \textbf{\ac{ROI} [\%]} \\ \hline
Fixo & $0,1637$ & $5$ anos, $7$ meses & $17,37$ & $19,79$ \\ 
\textit{Tracker} & $0,1832$ & $6$ anos, $4$ meses & $15,14$ & $17,68$ \\ \hline
	\end{tabular}
	\fonteautor
	\label{tab:rf500}
\end{table}

Nas usinas de $500$ kW, o sistema de estrutura fixa mantém desempenho econômico superior em relação ao sistema com \textit{tracker}. O \textit{payback} do sistema fixo é de aproximadamente $5$ anos e $7$ meses, enquanto o sistema com \textit{tracker} apresenta um retorno mais longo, de cerca de $6$ anos e $4$ meses. O custo de geração de energia do sistema fixo também é inferior, com valor médio de R\$ $0,1637$/kWh, frente a R\$ $0,1832$/kWh no sistema com \textit{tracker}.

No que se refere aos indicadores de rentabilidade, o sistema fixo apresenta \ac{TIR} de $17,37$\% e \ac{ROI} de $19,79$\%, ambos superiores aos obtidos pelo sistema com \textit{tracker}, que registra \ac{TIR} de $15,14$\% e \ac{ROI} de $17,68$\%. Dessa forma, para a escala de $500$ kW, os resultados indicam que o sistema de estrutura fixa é economicamente mais atrativo, apresentando menor custo de geração, retorno mais rápido do investimento e melhores indicadores financeiros globais.

\section{Usina de $800$kW}
\subsection{Resultados técnicos}
Nesta seção será feita um resumo com os principais resultados técnicos obtidos através da simulação das usinas de $800$kW de eixo fixo e móvel. A \autoref{tab:rt800} mostra os resultados da simulação:
\begin{table}[h!]
	\centering
	\caption{Resultados técnicos obtidos no PV*SOL para a usina de $800$ kW em Belo Horizonte--MG, comparando estrutura fixa e rastreamento solar de um eixo, com energia anual gerada, rendimento específico e eficiência global do sistema.}
	\begin{tabular}{C{3cm}C{3cm}C{3cm}C{3cm}} \hline
		\textbf{Sistema} & \textbf{Energia anual [kWh]} & \textbf{Rendimento [kWh/kWp]} & \textbf{Eficiência [\%]} \\ \hline
Fixo & $1.202.618$ & $1.538,66$ & $73,84$ \\ 
\textit{Tracker} & $1.498.156$ & $1.872,61$  & $76,88$ \\ \hline
	\end{tabular}
	\fonteautor
	\label{tab:rt800}
\end{table}

Nas usinas de $800$ kW, a diferença é ainda mais expressiva: o sistema fixo gera $1.202.618$ kWh/ano, enquanto o \textit{tracker}) atinge $1.498.156$ kWh/ano, o que corresponde a cerca de $25$\% de aumento na produção. O rendimento específico cresce de $1.538$ kWh/kWp para $1.872$ kWh/kWp, acompanhado de uma elevação na eficiência global ($73,84$\% no fixo contra $76,88$\% no \textit{tracker}).

\subsection{Resultados Financeiros}
Nesta seção será feita um resumo com os principais resultados financeiros obtidos através da simulação das usinas de $800$kW de eixo fixo e móvel. A \autoref{tab:rf800} mostra os resultados da simulação:
\begin{table}[h!]
	\centering
	\caption{Indicadores econômico-financeiros estimados no PV*SOL para a usina de $800$ kW em Belo Horizonte--MG, comparando estrutura fixa e rastreamento solar de um eixo: custo de geração, \textit{payback}, \ac{TIR} e \ac{ROI}.}
	\begin{tabular}{C{3cm}C{3cm}C{3cm}C{2.5cm}C{2.5cm}} \hline
		\textbf{Sistema} & \textbf{Custo de geração [R\$/kWh]} & \textbf{\textit{PayBack} [anos]} & \textbf{\ac{TIR} [\%]} & \textbf{\ac{ROI} [\%]} \\ \hline
Fixo & $0,1427$ & $5$ anos, $2$ meses & $18,97$ & $20,81$ \\ 
\textit{Tracker} & $0,1487$ & $5$ anos, $1$ mês  & $19,47$ & $21,79$ \\ \hline
	\end{tabular}
	\fonteautor
	\label{tab:rf800}
\end{table}

Nas usinas de $800$ kW, observa-se uma maior proximidade entre os resultados econômicos das duas configurações analisadas. O \textit{payback} do sistema de estrutura fixa é de aproximadamente $5$ anos e $2$ meses, enquanto o sistema com \textit{tracker} apresenta um retorno ligeiramente mais rápido, de cerca de $5$ anos e $1$ mês. Em relação ao custo de geração de energia, o sistema fixo ainda apresenta vantagem, com valor médio de R\$ $0,1427$/kWh, frente a R\$ $0,1487$/kWh no sistema com \textit{tracker}.

No entanto, ao analisar os indicadores de rentabilidade, o sistema com \textit{tracker} passa a apresentar resultados superiores, registrando \ac{TIR} de $19,47$\% e \ac{ROI} de $21,79$\%, enquanto o sistema fixo apresenta \ac{TIR} de $18,97$\% e \ac{ROI} de $20,81$\%. Dessa forma, para a escala de 800 kW, os resultados indicam um cenário de equilíbrio econômico entre as configurações, no qual o sistema fixo mantém menor custo de geração, enquanto o uso de \textit{tracker} proporciona maior rentabilidade percentual, sinalizando uma transição na viabilidade econômica do rastreamento solar em potências mais elevadas.

\section{Usina de $1$MW}
\subsection{Resultados técnicos}
Nesta seção será feita um resumo com os principais resultados técnicos obtidos através da simulação das usinas de $1$MW de eixo fixo e móvel.
A \autoref{tab:rt1} mostra os resultados da simulação:
\begin{table}[h!]
	\centering
	\caption{Resultados técnicos obtidos no PV*SOL para a usina de $1$ MW em Belo Horizonte--MG, comparando estrutura fixa e rastreamento solar de um eixo, com energia anual gerada, rendimento específico e eficiência global do sistema.}
	\begin{tabular}{C{3cm}C{3cm}C{3cm}C{3cm}} \hline
		\textbf{Sistema} & \textbf{Energia anual [kWh]} & \textbf{Rendimento [kWh/kWp]} & \textbf{Eficiência [\%]} \\ \hline
Fixo & $1.538.740$ & $1.538,66$ & $75,60$ \\ 
\textit{Tracker} & $1.872.695$ & $1.872,61$  & $76,90$  \\ \hline
	\end{tabular}
	\fonteautor
	\label{tab:rt1}
\end{table}

Nas usinas de $1$MW, a produção do sistema fixo alcança $1.538.740$ kWh/ano, enquanto o \textit{tracker} chega a $1.872.695$ kWh/ano, representando novamente um ganho de $22$\%. O rendimento específico cresce de $1.538$ para $1.872$ kWh/kWp, com a eficiência global também sendo superior no rastreador ($76,9$\% contra $75,6$\% no fixo).

\subsection{Resultados Financeiros}
Nesta seção será feita um resumo com os principais resultados financeiros obtidos através da simulação das usinas de $1$MW de eixo fixo e móvel. A \autoref{tab:rf1} mostra os resultados da simulação:
\begin{table}[h!]
	\centering
	\caption{Indicadores econômico-financeiros estimados no PV*SOL para a usina de $1$ MW em Belo Horizonte--MG, comparando estrutura fixa e rastreamento solar de um eixo: custo de geração, \textit{payback}, \ac{TIR} e \ac{ROI}.}
	\begin{tabular}{C{3cm}C{3cm}C{3cm}C{2.5cm}C{2.5cm}} \hline
		\textbf{Sistema} & \textbf{Custo de geração [R\$/kWh]} & \textbf{\textit{PayBack} [anos]} & \textbf{\ac{TIR} [\%]} & \textbf{\ac{ROI} [\%]} \\ \hline
Fixo & $0,1251$ & $4$ anos, $6$ meses & $21,99$ & $23,74$ \\ 
\textit{Tracker} & $0,13$ & $4$ anos, $4$ meses  & $22,67$ & $24,91$ \\ \hline
	\end{tabular}
	\fonteautor
	\label{tab:rf1}
\end{table}

Nas usinas de $1$ MW, o sistema com estrutura fixa e o sistema com \textit{tracker} apresentam desempenho econômico bastante próximo, com leve vantagem para a configuração com rastreamento solar em alguns indicadores. O \textit{payback} do sistema fixo é de aproximadamente $4$ anos e $6$ meses, enquanto o sistema com \textit{tracker} apresenta um retorno ligeiramente mais rápido, de cerca de $4$ anos e $4$ meses. Quanto ao custo de geração de energia, o sistema fixo mantém valor inferior, com média de R$ $0,1251$/kWh, frente a R$ $0,13$/kWh no sistema com \textit{tracker}.

Em relação aos indicadores de rentabilidade, o sistema com \textit{tracker} apresenta \ac{TIR} de $22,67$\% e \ac{ROI} de $24,91$\%, superiores aos valores obtidos pelo sistema fixo, que registra \ac{TIR} de $21,99$\% e \ac{ROI} de $23,74$\%. Dessa forma, para a escala de $1$ MW, os resultados indicam que, apesar do ligeiro aumento no custo de geração, o uso de \textit{trackers} se torna economicamente mais atrativo, proporcionando maior retorno financeiro e menor tempo de recuperação do investimento.

\section{Resumo dos Resultados}
\subsection{Resumo Técnico das Usinas}

Com o intuito de facilitar a visualização e a consulta dos principais resultados técnicos obtidos nas simulações, a Tabela 20 apresenta um resumo dos parâmetros energéticos das usinas fotovoltaicas analisadas. Os dados estão organizados de forma a permitir uma comparação direta entre as diferentes potências instaladas e as configurações de sistema consideradas, reunindo informações referentes à energia anual gerada, rendimento específico e eficiência dos sistemas. Dessa forma, a \autoref{tab:resumot} atua como um instrumento de apoio às análises técnicas já realizadas e às discussões que serão aprofundadas nas seções subsequentes deste capítulo.

\begin{table}[h!]
\centering
\caption{Resumo dos resultados técnicos obtidos no PV*SOL para usinas fotovoltaicas de diferentes potências em Belo Horizonte--MG, comparando sistemas de estrutura fixa e com rastreamento solar de um eixo.}
\begin{tabular}{C{2cm}C{2cm}C{3cm}C{3cm}C{2cm}} \hline
\textbf{Potência} &\textbf{Sistema} & \textbf{Energia anual [kWh]} & \textbf{Rendimento [kWh/kWp]} & \textbf{Eficiência [\%]} \\ \hline
\multirow{2}{*}{$10$ kW} & Fixo & 14.523 & 1.376,15 & 67,60 \\ 
& \textit{Tracker} & 17.746 & 1.698,48 & 69,70 \\ \hline
\multirow{2}{*}{$50$ kW} & Fixo & 72.251 & 1.430,18 & 70,25 \\ 
& \textit{Tracker} & 88.295 & 1.751,08 & 71,89 \\ \hline
\multirow{2}{*}{$100$ kW} & Fixo & 144.501 & 1.430,18 & 70,25 \\ 
& \textit{Tracker} & 176.590 & 1.751,08 & 71,89 \\ \hline
\multirow{2}{*}{$300$ kW} & Fixo & 457.813 & 1.525,68 & 74,90 \\ 
& \textit{Tracker} & 557.047 & 1.856,46 & 76,20 \\ \hline
\multirow{2}{*}{$500$ kW} & Fixo & 769.662 & 1.539,20 & 75,61 \\ 
& \textit{Tracker} & 935.237 & 1.870,36 & 76,80 \\ \hline
\multirow{2}{*}{$800$ kW} & Fixo & 1.202.618 & 1.538,66 & 73,84 \\ 
& \textit{Tracker} & 1.498.156 & 1.872,61 & 76,88 \\ \hline
\multirow{2}{*}{$1$ MW} & Fixo & 1.538.740 & 1.538,66 & 75,58 \\ 
& \textit{Tracker} & 1.872.695 & 1.872,61 & 76,90 \\ \hline
\end{tabular}
\fonteautor
\label{tab:resumot}
\end{table}

\subsection{Resumo Financeiro das Usinas}
De maneira complementar, a Tabela 21 consolida os principais indicadores financeiros associados às usinas fotovoltaicas estudadas, com o objetivo de proporcionar uma visão geral e organizada dos resultados econômicos obtidos. A apresentação sintetizada dos custos, do tempo de retorno do investimento e dos indicadores de rentabilidade possibilita uma consulta mais ágil e eficiente aos dados, servindo como base para as análises comparativas e para a discussão dos resultados financeiros que serão desenvolvidas posteriormente. Assim a \autoref{tab:resumof} contribui para a clareza e a sistematização das informações econômicas do estudo.

\begin{table}[h!]
\centering
\caption{Resumo dos resultados econômico-financeiros estimados no PV*SOL para usinas fotovoltaicas de diferentes potências em Belo Horizonte--MG, comparando sistemas de estrutura fixa e com rastreamento solar de um eixo.}
\begin{tabular}{C{2cm}C{2cm}C{2cm}C{3cm}C{2cm}C{2cm}} \hline
\textbf{Potência} & \textbf{Sistema} & \textbf{Custo de geração [R\$/kWh]} & \textbf{\textit{PayBack} [anos]]} & \textbf{\ac{TIR} [\%]} & \textbf{\ac{ROI} [\%]} \\ \hline
\multirow{2}{*}{$10$ kW} & Fixo & 0,2198 & $7$ anos, $10$ meses & 11,91 & 14,26 \\
& \textit{Tracker} & 0,2331 & $8$ anos & 11,64 & 14,52 \\ \hline
\multirow{2}{*}{$50$ kW} & Fixo & 0,1961 & $7$ anos, $3$ meses & 13,07 & 15,30 \\
& \textit{Tracker} & 0,2340 & $8$ anos, $4$ meses & 11,00 & 13,95 \\ \hline
\multirow{2}{*}{$100$ kW} & Fixo & 0,1846 & $6$ anos, $9$ meses & 14,13 & 16,26 \\
& \textit{Tracker} & 0,2200 & $7$ anos, $10$ meses & 11,99 & 14,83 \\ \hline
\multirow{2}{*}{$300$ kW} & Fixo & 0,1622 & $5$ anos, $11$ meses & 16,34 & 18,31 \\
& \textit{Tracker} & 0,2133 & $7$ anos, $8$ meses & 12,24 & 15,06 \\ \hline
\multirow{2}{*}{$500$ kW} & Fixo & 0,1637 & $5$ anos, $7$ meses & 17,37 & 19,79 \\
& \textit{Tracker} & 0,1832 & $6$ anos, $4$ meses & 15,14 & 17,68 \\ \hline
\multirow{2}{*}{$800$ kW} & Fixo & 0,1427 & $5$ anos, $2$ meses & 18,97 & 20,81 \\
& \textit{Tracker} & 0,1487 & $5$ anos, $1$ mês & 19,47 & 21,79 \\ \hline
\multirow{2}{*}{$1$ MW} & Fixo & 0,1251 & $4$ anos, $6$ meses & 21,99 & 23,74 \\
& \textit{Tracker} & 0,1300 & $4$ anos, $4$ meses & 24,91 & 24,08 \\ \hline
\end{tabular}
\fonteautor
\label{tab:resumof}
\end{table}

\section{Análise geral}
%Nesta seção são apresentados os principais resultados consolidados do estudo, abordando tanto os aspectos técnicos quanto os financeiros dos sistemas fotovoltaicos avaliados. A análise técnica contempla indicadores como a energia anual gerada, o rendimento específico e a eficiência global, discutindo o comportamento desses parâmetros em função da potência instalada e do tipo de tecnologia adotada (sistema fixo ou com rastreamento). Já a análise financeira foca em métricas como custo nivelado de geração (R\$/kWh), \textit{payback}, taxa interna de retorno (\ac{TIR}) e retorno sobre o investimento (\ac{ROI}), permitindo verificar a viabilidade econômica e comparar o desempenho entre as diferentes configurações estudadas. Dessa forma, busca-se oferecer uma visão integrada que relacione ganhos de desempenho energético com os impactos nos custos de implantação e retorno financeiro, apontando em que situações a utilização de \textit{trackers} passa a ser mais vantajosa.
Nesta seção realiza-se a síntese integrada dos resultados obtidos, articulando os desempenhos energético e econômico dos sistemas fotovoltaicos avaliados. A análise considera, de forma conjunta, a variação dos indicadores de geração e eficiência com a escala de potência e com a tecnologia empregada, bem como seus desdobramentos nos custos e na rentabilidade dos projetos. Essa abordagem permite estabelecer a relação entre o ganho energético proporcionado pelo rastreamento solar e o investimento adicional requerido, evidenciando como esse equilíbrio evolui ao longo das diferentes faixas de potência. Desse modo, busca-se identificar as condições em que o aumento de produção se traduz efetivamente em vantagem econômica, caracterizando o ponto a partir do qual o uso de \textit{trackers} passa a apresentar maior atratividade no contexto analisado.

\subsection{Aspectos técnicos}
A \autoref{fig:img012} mostra relação entre energia anual injetada  e o rendimento específico para sistemas fotovoltaicos com estrutura fixa e com rastreamento solar de um eixo em Belo Horizonte--MG. A avaliação técnica evidencia que os sistemas com rastreamento de um eixo (vermelho) apresentam desempenho energético superior aos sistemas de estrutura fixa (azul) em todas as potências analisadas. Esse comportamento decorre da maior captação de radiação ao longo do dia, proporcionada pelo acompanhamento do movimento aparente do sol, o que amplia o período de operação em condições próximas ao ângulo ótimo de incidência.

A mesma tendência é observada no rendimento específico (kWh/kWp), indicando maior aproveitamento da potência instalada nos sistemas com \textit{tracker} (amarelo) em relação ao fixo (cinza). A capacidade de ajuste angular reduz perdas por desalinhamento entre a superfície dos módulos e a radiação incidente, enquanto nos sistemas fixos o ângulo constante limita o aproveitamento energético sobretudo nos períodos matutino e vespertino.
% A análise da energia anual gerada evidencia que os sistemas com rastreamento de um eixo apresentam desempenho superior em comparação aos sistemas fixos em todas as potências avaliadas. Esse comportamento se deve à maior captação da radiação solar proporcionada pelo movimento do \textit{tracker}, que acompanha o deslocamento aparente do sol ao longo do dia.
% A avaliação técnica evidencia que os sistemas com rastreamento de um eixo apresentam desempenho energético superior aos sistemas de estrutura fixa em todas as potências analisadas. Esse comportamento decorre da maior captação de radiação ao longo do dia, proporcionada pelo acompanhamento do movimento aparente do sol, o que amplia o período de operação em condições próximas ao ângulo ótimo de incidência.
% Quando se observa o rendimento específico (kWh/kWp), a mesma tendência é confirmada: o sistema com \textit{tracker} demonstra maior aproveitamento energético por unidade de potência instalada. Esse comportamento está diretamente associado à redução das perdas angulares, uma vez que o seguidor solar mantém os módulos mais próximos da inclinação ideal durante o dia. Em contrapartida, no sistema fixo, a limitação do ângulo de inclinação reduz o aproveitamento da radiação em horários de baixa incidência solar.  A \autoref{fig:img012} nos mostra na relação entre as tres grandezas mencionadas. 
\begin{figure}[h]
	\centering
	\caption{Relação entre energia anual injetada e rendimento específico para sistemas fotovoltaicos com estrutura fixa e com rastreamento solar de um eixo em Belo Horizonte--MG.}
	%\caption{Energia anual gerada x Potencia x Rendimento.}
	\includegraphics[width=0.8\linewidth]{./textuais/figs/graficoenergiaanualxpotenciaxrendimento}
    \fonteautor
	\label{fig:img012}
\end{figure}

\color{red}
CONTRADITÓRIO

O gráfico apresentado na \autoref{fig:img013} mostra o ganho percentual de geração proporcionado pelo rastreamento solar em função da potência instalada. Observa-se que o incremento relativo de energia permanece aproximadamente constante ao longo das diferentes faixas de potência analisadas, com variações pontuais associadas às condições de simulação e às configurações dos sistemas modelados. Esse comportamento pode indicar que o benefício técnico do rastreamento está predominantemente relacionado ao aumento do tempo de captação eficiente da radiação solar, e não a efeitos de escala intrínsecos à potência instalada.

Dessa forma, o rastreamento solar não deve ser interpretado como uma tecnologia cujo ganho percentual cresce com o porte do sistema, mas como uma solução que eleva de maneira semelhante o desempenho energético relativo em diferentes escalas. Consequentemente, embora o ganho percentual permaneça quase invariável, o ganho absoluto de energia torna-se mais expressivo em sistemas de maior potência, ampliando seu impacto prático.

Nesse contexto, a análise puramente técnica mostra-se insuficiente para determinar a viabilidade do uso de \textit{trackers}, tornando necessária a avaliação dos reflexos econômicos associados a esse ganho energético. Assim, na sequência, são analisados os aspectos financeiros dos sistemas estudados, visando identificar em que condições o aumento de geração proporcionado pelo rastreamento solar se traduz em vantagens econômicas efetivas.
\begin{figure}[htb!]
	\centering
	\caption{Ganho percentual de geração de energia anual de sistemas fotovoltaicos com rastreamento solar de um eixo em relação a sistemas fixos, em função da potência instalada, para as condições de Belo Horizonte--MG.}
	\includegraphics[width=0.65\linewidth]{./textuais/figs/ganhodegeracaoxpotencia}
    \fonteautor
	\label{fig:img013}
\end{figure}

No que se refere à eficiência global do sistema, observa-se também vantagem para os projetos com rastreamento. Isso se deve ao fato de que, além da maior captura de energia, o \textit{tracker} contribui para minimizar sombreamentos e melhorar o fator de utilização dos módulos. Contudo, é importante destacar que esse ganho de eficiência não ocorre de forma isolada, pois está diretamente associado a um aumento no custo de capital (\ac{CAPEX}). Em outras palavras, a maior produção de energia proporcionada pelo rastreamento só é viável mediante um investimento inicial mais elevado, decorrente da necessidade de estruturas móveis, sistemas de acionamento, componentes adicionais e maior demanda de manutenção preventiva. Para ilustrar essa relação entre desempenho e investimento, a \autoref{fig:img014} apresenta a comparação entre o \ac{CAPEX} e a eficiência global dos sistemas fixos e com rastreamento ao longo das diferentes potências analisadas, evidenciando o \textit{trade-off} existente entre maior rendimento energético e maior custo de implantação.
\begin{figure}[h]
	\centering
	\caption{Relação entre o investimento inicial (\ac{CAPEX}) e a eficiência global de sistemas fotovoltaicos de estrutura fixa e com rastreamento solar de um eixo, em função da potência instalada, para as condições de Belo Horizonte--MG.}
	\includegraphics[width=0.8\linewidth]{./textuais/figs/capexeficiencia}
    \fonteautor
	\label{fig:img014}
\end{figure}

\subsection{Aspectos Financeiros}
No âmbito econômico, os resultados mostram que o custo de geração (R\$/kWh) é diretamente influenciado pelo investimento inicial. Apesar de gerar mais energia, o sistema com rastreamento possui um custo de implantação mais elevado, o que tende a aumentar o custo nivelado de energia em sistemas de menor porte.
\begin{figure}[h]
	\centering
	\caption{Relação entre o \ac{CAPEX} e o custo de geração de energia (R\$/kWh) de sistemas fotovoltaicos fixos e com rastreamento de um eixo em função da potência instalada, para Belo Horizonte--MG.}
	\includegraphics[width=0.8\linewidth]{./textuais/figs/capexxcusto}
    \fonteautor
	\label{fig:img015}
\end{figure}

O comportamento ilustrado no Gráfico na \autoref{fig:img015}  evidencia a relação entre o investimento inicial e o custo de geração de energia para sistemas fotovoltaicos fixos e com rastreamento solar. Observa-se que o \ac{CAPEX} dos sistemas com \textit{tracker} apresenta crescimento mais acentuado em comparação aos sistemas fixos, refletindo os custos adicionais associados à aquisição, instalação e integração dos mecanismos de rastreamento.

Ao analisar o custo de geração de energia, verifica-se que, embora os sistemas com rastreamento apresentem uma redução progressiva desse indicador à medida que a potência instalada aumenta, o custo de geração permanece superior ao dos sistemas fixos na faixa de potência considerada neste estudo. Isso indica que, nas condições avaliadas, o ganho energético proporcionado pelo \textit{tracker} não foi suficiente para compensar integralmente o maior investimento inicial.

Ainda assim, a tendência de aproximação entre as curvas sugere que, em potências instaladas superiores às analisadas ou sob diferentes condições de custo e operação, pode ocorrer o cruzamento entre os custos de geração dos sistemas fixos e rastreados. Tal comportamento reforça o caráter dependente de escala da viabilidade dos sistemas com rastreamento solar, especialmente em projetos de maior porte.

Já o Gráfico apresentado na \autoref{fig:img016} ilustra de forma integrada a evolução do \textit{payback} e da Taxa Interna de Retorno (\ac{TIR}) em função da potência instalada, comparando sistemas fotovoltaicos de estrutura fixa e com rastreamento solar. Inicialmente, observa-se que, em sistemas de menor porte ($10$ a $100$ kW), os sistemas fixos apresentam melhor desempenho financeiro. Nessa faixa, o \textit{payback} dos sistemas fixos é sistematicamente inferior ao dos sistemas com \textit{tracker}, enquanto a \ac{TIR} também se mostra superior, indicando que o aumento do \ac{CAPEX} associado ao rastreamento ainda não é compensado pelo ganho adicional de geração.
\begin{figure}[h]
	\centering
	\caption{Relação entre o \textit{payback} e a \ac{TIR} de sistemas fotovoltaicos fixos e com rastreamento de um eixo em função da potência instalada, para Belo Horizonte--MG.}
	\includegraphics[width=0.8\linewidth]{./textuais/figs/paybackxtirxpotencia}
    \fonteautor
	\label{fig:img016}
\end{figure}

À medida que a potência instalada aumenta, nota-se uma redução progressiva do \textit{payback} em ambas as configurações, porém com comportamento mais acentuado nos sistemas com rastreamento. Na faixa intermediária ($300$ a $500$ kW), embora o sistema fixo ainda apresente menor tempo de retorno, a diferença entre as duas soluções diminui significativamente. Paralelamente, a \ac{TIR} dos sistemas com \textit{tracker} cresce de forma consistente, aproximando-se dos valores observados nos sistemas fixos, evidenciando um cenário de transição na viabilidade econômica.

Em sistemas de maior porte, a partir de aproximadamente $800$ kW, ocorre o cruzamento das curvas de \textit{payback} e de \ac{TIR}. Nessa faixa, os sistemas com rastreamento passam a apresentar \textit{payback} inferior ao dos sistemas fixos, além de \ac{TIR} superior, demonstrando que o investimento adicional requerido pelos \textit{trackers} é compensado pela maior eficiência energética e pelo aumento da geração anual. Esse comportamento caracteriza uma mudança clara na atratividade financeira, consolidando o rastreamento como a alternativa economicamente mais vantajosa em empreendimentos de maior escala.

Dessa forma, os resultados evidenciam que a viabilidade do uso de \textit{trackers} está diretamente associada à potência instalada do sistema. Enquanto em projetos de pequeno porte a solução fixa se mostra mais racional sob o ponto de vista financeiro, em usinas de médio a grande porte o rastreamento solar passa a oferecer melhores indicadores econômicos, reforçando a importância da análise conjunta dos aspectos técnicos e financeiros, conforme discutido nas seções seguintes.

A \autoref{fig:img017} apresenta o gráfico de bolhas, que integra em uma única visualização três indicadores financeiros fundamentais para a avaliação dos sistemas fotovoltaicos: o \textit{payback}, representado no eixo horizontal, o \ac{TIR}, no eixo vertical, e o\ac{ROI}, indicado pelo tamanho das bolhas. Essa abordagem permite comparar, de forma direta, o desempenho econômico de sistemas fixos e com rastreamento ao longo das diferentes faixas de potência analisadas.
\begin{figure}[h]
	\centering
	\caption{\textit{Payback} x \ac{TIR} x \ac{ROI}.}
	\includegraphics[width=0.8\linewidth]{./textuais/figs/bolhas}
    \fonteautor
	\label{fig:img017}
\end{figure}

Observa-se que, nas potências menores, entre 10 kW e 300 kW, os sistemas de estrutura fixa apresentam maior atratividade financeira. Nessa faixa, o \textit{payback} dos sistemas fixos é sistematicamente menor, enquanto a \ac{TIR} e o \ac{ROI} também se mantêm superiores aos dos sistemas com \textit{tracker}. Esse comportamento indica que, em projetos de pequeno porte, o aumento de geração proporcionado pelo rastreamento ainda não é suficiente para compensar o maior investimento inicial associado à tecnologia, tornando o sistema fixo a alternativa economicamente mais racional.

À medida que a potência instalada aumenta, especialmente na faixa intermediária (300 a 500 kW), percebe-se uma redução progressiva da diferença entre os dois arranjos. Embora o sistema fixo ainda apresente ligeira vantagem em termos de \textit{payback} e \ac{TIR}, os indicadores do \textit{tracker} se aproximam de forma significativa, evidenciando um cenário de transição. Nessa etapa, a escolha entre as configurações passa a depender mais do perfil do investidor e da estratégia adotada, já que o rastreamento começa a se mostrar tecnicamente competitivo, ainda que não plenamente dominante do ponto de vista financeiro.

Nas potências mais elevadas, especialmente a partir de 800 kW, ocorre o cruzamento claro das curvas financeiras, indicando uma mudança no cenário de viabilidade. O sistema com \textit{tracker} passa a apresentar \textit{payback} menor, além de \ac{TIR} e \ac{ROI}superiores aos do sistema fixo. Esse comportamento demonstra que, em usinas de maior porte, o custo adicional do rastreamento é diluído pelo ganho expressivo de geração de energia, resultando em maior rentabilidade e retorno mais rápido do capital investido.

Dessa forma, o gráfico evidencia que a atratividade do rastreamento solar está fortemente associada à escala do empreendimento. Enquanto sistemas de pequeno porte favorecem soluções fixas, projetos de médio porte representam uma zona de equilíbrio entre as tecnologias, e usinas de grande porte consolidam o \textit{tracker} como a opção economicamente mais vantajosa, reforçando sua relevância estratégica para empreendimentos com maior potência instalada.

\section{Conclusões Parciais}
De maneira geral, os resultados mostraram que os sistemas com rastreamento solar sempre apresentam melhor desempenho técnico, com maior geração de energia, rendimento específico e eficiência global em todas as potências analisadas. Porém, do ponto de vista financeiro, o comportamento é diferente: nas potências menores (até $300$ kW), os sistemas fixos se mostraram mais vantajosos, com menor custo de geração, \textit{payback} mais curto e melhores índices de \ac{TIR} e \ac{ROI}. A partir de $500$ kW, o cenário muda, e os trackers começam a apresentar resultados equivalentes, consolidando-se como a opção mais atrativa nas potências maiores ($800$ kW e $1$ MW), onde o ganho energético compensa plenamente o investimento adicional.

\color{magenta}
\section*{Inconsistências numéricas}

\begin{itemize}
	\item \textbf{Ganho percentual (22\% versus 25\%) não consistente ao longo do capítulo.}\\
	Trechos: 
	``mantendo a tendência de 22\% de ganho'' (100 kW) 
	vs ``corresponde a cerca de 25\% de aumento na produção'' (800 kW) 
	vs ``o incremento relativo de energia permanece aproximadamente constante'' (análise da Figura do ganho percentual).\\
	Inconsistência: o texto sustenta ganho percentual aproximadamente constante (em torno de 22\%), mas reporta 25\% em 800 kW. O autor deve revisar os dados e/ou justificar tecnicamente a exceção.
	
	\item \textbf{Resumo financeiro diverge das tabelas individuais no caso 10 kW (TIR do tracker).}\\
	Trechos: Tabela 10 kW (individual): ``Tracker ... TIR = 11,94'' 
	vs Tabela resumo financeiro: ``Tracker ... TIR = 11,64'' (10 kW).\\
	Inconsistência: mesma grandeza (TIR do caso 10 kW com tracker) aparece com valores diferentes. O autor deve revisar a consolidação.
	
	\item \textbf{Resumo financeiro diverge das tabelas individuais no caso 300 kW (TIR do fixo).}\\
	Trechos: Tabela 300 kW (individual): ``Fixo ... TIR = 18,31'' 
	vs Tabela resumo financeiro: ``Fixo ... TIR = 16,34'' (300 kW).\\
	Inconsistência: TIR do sistema fixo em 300 kW muda entre tabela individual e resumo. O autor deve revisar a transcrição/exportação.
	
	\item \textbf{Resumo financeiro diverge do texto e da tabela individual no caso 1 MW (tracker: TIR e ROI).}\\
	Trechos: Tabela 1 MW (individual) e texto: ``TIR = 22,67'' e ``ROI = 24,91'' 
	vs Tabela resumo financeiro: ``TIR = 24,91'' e ``ROI = 24,08'' (1 MW, tracker).\\
	Inconsistência: no resumo, TIR e ROI do tracker em 1 MW não batem com a tabela individual e com o texto (há forte indício de troca/erro de lançamento). O autor deve revisar.
	
	\item \textbf{Resumo técnico diverge da tabela individual no caso 100 kW (eficiência do fixo).}\\
	Trechos: Tabela 100 kW (individual): ``Fixo ... Eficiência = 70,30'' 
	vs Tabela resumo técnico: ``Fixo ... Eficiência = 70,25'' (100 kW).\\
	Inconsistência: a eficiência do sistema fixo em 100 kW aparece com dois valores. O autor deve revisar e padronizar arredondamento.
	
	\item \textbf{Resumo técnico diverge da tabela individual no caso 10 kW (eficiência do tracker).}\\
	Trechos: Tabela 10 kW (individual): ``Tracker ... Eficiência = 69,73'' 
	vs Tabela resumo técnico: ``Tracker ... Eficiência = 69,70'' (10 kW).\\
	Inconsistência: divergência pequena, mas deve haver consistência de arredondamento/lançamento.
	
	\item \textbf{Resumo técnico diverge das tabelas individuais nos casos 500 kW e 1 MW (eficiência do fixo).}\\
	Trechos: 500 kW (individual): ``Fixo ... Eficiência = 75,60'' 
	vs resumo técnico: ``Fixo ... Eficiência = 75,61''.\\
	Trechos: 1 MW (individual): ``Fixo ... Eficiência = 75,60'' 
	vs resumo técnico: ``Fixo ... Eficiência = 75,58''.\\
	Inconsistência: diferenças pequenas, mas afetam rastreabilidade. O autor deve revisar os valores originais e o critério de arredondamento.
	
	\item \textbf{Legenda (caption) da tabela financeira de 50 kW está factualmente errada.}\\
	Trecho: na seção 50 kW, a tabela financeira está descrita como ``para a usina de 10 kW''.\\
	Inconsistência: a legenda não corresponde à seção/tabela (50 kW). O autor deve corrigir para evitar erro factual de documentação.
	
	\item \textbf{Inconsistência de notação numérica na tabela técnica de 500 kW (formatação).}\\
	Trecho: na tabela técnica de 500 kW, aparece ``Tracker ... 935.237'' sem o mesmo padrão de marcação usado nos demais valores (por exemplo, com delimitadores e símbolo de modo matemático).\\
	Inconsistência: pode ser apenas formatação, mas pode mascarar erro de transcrição. O autor deve revisar o valor e padronizar a apresentação.
\end{itemize}


\section*{Contradições técnicas}

\begin{itemize}
	\item \textbf{Conclusão parcial generaliza “melhores índices de TIR e ROI” para fixo até 300 kW, mas 10 kW contradiz.}\\
	Trecho (Conclusões Parciais): ``até 300 kW ... melhores índices de TIR e ROI''.\\
	Trecho (10 kW): ``Fixo ... TIR 11,91\% e ROI 14,26\%'' e ``Tracker ... TIR 11,94\% e ROI 14,52\%''.\\
	Contradição: em 10 kW, TIR e ROI do tracker são maiores. Sugestão: tornar a afirmação tecnicamente direta, especificando quais métricas favorecem cada configuração por faixa de potência, sem generalizar.
	
	\item \textbf{Afirmação de ganho percentual “quase invariável” conflita com o próprio caso de 800 kW.}\\
	Trechos: ``incremento relativo permanece aproximadamente constante'' 
	vs ``cerca de 25\%'' (800 kW).\\
	Incoerência: se 800 kW é exceção, isso precisa ser dito e tecnicamente sustentado; caso contrário, os dados devem ser revisados. Sugestão: explicitar objetivamente se há exceção e por quê.
	
	\item \textbf{Uso de “vantagem” e “mais atrativo” com critérios diferentes no mesmo bloco (métricas conflitantes).}\\
	Trechos (800 kW): ``o sistema fixo ainda apresenta vantagem'' (custo de geração menor) 
	e ``tracker ... maior rentabilidade percentual'' (TIR/ROI maiores) 
	e ``cenário de equilíbrio econômico''.\\
	Incoerência: “vantagem” muda de critério sem aviso (custo de geração versus TIR/ROI versus payback). Sugestão: ser tecnicamente direto e declarar que as métricas divergem e que a conclusão depende do critério adotado.
	
	\item \textbf{Causalidade potencialmente frágil: “tracker minimiza sombreamentos”.}\\
	Trecho: ``o tracker contribui para minimizar sombreamentos''.\\
	Incoerência técnica potencial: rastreamento pode reduzir ou aumentar sombreamento dependendo de layout, espaçamento, backtracking e geometria. Sem declarar as premissas do modelo (se houve backtracking e como foi o arranjo), a afirmação fica ambígua. Sugestão: explicitar o que foi configurado na simulação antes de atribuir causa.
\end{itemize}


\section*{Repetições}

\begin{itemize}
	\item \textbf{Aberturas de seção repetidas em todas as potências, com mudança apenas dos números.}\\
	Trechos recorrentes: ``Nesta seção será feita um resumo... A Tabela X mostra...'' aparece em todas as subseções técnicas e financeiras.\\
	Ideia repetida: introdução padronizada que não agrega informação nova. Sugestão: condensar com uma frase padrão no início do capítulo ou reduzir essas introduções.
	
	\item \textbf{Mesma narrativa por faixas de potência repetida em dois gráficos diferentes.}\\
	Trechos: descrição do gráfico payback e TIR (pequeno: fixo melhor; intermediário: aproximação; grande: tracker melhor) 
	e descrição do gráfico de bolhas repete a mesma segmentação e conclusão.\\
	Ideia repetida: mesma análise qualitativa em duplicidade. Sugestão: condensar e usar um dos gráficos como síntese, referenciando o outro apenas como reforço.
	
	\item \textbf{Explicação física do ganho do tracker repetida em sequência curta.}\\
	Trechos: ``maior captação ao longo do dia'' + ``reduz perdas por desalinhamento'' + ``amplia o período de operação próximo ao ângulo ótimo''.\\
	Ideia repetida: mesma causa física (melhor alinhamento angular) descrita várias vezes. Sugestão: condensar em um único bloco explicativo.
\end{itemize}


\section*{Problemas de clareza técnica}

\begin{itemize}
	\item \textbf{Métrica “Eficiência global do sistema” não é definida.}\\
	Trechos: coluna ``Eficiência [\%]'' em todas as tabelas técnicas e discussão de “ganho de eficiência”.\\
	Problema: não fica claro se é PR, eficiência global do software, ou outra métrica; isso afeta interpretação e comparabilidade entre potências.
	
	\item \textbf{“Custo de geração (R\$/kWh)” não tem premissas explícitas.}\\
	Trechos: tabelas financeiras e análise de custo de geração.\\
	Problema: não está claro horizonte, taxa de desconto, vida útil, degradação, OPEX incluído, reposições (por exemplo, inversores), e se há atualização monetária. Sem isso, “custo de geração” fica pouco rastreável.
	
	\item \textbf{Figura CAPEX x Eficiência contém duas legendas.}\\
	Trecho: aparecem duas linhas de legenda para a mesma figura (uma curta e outra longa).\\
	Problema: pode causar erro de compilação/numeração e dificulta referência consistente.
	
	\item \textbf{Marca editorial “CONTRADITÓRIO” dentro do texto.}\\
	Trecho: aparece uma anotação em vermelho “CONTRADITÓRIO”.\\
	Problema: indica pendência não resolvida e quebra o fluxo científico do capítulo. Deve ser removida após revisão, ou substituída por discussão objetiva (com revisão dos dados).
	
	\item \textbf{Afirmação de “cruzamento” das curvas sem explicitar critério numérico.}\\
	Trecho: ``a partir de aproximadamente 800 kW, ocorre o cruzamento das curvas de payback e de TIR''.\\
	Problema: como os pontos são discretos (800 kW e 1 MW), não fica claro se houve interpolação ou se o “cruzamento” é apenas comparação ponto a ponto. Seria importante indicar explicitamente em quais potências cada métrica muda de sinal (payback e TIR) e se ocorre simultaneamente.
	
	\item \textbf{Notação monetária inconsistente (R\$, R\$).}\\
	Trechos: ``R\$ 0,2198/kWh'' em um ponto e ``R\$ 0,1251/kWh'' em outro.\\
	Problema: pode gerar erro tipográfico e confusão na leitura, além de dificultar padronização do documento.
\end{itemize}

\color{black}
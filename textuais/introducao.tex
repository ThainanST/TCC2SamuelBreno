\chapter{Introdução}

Este capítulo faz uma introdução sobre conceitos e assuntos introdutórios que envolvem energia solar, mais especificamente \ac{FS}. É feita uma breve contextualização, análise do cenário atual no Brasil. Além disso, é feita uma revisão bibliográfica, apresentação do problema e contribuições do trabalho.

A crise climática tem se tornado uma das maiores preocupações da sociedade contemporânea, impulsionando a busca por soluções energéticas sustentáveis que reduzam as emissões de gases de efeito estufa \cite{ipcc2021}. O aumento da temperatura global, eventos climáticos extremos e a crescente pressão por políticas ambientais rigorosas têm levado governos e setores produtivos a investirem em fontes renováveis de energia, reduzindo a dependência de combustíveis fósseis \cite{cresesbe2025}. Neste contexto, a energia solar fotovoltaica surge como uma alternativa promissora para mitigar os impactos ambientais da geração elétrica tradicional.

Além dos desafios ambientais, o custo da eletricidade tem sido um fator crítico para consumidores e indústrias em todo o mundo. A volatilidade nos preços da energia elétrica, impulsionada por fatores como a escassez hídrica e oscilações nos preços dos combustíveis fósseis, tem aumentado a demanda por soluções que garantam maior previsibilidade e eficiência econômica na geração elétrica \cite{epe2022}. No Brasil, onde grande parte da matriz energética é baseada em hidrelétricas, períodos de seca podem resultar em aumentos tarifários significativos, tornando a diversificação da matriz uma necessidade estratégica \cite{camargos2016analise}.

A energia solar fotovoltaica se destaca nesse cenário por sua abundância, baixo impacto ambiental e viabilidade técnica em diversas regiões do mundo. Nos últimos anos, os avanços tecnológicos e a redução dos custos dos módulos fotovoltaicos tornaram essa fonte cada vez mais competitiva, viabilizando sua adoção em larga escala \cite{lima2024evoluccao}. Além disso, políticas de incentivos e mecanismos regulatórios têm impulsionado investimentos no setor, consolidando a energia solar como uma solução viável para suprir a crescente demanda energética de forma sustentável \cite{lucas2024influencia}. A \autoref{fig:img001}, mostra o quão importante a energia solar vem se tornando na matriz energética brasileira.

\begin{figure}[h]
	\centering
	\caption{Crescimento da Energia Solar no Brasil de 2013 a 2024.}
	\includegraphics[width=0.6\linewidth]{./textuais/figs/img001}
    \fonteretirado{absolar2024}
	\label{fig:img001}
\end{figure}

O Brasil possui um dos maiores potenciais solares do mundo, com uma irradiação média superior a $5$ kWh/m²/dia na maior parte do território \cite{pereira2021energia}. Esse potencial, aliado a políticas de fomento e incentivos à geração distribuída, tem acelerado a expansão da energia solar no país, que já representa uma parcela significativa da matriz elétrica nacional \cite{resolucaoaneel}.A \autoref{fig:fig002} nos mostra o potencial de irradiação solar no Brasil.

\begin{figure}[h]
	\centering
	\caption{Radiação solar no Brasil, plano inclinado - média anual.}
	\includegraphics[width=0.6\linewidth]{./textuais/figs/img002}
    \fonteretirado{absolar2024}
	\label{fig:fig002}
\end{figure}

No entanto, para que o setor alcance sua máxima eficiência e competitividade, torna-se essencial a adoção de tecnologias que otimizem a captação da radiação solar, aumentando a geração de energia sem a necessidade de ampliação da área ocupada pelos sistemas.

Nesse contexto, os sistemas de rastreamento solar \textit{(trackers)} surgem como uma alternativa tecnológica capaz de maximizar a captação da luz solar ao longo do dia, aumentando a eficiência da geração fotovoltaica em comparação aos sistemas fixos. Estudos indicam que os \textit{trackers} podem elevar a geração de energia em até $25$\% para sistemas de eixo único e até $40$\% para sistemas de duplo eixo, dependendo da localização e das condições climáticas \cite{denholm2021challenges}. No entanto, essa tecnologia também implica custos adicionais de instalação, operação e manutenção, o que levanta um questionamento central sobre sua viabilidade financeira em diferentes escalas de potência.

Assim, este trabalho tem como objetivo analisar a viabilidade econômica do uso de \textit{trackers} em sistemas fotovoltaicos, buscando identificar a partir de qual potência instalada essa tecnologia se torna financeiramente vantajosa. Para isso, serão consideradas variáveis como custos de implementação, ganhos energéticos e métricas financeiras, como Retorno do Investimento \textit{(ROI)}, Taxa Interna de Retorno \ac{TIR} e \textit{Payback}. A análise contribuirá para a tomada de decisão no setor fotovoltaico, auxiliando investidores e projetistas a otimizar seus projetos com base em critérios técnicos e econômicos.

%\section{Considerações iniciais} \label{sec:introConsideracoesIniciais}

%\section{Problema de Pesquisa}


%\section{Estado da Arte}



%\textcolor{red}{Nesta seção você fará uma revisão dos principais trabalhos, principalmente artigos, relacionado ao seu. É necessário destacar os objetivos, contribuições, avanços e principalmente lacunas. Use um parágrafo para cada autor/trabalho. Faça uma revisão de pelo menos três.}

%\section{Objetivos}


%\section{Justificativa}





%A \autoref{tab:fatores_escala} mostra os dados \cite{dantas2018viabilidade}. O trabalho de \citeonline{dantas2018viabilidade} mostra...
%\begin{table}[h]
%	\centering
%	\caption{Fatores de escala do controlador %\textit{fuzzy}.}
%	\begin{tabular}{c|c|c}
%		\hline
%		Descrição & Valor                & Unidade\\ %\hline
%		$GE^{-1}$ & $1.5 \cdot 10^{3}$   & adm \\
%		$GC^{-1}$ & $1.0 \cdot 10^{-1}$  & adm\\
%		$GU$      & $1.0 \cdot 10^{6}$   & adm \\
%		$T_s$     & $50.0 \cdot 10^{-6}$ & adm \\ 
%		\hline
%	\end{tabular}
%	\label{tab:fatores_escala}
%	\fonteadaptado{lamberts2013eficiencia}
%\end{table}


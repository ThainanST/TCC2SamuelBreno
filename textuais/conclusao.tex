\chapter{Conclusão}\label{cap:capitulo_5}


Este trabalho teve como objetivo avaliar a viabilidade técnica e financeira da utilização de sistemas fotovoltaicos fixos e com rastreamento solar de um eixo em diferentes portes de usinas, variando de 10 kW até 1 MW. Para isso, foram realizadas simulações no software PV*SOL, ferramenta amplamente utilizada no setor fotovoltaico, o que confere confiabilidade e consistência aos resultados obtidos.

As análises mostraram que os sistemas com rastreamento apresentam maior geração de energia em todas as potências estudadas, com ganhos médios de $20$ a $25$\% em relação aos sistemas fixos. Contudo, o ponto-chave está na viabilidade econômica. Em sistemas de menor porte (até 300 kW), os fixos se mostraram mais atrativos financeiramente, já que possuem menor custo de geração, \textit{payback} mais curto e melhores indicadores de \ac{TIR} e \ac{ROI}.

A partir da faixa de 500 kW, os resultados passam a indicar equilíbrio entre as duas configurações, e em potências acima de 800 kW o uso de \textit{trackers} se torna claramente mais vantajoso, apresentando menor tempo de retorno e índices financeiros superiores, além de manter o ganho técnico de produção de energia.

Dessa forma, conclui-se que a adoção de rastreadores solares de um eixo é recomendada principalmente para usinas de grande porte (maior que $800$ kW), onde o investimento adicional é compensado pelo aumento de geração e pelo melhor desempenho financeiro. Já para projetos menores, a configuração fixa ainda permanece como a opção mais eficiente em termos de custo-benefício. 

\ac{TIR}, \ac{VPL} e \textit{Payback}, 

%\section{Trabalhos Futuros}
%%Para dar continuidade a ferramenta desenvolvida, são elencadas as seguintes propostas:

%\begin{enumerate}[]
    %\item \textbf{\underline{Migrar ferramenta}}: A ferramenta utilizada foi uma planilha eletrônica. A ideia migrar para uma abordagem de código aberto cooperação da comunidade. São duas vertentes possíveis atualmente: \textit{(i)} javascript ou (\textit{ii}) python; 
    
   % \item \textbf{\underline{Disponibilizar ferramenta online}}: Dependendo da migração feita, pretende-se disponibilizar a ferramenta gratuitamente em ambiente online. Existem alguns serviços de hospedagem gratuita que estão sendo estudados.
    
   % \item \textbf{\underline{Estudos de casos}}: Realizar estudos de casos reais com a ferramenta e mostrar tutoriais.
%\end{enumerate}

%a principal tarefa será a otimização, trabalhando com o máximo de variáveis possíveis, para que a resposta seja mais próxima possível da realidade. Outro aspecto importante será migrar essa ferramenta para o Python.

%O python é uma linguagem de programação de alto nível, dinâmica e modular,o que possibilita maior controle e estabilidade de códigos para projetos de grandes proporções.


%\textcolor{red}{Descrever como você planeja a execução do seu trabalho para o TCC2. O que você irá fazer e como. Essa é uma das partes mais importantes, pois sua banca pode te ajudar a melhorá-lo.}


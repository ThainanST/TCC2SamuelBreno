\chapter{Conclusão}\label{cap:capitulo_5}

Este trabalho teve como objetivo principal avaliar a viabilidade técnica e econômica de sistemas fotovoltaicos de diferentes potências, comparando configurações com estrutura fixa e com sistemas de rastreamento solar, por meio de simulações computacionais realizadas no software PV*SOL. A análise buscou identificar em quais faixas de potência o investimento em rastreadores solares se justifica do ponto de vista financeiro, considerando indicadores como geração de energia, \textit{payback}, \ac{TIR} e custos associados à implantação dos sistemas.

Para embasar essa análise, foram apresentados os fundamentos teóricos relacionados à energia solar fotovoltaica, abordando os princípios de funcionamento dos sistemas, as diferentes configurações de instalação e os conceitos associados aos sistemas de rastreamento solar. Também foram discutidos aspectos técnicos relevantes, como perdas do sistema, eficiência energética e desempenho global, além dos principais indicadores econômico-financeiros utilizados na avaliação de projetos fotovoltaicos, os quais permitem uma análise integrada entre desempenho técnico e retorno financeiro.

A metodologia adotada consistiu no desenvolvimento de modelos de simulação no software PV*SOL, considerando diferentes potências instaladas e mantendo premissas técnicas e financeiras homogêneas entre os cenários analisados. Foram simulados sistemas com estrutura fixa e com rastreamento solar, utilizando dados climáticos representativos do local de estudo, bem como parâmetros realistas de custos de implantação, operação e manutenção. Essa abordagem permitiu comparar de forma consistente os impactos do uso de rastreadores tanto na geração de energia quanto nos indicadores econômicos.

% Os resultados obtidos demonstraram que, do ponto de vista técnico, os sistemas com rastreamento solar apresentaram maior geração anual de energia e melhor desempenho energético em todas as potências analisadas. Entretanto, sob a ótica econômica, observou-se que, nas faixas de menor potência, os sistemas fixos tendem a apresentar maior atratividade financeira, devido ao menor investimento inicial. À medida que a potência instalada aumenta, ocorre uma mudança nesse comportamento, sendo identificada, a partir de aproximadamente 800 kW, uma inversão na atratividade econômica, na qual os sistemas com rastreamento passam a apresentar menor \textit{payback} e maior \ac{TIR} em comparação aos sistemas fixos.
Os resultados indicaram que, sob o ponto de vista técnico, os sistemas com rastreamento apresentaram maior geração anual de energia e melhor desempenho energético em todas as potências analisadas. Entretanto, na avaliação econômica, verificou-se que sistemas de menor porte tendem a favorecer a estrutura fixa, devido ao menor investimento inicial, enquanto em potências mais elevadas observou-se uma aproximação progressiva dos indicadores financeiros entre as configurações. A partir das potências simuladas de aproximadamente 800 kW, foi identificada uma inversão no desempenho relativo, com o sistema com rastreamento passando a apresentar \textit{payback} ligeiramente inferior e \ac{TIR} superior. Contudo, é importante destacar que esse resultado decorre das premissas teóricas adotadas nas simulações, incluindo parâmetros de custo e desempenho típicos de mercado, não representando necessariamente um comportamento universal. Assim, a inversão observada deve ser interpretada como uma tendência indicativa dentro das condições modeladas, cuja consolidação requer validação com dados empíricos e análises aplicadas em projetos reais.

% Como contribuição, este trabalho fornece uma análise comparativa clara e estruturada entre sistemas fotovoltaicos fixos e com rastreamento solar, organizada por faixas de potência, o que pode auxiliar projetistas, investidores e tomadores de decisão na escolha da configuração mais adequada para cada porte de empreendimento. Além disso, o uso de múltiplos indicadores técnicos e financeiros permite uma avaliação mais robusta da viabilidade dos sistemas, contribuindo para a literatura técnica ao integrar desempenho energético e análise econômica de forma aplicada.
Como contribuição para a academia, os resultados evidenciam tendências de viabilidade técnico-econômica em função da potência instalada, que podem orientar estudos posteriores e validações empíricas. Para o setor produtivo e o mercado fotovoltaico, a organização dos resultados por porte de empreendimento permite visualizar de forma direta a relação entre o investimento adicional em rastreadores, o ganho energético e os indicadores de retorno. Do ponto de vista da sociedade, o trabalho contribui para o entendimento das condições em que tecnologias de maior eficiência podem se tornar economicamente viáveis, com potencial impacto na eficiência do uso de recursos e na sustentabilidade da expansão da geração fotovoltaica.

% Por fim, como trabalhos futuros, recomenda-se a ampliação da análise para diferentes localidades e regimes climáticos, de forma a avaliar a sensibilidade dos resultados às variações de irradiação solar e temperatura ambiente. Sugere-se também a inclusão de sistemas com rastreamento solar de dois eixos e a consideração de diferentes estratégias de operação dos \textit{trackers}, permitindo uma comparação mais abrangente entre as alternativas tecnológicas disponíveis. Adicionalmente, a realização de análises de sensibilidade econômica, contemplando variações nos custos de investimento, operação e manutenção, tarifas de energia, taxas de desconto e degradação dos módulos, pode contribuir para uma avaliação mais robusta da viabilidade financeira. Por fim, a incorporação de critérios ambientais e de análise de risco, como indicadores de emissões evitadas e incertezas associadas aos parâmetros econômicos, pode enriquecer o processo de tomada de decisão em projetos fotovoltaicos de médio e grande porte. 
Como trabalhos futuros, recomenda-se inicialmente a validação dos resultados obtidos por meio de dados reais de implantação e operação de usinas fotovoltaicas com e sem rastreamento, de modo a verificar a robustez da inversão de viabilidade econômica observada na faixa de potência próxima de 800 kW. Nesse contexto, destaca-se a importância da obtenção de informações de mercado mais detalhadas, como cotações técnicas e econômicas de fornecedores e integradores, etapa que foi parcialmente tentada neste trabalho, mas não pôde ser explorada de forma conclusiva devido à limitada disponibilidade de dados consistentes. Adicionalmente, estudos futuros podem incorporar análises de sensibilidade e incerteza nos parâmetros econômicos e técnicos, avaliar diferentes estratégias de rastreamento e condições de \textit{layout}, e investigar o comportamento do custo e do desempenho em outras regiões climáticas. A ampliação da base empírica e a comparação com casos reais contribuirão para aumentar a confiabilidade das conclusões e aprofundar a compreensão da influência da escala na viabilidade de sistemas fotovoltaicos com rastreamento solar.

%\ac{TIR}, \ac{VPL} e \textit{Payback}, 

%\section{Trabalhos Futuros}
%%Para dar continuidade a ferramenta desenvolvida, são elencadas as seguintes propostas:

%\begin{enumerate}[]
    %\item \textbf{\underline{Migrar ferramenta}}: A ferramenta utilizada foi uma planilha eletrônica. A ideia migrar para uma abordagem de código aberto cooperação da comunidade. São duas vertentes possíveis atualmente: \textit{(i)} javascript ou (\textit{ii}) python; 
    
   % \item \textbf{\underline{Disponibilizar ferramenta online}}: Dependendo da migração feita, pretende-se disponibilizar a ferramenta gratuitamente em ambiente online. Existem alguns serviços de hospedagem gratuita que estão sendo estudados.
    
   % \item \textbf{\underline{Estudos de casos}}: Realizar estudos de casos reais com a ferramenta e mostrar tutoriais.
%\end{enumerate}

%a principal tarefa será a otimização, trabalhando com o máximo de variáveis possíveis, para que a resposta seja mais próxima possível da realidade. Outro aspecto importante será migrar essa ferramenta para o Python.

%O python é uma linguagem de programação de alto nível, dinâmica e modular,o que possibilita maior controle e estabilidade de códigos para projetos de grandes proporções.


%\textcolor{red}{Descrever como você planeja a execução do seu trabalho para o TCC2. O que você irá fazer e como. Essa é uma das partes mais importantes, pois sua banca pode te ajudar a melhorá-lo.}


\chapter{Metodologia}\label{cap:capitulo_3}

Nesta seção será a exibida a ferramenta utilizada para auxiliar na coleta de dados e simulação de casos, para obtenção de resultados 

\section{ PVSOL}

Para o desenvolvimento deste trabalho, foi utilizada a ferramenta PVSOL, um software especializado no dimensionamento, simulação e análise de sistemas fotovoltaicos conectados à rede e isolados. O PVSOL foi criado pela empresa alemã Valentin Software GmbH, reconhecida mundialmente pelo desenvolvimento de soluções computacionais voltadas para a modelagem de sistemas de energia renovável, especialmente solar fotovoltaica e térmica.

O PVSOL é amplamente empregado em estudos acadêmicos e profissionais devido à sua capacidade de realizar simulações dinâmicas que consideram fatores como radiação solar, orientação e inclinação dos módulos, sombreamentos parciais, eficiência dos inversores, perdas elétricas, além de aspectos econômicos do investimento. Essas funcionalidades permitem uma análise precisa da geração de energia ao longo do tempo, possibilitando estimativas realistas do desempenho energético e financeiro do sistema projetado.

Dentre suas principais funções, destacam-se:
\begin{itemize}
\item Modelagem detalhada do arranjo fotovoltaico: escolha do tipo de módulo, inversores, número de strings e configuração elétrica;

\item Análise de sombreamento em 3D: criação de cenários com edificações, árvores e obstáculos que possam impactar a produção de energia;

\item Simulação de geração horária, mensal e anual: possibilitando o cálculo da curva de geração de acordo com dados climáticos locais;

\item Avaliação financeira: análise de custos de investimento, operação e manutenção, fluxo de caixa, tempo de retorno e viabilidade econômica;

\item Comparação de diferentes configurações de sistemas: como sistemas fixos, trackers de um eixo e trackers de dois eixos, permitindo identificar a opção mais vantajosa tecnicamente e economicamente.
\end{itemize}
Assim, a utilização do PVSOL neste trabalho se justifica por sua confiabilidade e precisão, sendo uma ferramenta consolidada no mercado internacional e já validada em diversas pesquisas científicas e projetos de engenharia. A aplicação do software possibilita a obtenção de resultados consistentes, fundamentais para a análise comparativa proposta neste estudo. A \autoref{fig:img004}, mostra a tela inicial a ferramenta PV*SOL utilizada neste trabalho. 

\begin{figure}[h]
	\centering
	\caption{Tela Inicial.}
	\includegraphics[width=0.6\linewidth]{./textuais/figs/pv001}
    \fonteautor
	\label{fig:img004}
\end{figure}


\section{Cenários propostos}

Neste estudo, serão realizadas simulações para diferentes potências de sistemas fotovoltaicos ($10$kW, $50$kW, $100$kW, $300$kW, $500$kW e $800$kW e $1$MW ), comparando duas configurações distintas: sistema fixo, sistema com rastreamento de um eixo (\textit{single-axis tracker}). As análises incluirão a geração de energia anual (kWh/ano), os custos de implantação e manutenção, e o tempo de retorno do investimento (\textit{payback}).


A metodologia adotada consistirá nos seguintes passos:



\begin{itemize}
    \item Definição dos parâmetros de simulação: Seleção da localização geográfica, escolha dos módulos fotovoltaicos e inversores, e configuração dos tipos de montagem dos painéis.

\item Simulação no PVSOL: Entrada dos dados técnicos e financeiros no software e execução das simulações para cada cenário.

\item Coleta e organização dos resultados: Extração dos dados gerados pelo PVSOL para análise comparativa entre os sistemas fixo e com rastreamento.

\item Análise dos resultados: Avaliação da viabilidade técnica e econômica do uso de trackers em comparação com sistemas fixos, considerando geração de energia, custos e payback.

\item Elaboração de gráficos e tabelas: Representação visual dos resultados para facilitar a interpretação e a comparação entre os diferentes cenários simulados.

\end{itemize}



 Com essa abordagem, buscamos garantir que os resultados apresentados sejam tecnicamente embasados e aplicáveis a estudos de viabilidade econômica de sistemas fotovoltaicos, contribuindo para a análise do impacto do uso de \textit{trackers} na eficiência e no retorno financeiro dos investimentos em energia solar.

\section{Fluxograma}
Com o objetivo de organizar e sistematizar as etapas metodológicas adotadas neste trabalho, a \autoref{fig:img005} apresenta o fluxograma que resume o processo de desenvolvimento do estudo. O fluxograma descreve, de forma sequencial, as principais etapas realizadas, desde a definição da ferramenta de simulação e do local do projeto até a análise e comparação dos resultados obtidos.

As simulações foram conduzidas utilizando o software PV*SOL, contemplando diferentes configurações de sistemas fotovoltaicos, incluindo estruturas fixas e com rastreamento solar. Dessa forma, o fluxograma permite uma visualização clara da metodologia empregada, facilitando a compreensão do encadeamento lógico das atividades e garantindo a reprodutibilidade dos resultados apresentados.

    
  \begin{figure}[h]
	\centering
	\caption{Fluxograma.}
	\includegraphics[width=0.7\linewidth]{./textuais/figs/Fluxograma.drawio}
    \fonteautor
	\label{fig:img005}
\end{figure}

\section{Dimensionamento}
O dimensionamento de sistemas fotovoltaicos consiste em definir, de maneira técnica e criteriosa, a configuração adequada de todos os componentes do sistema, desde os módulos e inversores até os cabeamentos, arranjos físicos e eventuais transformadores necessários para a conexão à rede elétrica. Esse processo é essencial para garantir a eficiência energética, a segurança operacional e a viabilidade econômica do empreendimento. No presente trabalho, o dimensionamento foi realizado por meio do software PV*SOL, que possibilita a simulação de diferentes cenários, considerando as condições climáticas locais, a orientação dos módulos, as características elétricas dos equipamentos e as normas aplicáveis ao setor. Nas subseções a seguir, são apresentados os critérios adotados para a padronização dos módulos, a escolha dos inversores, o dimensionamento de cabos, a utilização de transformadores e a definição dos arranjos dos painéis fotovoltaicos, justificando tecnicamente cada decisão tomada no processo. A \autoref{fig:img006} mostra a tela inicial do PV*SOL, que é de onde partimos para o dimensionamento dos sistemas fotovoltaico. 

  \begin{figure}[h]
	\centering
	\caption{Dados do projeto.}
	\includegraphics[width=0.7\linewidth]{./textuais/figs/pv002}
    \fonteautor
	\label{fig:img006}
\end{figure}




\subsection{Posicionamento Geográfico dos Módulos}


As simulações foram realizadas considerando a cidade de Belo Horizonte MG, situada na latitude $19,9$° Sul. Em localidades do hemisfério sul, a orientação mais favorável dos módulos fotovoltaicos é voltada para o norte geográfico, de modo a maximizar a captação da radiação solar ao longo do ano.

De acordo com as recomendações do CRESESB (Centro de Referência para Energia Solar e Eólica Sérgio Brito), a inclinação ótima dos painéis é de aproximadamente $20$°. Essa configuração possibilita um melhor aproveitamento da irradiação solar global, garantindo o equilíbrio entre a geração de energia nos diferentes períodos sazonais. Dessa forma, a adoção da inclinação de $20$° e orientação para o norte atende ao projeto. A \autoref{fig:img007}, mostra onde é definida a localização geográfica da usina fotovoltaica na ferramenta PV*SOL. 

\begin{figure}[h]
	\centering
	\caption{Localização da usina.}
	\includegraphics[width=0.6\linewidth]{./textuais/figs/pv003}
    \fonteautor
	\label{fig:img007}
\end{figure}


\subsection{Módulos Fotovoltaicos}
Para garantir uniformidade na análise dos diferentes sistemas simulados, adotou-se a padronização dos módulos fotovoltaicos no software PV*SOL. A utilização de um mesmo modelo de painel, com potência nominal, eficiência e características técnicas bem definidas, permite uma comparação justa entre diferentes cenários de potência instalada e configurações de rastreamento solar.
A escolha do módulo considerou critérios como eficiência média do mercado, confiabilidade do fabricante e disponibilidade comercial no Brasil. Segundo \cite{vilela2012analise}, a seleção de módulos deve priorizar não apenas a potência unitária, mas também a eficiência de conversão e o coeficiente de temperatura, que influenciam diretamente no desempenho energético do sistema.

Para a realização das simulações no software PV*SOL, optou-se pela utilização do módulo \textit{Canadian Solar CS3Y-500MS}, com potência nominal de $500$Wp e tecnologia de células monocristalinas. Esse modelo apresenta eficiência de conversão próxima a $21$\%, garantindo maior aproveitamento da radiação solar incidente e consequente aumento da geração elétrica por área instalada.

Além da alta eficiência, o módulo incorpora a tecnologia \textit{half-cell}, que reduz perdas resistivas e melhora a performance em condições de sombreamento parcial, característica relevante para aplicações em usinas de médio e grande porte. A escolha desse módulo justifica-se, ainda, por sua ampla disponibilidade comercial no mercado brasileiro e pela sua representatividade em projetos fotovoltaicos de maior escala.

Outro aspecto relevante é que a adoção de módulos de maior potência unitária contribui para a redução de custos indiretos, como cabeamento, estruturas de suporte e mão de obra, uma vez que se necessita de menor número de painéis para atingir a mesma potência instalada. Dessa forma, a utilização do CS3Y-500MS garante não apenas a confiabilidade e o desempenho energético do sistema, mas também a viabilidade econômica e a consistência metodológica das análises realizadas neste trabalho.A \autoref{fig:img008}, mostra onde é feita a seleção dos módulos na ferramenta PV*SOL. 

\begin{figure}[h]
	\centering
	\caption{Seleção do painel fotovoltaico, }
	\includegraphics[width=0.6\linewidth]{./textuais/figs/pv004}
    \fonteautor
	\label{fig:img008}
\end{figure}

%%Para a simulação do sistema no software PV*SOL, foi necessário determinar parâmetros fundamentais que influenciam diretamente o desempenho e a confiabilidade da instalação. Entre esses aspectos, destacam-se o dimensionamento das strings fotovoltaicas no lado em corrente contínua (CC) e a escolha adequada dos condutores responsáveis pela condução da corrente alternada (CA). Esse processo foi conduzido de forma criteriosa, buscando reduzir perdas elétricas e assegurar que o sistema opere de maneira eficiente e segura.

%No caso do arranjo das strings, definiu-se a quantidade ideal de módulos conectados em série e em paralelo, observando limites de tensão suportados pelo inversor, a faixa operacional de trabalho e as variações climáticas da região de instalação. Esse cuidado garante que os módulos operem dentro de condições adequadas, prevenindo riscos de sobrecarga e mantendo a eficiência global do sistema.

%Quanto ao circuito em corrente alternada, a seleção dos cabos foi realizada considerando fatores como a queda de tensão admissível e a segurança elétrica, em conformidade com os critérios estabelecidos pela norma NBR 5410. A adoção de condutores adequados é essencial para evitar perdas excessivas de energia e para garantir a integridade da instalação ao longo de sua vida útil.

\subsection{Inversores}

No que se refere aos inversores, sua seleção foi realizada de acordo com a potência de cada sistema simulado, observando-se a disponibilidade de modelos comerciais no Brasil e compatíveis com o banco de dados do PV*SOL. A \autoref{tab:inversores}, mostra os inversores adotados para as respectivas potências estudadas,

\begin{table}[h!]
\centering
\begin{tabular}{|c|c|c|c|}
\hline
\textbf{Porte da Usina } & \textbf{Fabricante} & \textbf{Quantidade}   & \textbf{Inversor}  \\ \hline
    $10$kW & SOLSYSTEMS GPE  & $1$ & SOL GPE $10$kW \\ \hline
$50$kW & SOLSYSTEMS GPE & $1$ & SOL GPE $50$kW \ \\  \hline
$100$kW & SOLSYSTEMS GPE & $2$ & SOL GPE $50$kW  \ \\ \hline
$300$kW & CANADIAN SOLAR INC. & $3$ & CSI-100K-T4001B-E \ \\ \hline
$500$kW & CANADIAN SOLAR INC. & $4$ & CSI-125KTL-GI-E \ \\ \hline
$800$kW & WEG & $4$ & SIW500H ST200 H3 \ \\ \hline
$1$MW & WEG & $5$ & SIW500H ST200 H3 \ \\ \hline
\end{tabular}
\caption{Inversores utilizados nas simulações}
\label{tab:inversores}
\end{table}


A adoção desses equipamentos não ocorreu de forma aleatória, mas baseou-se em critérios técnicos e de mercado, considerando a confiabilidade dos fabricantes, a disponibilidade no território nacional e a compatibilidade com projetos de diferentes portes. Dessa forma, a padronização estabelecida contribui para a consistência das análises, permitindo comparações precisas entre os diferentes cenários de potência e configuração avaliados neste trabalho. A \autoref{fig:img009}, mostra onde é feita a seleção dos inversores na ferramenta PV*SOL. 

\begin{figure}[h]
	\centering
	\caption{Escolha dos inversores }
	\includegraphics[width=0.6\linewidth]{./textuais/figs/pv005}
    \fonteautor
	\label{fig:img009}
\end{figure}



\subsection{Perdas do Sistema}


O software PV*SOL disponibiliza duas abordagens distintas para a definição das perdas em sistemas fotovoltaicos. A primeira delas é o modelo detalhado, no qual cada tipo de perda pode ser configurado individualmente, contemplando aspectos como perdas ôhmicas em cabos, eficiência dos inversores, perdas por \textit{mismatch}, sombreamento, sujeira, temperatura, entre outros fatores técnicos. Essa abordagem permite maior precisão no dimensionamento, sobretudo em projetos executivos, onde as condições reais da instalação já estão definidas com clareza.

Por outro lado, o modelo simplificado possibilita a inserção de um único percentual global de perdas, que agrega todos os efeitos mencionados de forma consolidada. Esse método é frequentemente utilizado em estudos de pré-viabilidade ou em análises comparativas, uma vez que simplifica a modelagem sem comprometer a representatividade dos resultados.

Neste trabalho, optou-se pela utilização do modelo simplificado, aplicando um valor global de $14$\% de perdas em todas as simulações realizadas. O valor de $14$\% fundamenta-se em referências amplamente reconhecidas. O \textit{PVGIS (Photovoltaic Geographical Information System)}, ferramenta de referência internacional para estimativas solares, adota esse percentual como padrão para estudos simplificados de desempenho de sistemas fotovoltaicos. Além disso, pesquisas acadêmicas nacionais, como o trabalho de \cite{tonolo2019analise} na Universidade Tecnológica Federal do Paraná (UTFPR), também apresentam valores típicos de perdas globais em faixas de $12$\% a $16$\%, reforçando a adequação do percentual adotado.

%Neste trabalho, optou-se pela utilização do modelo de perdas simplificado no PV*SOL, aplicando um percentual global fixo de 14\% em todas as simulações. Essa decisão está alinhada ao foco principal da pesquisa, que se concentra na avaliação da viabilidade técnica e econômica dos sistemas fotovoltaicos em diferentes escalas de potência e configurações de rastreamento. Assim, evita-se que a análise seja dispersa em um detalhamento minucioso das perdas individuais, permitindo que as comparações entre os cenários tenham como base apenas as variáveis de interesse.

%Embora seja sabido que as perdas técnicas podem variar de acordo com a escala da usina — influenciadas por fatores como distâncias de cabeamento, uso de transformadores e número de inversores —, a adoção de um valor único e padronizado assegura consistência metodológica. Dessa forma, qualquer diferença nos resultados das simulações pode ser atribuída exclusivamente às variáveis estudadas, e não a flutuações decorrentes de diferentes critérios de cálculo de perdas.

%A escolha do valor de 14\% fundamenta-se em referências amplamente reconhecidas. O PVGIS (Photovoltaic Geographical Information System), ferramenta de referência internacional para estimativas solares, adota esse percentual como padrão para estudos simplificados de desempenho de sistemas fotovoltaicos (PVGIS, 2023). No contexto brasileiro, a ANEEL reporta que perdas totais médias no setor elétrico (técnicas e não técnicas) oscilam entre 14\% e 15\%, valor próximo ao utilizado como parâmetro neste estudo (ANEEL, 2021). Além disso, pesquisas acadêmicas nacionais, como o trabalho de Tonolo (2019) na UTFPR, também apresentam valores típicos de perdas globais em faixas de 12\% a 16\%, reforçando a adequação do percentual adotado.

Portanto, a utilização do modelo simplificado, com perdas fixadas em 14\%, garante a confiabilidade dos resultados, preserva a comparabilidade entre diferentes cenários e mantém o foco no objetivo central deste trabalho: analisar a viabilidade econômica  da utilização de \textit{trackers} em sistemas fotovoltaicos com distintas escalas de potência.A \autoref{fig:img010}, mostra onde é definida as perdas do sistema fotovoltaico na ferramenta PV*SOL. 

\begin{figure}[h]
	\centering
	\caption{Perdas do sistewa.}
	\includegraphics[width=0.6\linewidth]{./textuais/figs/pv006}
    \fonteautor
	\label{fig:img010}
\end{figure}


\section{Aspectos financeiros de uma Usina }

\subsection{CAPEX}
O \ac{CAPEX}, ou despesa de capital, corresponde ao investimento inicial necessário para a implantação de um sistema fotovoltaico, englobando todos os custos relacionados à aquisição, transporte, instalação e comissionamento do empreendimento. Em projetos de geração solar, o \ac{CAPEX} é considerado o principal fator econômico a ser avaliado, uma vez que representa a maior parcela dos desembolsos financeiros do projeto, sendo determinante para a viabilidade econômico-financeira do mesmo \cite{epe2022}.

Para a determinação do \ac{CAPEX} neste trabalho, adotou-se como referência os valores apresentados por \cite{Sisquini2024}, apresentados na \autoref{tab:customedio}, no estudo os autores apresentam uma análise detalhada dos custos médios de implantação de usinas fotovoltaicas no Brasil, contemplando diferentes portes de sistemas, desde aplicações residenciais até grandes usinas solares.

\begin{table}[h!]
\centering
\begin{tabular}{|c|c|}
\hline
\textbf{Porte da Usina (kWp)} & \textbf{Custo Médio (RS/Wp)}  \\ \hline
Residencial (até $10$ kWp) & $5,50$ - $7,50$  \\ \hline
Comercial ($10$ kWp a $500$ kWp) & $4,50$ - $6,50$ \\  \hline
Industrial (acima de $500$ kWp) & $3,80$ - $5,80$ \\ \hline
Grandes usinas (acima de $1$ MWp) & $3,50$ -$ 5,50$  \\ \hline
\end{tabular}
\caption{Custo médio por watt-pico (Wp) instalado}
\label{tab:customedio}
\end{table}

Já os sistemas com \textit{trackers} exigem componentes mecânicos adicionais e maior complexidade de instalação, o que impacta diretamente no seu custo inicial. Os \textit{trackers} podem elevar o \ac{CAPEX} não apenas pela estrutura móvel em si, mas também por requererem manutenção mais frequente, ocuparem mais área e demandarem cabeamentos e proteções especiais, fatores que não são necessários em sistemas fixos.

Com o objetivo de embasar a diferença de custos de investimento entre sistemas fotovoltaicos de estrutura fixa e sistemas com rastreamento solar, foi realizada uma revisão de trabalhos acadêmicos nacionais e internacionais que abordam comparações econômicas entre essas configurações para diferentes faixas de potência instalada. A \autoref{tab:capextracker} apresenta a relação de custo entre sistemas com \textit{tracker} e sistemas fixos, obtida a partir de artigos científicos, trabalhos de conclusão de curso, dissertações e publicações em periódicos técnicos. Observa-se que a razão de custo \textit{tracker}/fixo varia conforme a potência do sistema, o escopo do estudo e as premissas adotadas por cada autor, apresentando valores típicos entre aproximadamente $1,15$ e $1,80$. Dessa forma, os dados consolidados permitem estimar um acréscimo percentual no \ac{CAPEX} associado à utilização de rastreadores solares, servindo como base técnica e bibliográfica para a definição dos custos de investimento utilizados nas simulações e análises financeiras desenvolvidas neste trabalho.

\begin{table}[h!]
\centering
\begin{tabular}{|
>{\centering\arraybackslash}p{2.2cm}|
>{\centering\arraybackslash}p{2.2cm}|
>{\centering\arraybackslash}p{3.0cm}|
>{\centering\arraybackslash}p{1.4cm}|
>{\centering\arraybackslash}p{3.5cm}|
}
\hline
\textbf{Potência do sistema (kW)} &
\textbf{Potência do estudo (kW)} &
\textbf{Custo relativo tracker/fixo} &
\textbf{Média} &
\textbf{Fonte} \\ \hline

$10$ & $8,76$ & $1,19$ & $1,195$ & \cite{nucleodoconhecimento_tracker} \\ \hline
$10$ & $3$ & $1,20$ &$ 1,195$ & \cite{revistaft_seguidor_solar} \\ \hline
$50$ & $75$ & $1,50$ & $1,345$ & \cite{nobrega2019comparaccao} \\ \hline
$50$ & $75$ & $1,19$ & $1,345$ & \cite{pinto2025analise} \\ \hline
$100$ & $100$ & $1,50$ & $1,345$ & \cite{nobrega2019comparaccao} \\ \hline
$100$ & $75$ & $1,19$ & $1,345$ & \cite{pinto2025analise} \\ \hline
$300$ & $300$ & $1,80$ & $1,485$ & \cite{araujo2023avaliaccao} \\ \hline
$300$ & $300$ & $1,17$ & $1,485$ & \cite{casotto2021avaliaccao} \\ \hline
$500$ & $500$ & $1,36$ & $1,36$ & \cite{pinto2025analise} \\ \hline
$800$ & $781$ & $1,18$ & $1,19$ & \cite{elahi2023_ijred} \\ \hline
$800$ & $800$ & $1,20$ & $1,19$ & \cite{regia2022_tcc_ufsc} \\ \hline
$1000$ & $1000$ & $1,18$ & $1,165$ & \cite{pinto2025analise} \\ \hline
$1000$ & $2000$ & $1,15$ &  $1,165$ & \cite{de2022analise} \\ \hline

\end{tabular}
\caption{Relação de custo Fixo/\textit{Tracker}}
\label{tab:capextracker}
\end{table}

\subsection{OPEX}
O \ac{OPEX} (Operational Expenditure) refere-se aos custos operacionais recorrentes, incluindo manutenção, monitoramento, limpeza, seguros e eventuais falhas técnicas. Em sistemas fotovoltaicos com rastreadores solares (\textit{trackers}), os custos de manutenção apresentam um incremento relevante em relação às estruturas fixas, devido à presença de componentes móveis como motores, sensores e sistemas de controle que requerem inspeções periódicas, lubrificação e eventuais reparos. Assim como no Caderno de Preços da Geração (2022/2023)\cite{epe2022} neste trabalho sera utilizado \ac{OPEX} de 1,5\% para sistemas fixos e 2\% para sistemas com (\textit{trackers}). A \autoref{fig:img011}, mostra onde é definido \ac{CAPEX} E \ac{OPEX} do sistema fotovoltaico na ferramenta PV*SOL. 

\begin{figure}[h]
	\centering
	\caption{Capex e Opex do sistema.}
	\includegraphics[width=0.6\linewidth]{./textuais/figs/pv007}
    \fonteautor
	\label{fig:img011}
\end{figure}


\section{Estruturação dos Cenários}

Nesta etapa, são apresentados os diferentes cenários de simulação desenvolvidos no software PV*SOL, contemplando variadas potências de sistemas fotovoltaicos e distintas configurações de instalação. Foram considerados arranjos com estrutura fixa, bem como sistemas equipados com rastreadores solares de um eixo, de modo a possibilitar uma análise comparativa tanto em termos de desempenho energético quanto de viabilidade econômica. Essa estruturação permite avaliar de forma consistente o impacto da escala de potência e da tecnologia empregada sobre os indicadores técnicos e financeiros do projeto.

%\subsection{Usina de 10kW}
%Para a realização das simulações no software PV*SOL, é imprescindível a parametrização adequada de todos os aspectos técnicos do sistema, assegurando que os resultados obtidos representem com fidelidade as condições reais de operação. Esse processo compreende a definição dos módulos fotovoltaicos, o correto dimensionamento das strings, a seleção dos inversores compatíveis com a potência instalada e a consideração das condições estruturais adotadas em cada cenário. 

%Além disso, incluem-se na modelagem as perdas globais do sistema, e a definição dos custos de implantação (CAPEX) e operação (OPEX), permitindo integrar a análise técnica à avaliação econômico-financeira. Dessa forma, garante-se consistência metodológica tanto nos cálculos de desempenho energético quanto na mensuração da viabilidade dos projetos simulados.
\subsection{Parâmetros Técnicos}
Para a realização das simulações no software PV*SOL, foi necessário definir os parâmetros técnicos que caracterizam o sistema fotovoltaico em cada cenário analisado. Esses aspectos incluem a localização geográfica, o modelo de módulo utilizado, a inclinação e orientação dos painéis, a configuração dos rastreadores solares quando aplicáveis, os inversores selecionados e a representação gráfica do arranjo do sistema, garantindo consistência metodológica em todas as análises realizadas.

%Os aspectos técnicos considerados para a execução das simulações no software PV*SOL foram definidos de forma a representar fielmente as condições de instalação e operação do sistema fotovoltaico. A seguir, estão descritos os principais parâmetros adotados:
\begin{itemize}
    

    \item Localização da Usina: As simulações foram realizadas para a cidade de Belo Horizonte – MG, situada na latitude $19,9$°S, longitude $43,9$°O. A escolha da localidade foi feita considerando sua alta incidência solar e relevância para estudos de viabilidade fotovoltaica.

    \item Módulo Fotovoltaico: Foi utilizado o módulo \textit{Canadian Solar Inc.} – CS3Y-500MS, de $500$ Wp. Esse modelo foi escolhido por sua ampla disponibilidade no mercado brasileiro e eficiência em torno de $21$\%.

    \item Inclinação e Orientação: Para os sistemas fixos, os módulos foram orientados para o norte geográfico ($0$°) e com inclinação de $20$°, conforme recomendações do CRESESB para Belo Horizonte.

    \item Sistemas com Rastreadores (\textit{trackers}): Nos cenários em que foram utilizados \textit{trackers}, a angulação de movimento considerada foi de -$60$° a +$60$°, contemplando o rastreamento solar ao longo do dia.

    %\item Inversores: Para o sistema de 10 kW, foi utilizado o inversor SOL GPE 10kW da empresa fabricante SOLSYSTEMS GPE.

    \item Adotou-se o modelo simplificado de perdas com percentual global fixo de $14$\% aplicado a todos os cenários, assegurando padronização metodológica.

   % \item Representação Gráfica: Para complementar a descrição técnica, cada cenário contou com a elaboração de um diagrama do sistema no PV*SOL, possibilitando a visualização da configuração adotada em termos de módulos, inversores e conexões principais.
    

\end{itemize}

\subsection{Parâmetos Financeiros}

    Para garantir a consistência e a comparabilidade entre os cenários analisados, adotou-se como premissa que o custo operacional e de manutenção \ac{OPEX}, o custo da energia elétrica e a vida útil dos sistemas serão mantidos iguais para todas as configurações, variando apenas em função das respectivas potências instaladas. Dessa forma, as diferenças observadas nos resultados financeiros refletem exclusivamente o impacto da adoção de estruturas fixas ou com rastreador solar.
    
    A \autoref{tab:parametrosfinanceiros} apresenta um resumo dos parâmetros financeiros empregados como dados de entrada nas simulações realizadas durante os testes.

\begin{table}[h!]
\centering
\begin{tabular}{|c|c|c|c|c|}
\hline
\textbf{Tipo de Sistema} &  \textbf{\ac{OPEX}}& \textbf{Custo de Energia}& \textbf{Vida Útil}  \\ \hline
Fixo &  $1,5$\% & $0,54$ & $25$ anos \\ \hline
Tracker  & $2,0$\%  & $0,54$ & $25$ anos \\  \hline

\end{tabular}
\caption{Parâmetros Financeiros}
\label{tab:parametrosfinanceiros}
\end{table}

Na sequência,  a \autoref{tab:capex} apresentam os custos de investimento inicial \ac{CAPEX} referentes a cada sistema fotovoltaico analisado, contemplando as configurações de estrutura fixa e com rastreador solar. Esses custos englobam os principais componentes do sistema, como módulos fotovoltaicos, inversores, estruturas de suporte, equipamentos elétricos, instalação e demais despesas associadas à implantação. A apresentação do \ac{CAPEX} por configuração e por potência instalada permite uma comparação direta entre os sistemas, evidenciando o impacto econômico da adoção de estruturas fixas ou com rastreamento solar no investimento inicial do empreendimento.

\begin{table}[h!]
\centering
\begin{tabular}{|c|c|c|c|c|}
\hline
\textbf{Potência do Sistema (kW)} &  
\textbf{Tipo de Sistema} & \textbf{CAPEX (R\$)}  \\ \hline
$10$ & Fixo &  $55.000,00$ \\ \hline
$10$ & Tracker  & $66.000,00$   \\  \hline
$50$ & Fixo &   $255.000,00$ \\ \hline
$50$ & Tracker  &  $318.750,00$   \\  \hline
$100$ & Fixo &    4$80.000,00$ \\ \hline
$100$ & Tracker  &   $600.000,00$   \\  \hline
$300$ & Fixo &   $1.350.000,00$ \\ \hline
$300$ & Tracker  & $1.687.000,00$   \\  \hline
$500$ & Fixo & $ 2.100.000,00$ \\ \hline
$500$ & Tracker  & $2.520.000,00$   \\  \hline
$800$ & Fixo &   $3.120.000,00$ \\ \hline
$800$ & Tracker  &  $3.740.000,00$   \\  \hline
$1000$ & Fixo &  $3.500.000,00$ \\ \hline
$1000$ & Tracker  & $4.200.000,00$   \\  \hline
\end{tabular}
\caption{\ac{CAPEX} das Usinas }
\label{tab:capex}
\end{table}


%\subsection{Usina de 50kW}
%\subsubsection{Parâmetros Técnicos}
%Para a realização das simulações no software PV*SOL, foi necessário definir os parâmetros técnicos que caracterizam o sistema fotovoltaico em cada cenário analisado. Esses aspectos incluem a localização geográfica, o modelo de módulo utilizado, a inclinação e orientação dos painéis, a configuração dos rastreadores solares quando aplicáveis, os inversores selecionados e a representação gráfica do arranjo do sistema, garantindo consistência metodológica em todas as análises realizadas.
%\begin{itemize}
    

   % \item Localização da Usina: As simulações foram realizadas para a cidade de Belo Horizonte – MG, situada na latitude 19,9°S, longitude 43,9°O. A escolha da localidade foi feita considerando sua alta incidência solar e relevância para estudos de viabilidade fotovoltaica.

   % \item Módulo Fotovoltaico: Foi utilizado o módulo Canadian Solar Inc. – CS3Y-500MS, de 500 Wp. Esse modelo foi escolhido por sua ampla disponibilidade no mercado brasileiro e eficiência em torno de 21\%.

  %  \item Inclinação e Orientação: Para os sistemas fixos, os módulos foram orientados para o norte geográfico (0°) e com inclinação de 20°, conforme recomendações do CRESESB para Belo Horizonte.

   % \item Sistemas com Rastreadores (Trackers): Nos cenários em que foram utilizados trackers, a angulação de movimento considerada foi de -60° a +60°, contemplando o rastreamento solar ao longo do dia.

  % \item Inversores: Para o sistema de 50 kW, foi utilizado o inversor SOL GPE 50kW da empresa fabricante SOLSYSTEMS GPE.

  %  \item adotou-se o modelo simplificado de perdas com percentual global fixo de 14\% aplicado a todos os cenários, assegurando padronização metodológica.

   % \item Representação Gráfica: Para complementar a descrição técnica, cada cenário contou com a elaboração de um diagrama do sistema no PV*SOL, possibilitando a visualização da configuração adotada em termos de módulos, inversores e conexões principais.
    

%\end{itemize}

%\subsubsection{Parâmetos Financeiros}

    %A Tabela  apresenta um resumo de todos os parâmetros financeiros empregados como dados de entrada nas simulações realizadas durante os testes.

%\begin{table}[h!]
%\centering
%\begin{tabular}{|c|c|c|c|c|}
%\hline
%\textbf{Tipo de Sistema} & \textbf{CAPEX} & \textbf{OPEX}& \textbf{Custo de Energia}& \textbf{Vida Útil}  \\ \hline
%Fixo & R\$ 255.000,00 & 1,5\% & 0,54 & 25 anos \\ \hline
%Tracker & R\$ 318.750,00 & 2,0\%  & 0,54 & 25 anos \\  \hline

%\end{tabular}
%\caption{Parametros Financeitos das Usinas de 50kW}
%\label{tab:exemplo}
%\end{table}



%\subsection{Usina de 100kW}
%\subsubsection{Parâmetros Técnicos}
%Para a realização das simulações no software PV*SOL, foi necessário definir os parâmetros técnicos que caracterizam o sistema fotovoltaico em cada cenário analisado. Esses aspectos incluem a localização geográfica, o modelo de módulo utilizado, a inclinação e orientação dos painéis, a configuração dos rastreadores solares quando aplicáveis, os inversores selecionados e a representação gráfica do arranjo do sistema, garantindo consistência metodológica em todas as análises realizadas.
%\begin{itemize}
    

   % \item Localização da Usina: As simulações foram realizadas para a cidade de Belo Horizonte – MG, situada na latitude 19,9°S, longitude 43,9°O. A escolha da localidade foi feita considerando sua alta incidência solar e relevância para estudos de viabilidade fotovoltaica.

  %  \item Módulo Fotovoltaico: Foi utilizado o módulo Canadian Solar Inc. – CS3Y-500MS, de 500 Wp. Esse modelo foi escolhido por sua ampla disponibilidade no mercado brasileiro e eficiência em torno de 21\%.

  %  \item Inclinação e Orientação: Para os sistemas fixos, os módulos foram orientados para o norte geográfico (0°) e com inclinação de 20°, conforme recomendações do CRESESB para Belo Horizonte.

   %\item Sistemas com Rastreadores (Trackers): Nos cenários em que foram utilizados trackers, a angulação de movimento considerada foi de -60° a +60°, contemplando o rastreamento solar ao longo do dia.

 %   \item Inversores: Para o sistema de 100 kW, foram utilizados dois inversores SOL GPE 50kW da empresa fabricante SOLSYSTEMS GPE..

 %   \item adotou-se o modelo simplificado de perdas com percentual global fixo de 14\% aplicado a todos os cenários, assegurando padronização metodológica.

 %  % \item Representação Gráfica: Para complementar a descrição técnica, cada cenário contou com a elaboração de um diagrama do sistema no PV*SOL, possibilitando a visualização da configuração adotada em termos de módulos, inversores e conexões principais.
    

%\end{itemize}

%\subsubsection{Parâmetos Financeiros}

   % A Tabela  apresenta um resumo de todos os parâmetros financeiros empregados como dados de entrada nas simulações realizadas durante os testes.

%\begin{table}[h!]
%\centering
%\begin{tabular}{|c|c|c|c|c|}
%\hline
%\textbf{Tipo de Sistema} & %\textbf{CAPEX} & \textbf{OPEX}& \textbf{Custo de Energia}& \textbf{Vida Útil}  \\ \hline
%Fixo & R\$ 480.000,00 & 1,5\% & 0,54 & 25 anos \\ \hline
%Tracker & R\$ 600.000,00 & 2,0\%  & 0,54 & 25 anos \\  \hline

%\end{tabular}
%\caption{Parametros Financeitos das Usinas de 100kW}
%\label{tab:exemplo}
%\end{table}



%\subsection{Usina de 300kW}
%\subsubsection{Parâmetros Técnicos}
%Para a realização das simulações no software PV*SOL, foi necessário definir os parâmetros técnicos que caracterizam o sistema fotovoltaico em cada cenário analisado. Esses aspectos incluem a localização geográfica, o modelo de módulo utilizado, a inclinação e orientação dos painéis, a configuração dos rastreadores solares quando aplicáveis, os inversores selecionados e a representação gráfica do arranjo do sistema, garantindo consistência metodológica em todas as análises realizadas.
%\begin{itemize}
    

%    \item Localização da Usina: As simulações foram realizadas para a cidade de Belo Horizonte – MG, situada na latitude 19,9°S, longitude 43,9°O. A escolha da localidade foi feita considerando sua alta incidência solar e relevância para estudos de viabilidade fotovoltaica.

 %   \item Módulo Fotovoltaico: Foi utilizado o módulo Canadian Solar Inc. – CS3Y-500MS, de 500 Wp. Esse modelo foi escolhido por sua ampla disponibilidade no mercado brasileiro e eficiência em torno de 21\%.

%    \item Inclinação e Orientação: Para os sistemas fixos, os módulos foram orientados para o norte geográfico (0°) e com inclinação de 20°, conforme recomendações do CRESESB para Belo Horizonte.

  %  \item Sistemas com Rastreadores (Trackers): Nos cenários em que foram utilizados trackers, a angulação de movimento considerada foi de -60° a +60°, contemplando o rastreamento solar ao longo do dia.

 %   \item Inversores: Para o sistema de 300 kW, foram utilizados três inversores CSI-100K-T4001B-E da empresa fabricante CANADIAN SOLAR INC..

  %  \item adotou-se o modelo simplificado de perdas com percentual global fixo de 14\% aplicado a todos os cenários, assegurando padronização metodológica.

   % \item Representação Gráfica: Para complementar a descrição técnica, cada cenário contou com a elaboração de um diagrama do sistema no PV*SOL, possibilitando a visualização da configuração adotada em termos de módulos, inversores e conexões principais.
    

%\end{itemize}

%\subsubsection{Parâmetos Financeiros}

   % A Tabela  apresenta um resumo de todos os parâmetros financeiros empregados como dados de entrada nas simulações realizadas durante os testes.

%\begin{table}[h!]
%\centering
%\begin{tabular}{|c|c|c|c|c|}
%\hline
%\textbf{Tipo de Sistema} & \textbf{CAPEX} & \textbf{OPEX}& \textbf{Custo de Energia}& \textbf{Vida Útil}  \\ \hline
%Fixo & R\$ 1.350.000,00 & 1,5\% & 0,54 & 25 anos \\ \hline
%Tracker & R\$ 1.687.000,00 & 2,0\%  & 0,54 & 25 anos \\  \hline

%\end{tabular}
%\caption{Parametros Financeitos das Usinas de 300kW}
%\label{tab:exemplo}
%\end{table}



%\subsection{Usina de 500kW}
%\subsubsection{Parâmetros Técnicos}
%Para a realização das simulações no software PV*SOL, foi necessário definir os parâmetros técnicos que caracterizam o sistema fotovoltaico em cada cenário analisado. Esses aspectos incluem a localização geográfica, o modelo de módulo utilizado, a inclinação e orientação dos painéis, a configuração dos rastreadores solares quando aplicáveis, os inversores selecionados e a representação gráfica do arranjo do sistema, garantindo consistência metodológica em todas as análises realizadas.
%\begin{itemize}
    

    %\item Localização da Usina: As simulações foram realizadas para a cidade de Belo Horizonte – MG, situada na latitude 19,9°S, longitude 43,9°O. A escolha da localidade foi feita considerando sua alta incidência solar e relevância para estudos de viabilidade fotovoltaica.

  %  \item Módulo Fotovoltaico: Foi utilizado o módulo Canadian Solar Inc. – CS3Y-500MS, de 500 Wp. Esse modelo foi escolhido por sua ampla disponibilidade no mercado brasileiro e eficiência em torno de 21\%.

   % \item Inclinação e Orientação: Para os sistemas fixos, os módulos foram orientados para o norte geográfico (0°) e com inclinação de 20°, conforme recomendações do CRESESB para Belo Horizonte.

  %  \item Sistemas com Rastreadores (Trackers): Nos cenários em que foram utilizados trackers, a angulação de movimento considerada foi de -60° a +60°, contemplando o rastreamento solar ao longo do dia.

  %  \item Inversores: Para o sistema de 500 kW, foram utilizados quatro inversores CSI-125KTL-GI-E da empresa fabricante CANADIAN SOLAR INC..

  %  \item adotou-se o modelo simplificado de perdas com percentual global fixo de 14\% aplicado a todos os cenários, assegurando padronização metodológica.

   % \item Representação Gráfica: Para complementar a descrição técnica, cada cenário contou com a elaboração de um diagrama do sistema no PV*SOL, possibilitando a visualização da configuração adotada em termos de módulos, inversores e conexões principais.
    

%\end{itemize}

%\subsubsection{Parâmetos Financeiros}

   % A Tabela  apresenta um resumo de todos os parâmetros financeiros empregados como dados de entrada nas simulações realizadas durante os testes.

%\begin{table}[h!]
%\centering
%\begin{tabular}{|c|c|c|c|c|}
%\hline
%\textbf{Tipo de Sistema} & \textbf{CAPEX} & \textbf{OPEX}& \textbf{Custo de Energia}& \textbf{Vida Útil}  \\ \hline
%Fixo & R\$ 2.100.000,00 & 1,5\% & 0,54 & 25 anos \\ \hline
%Tracker & R\$ 2.520.000,00 & 2,0\%  & 0,54 & 25 anos \\  \hline

%\end{tabular}
%\caption{Parametros Financeitos das Usinas de 500kW}
%\label{tab:exemplo}
%\end{table}


%\subsection{Usina de 800kW}
%\subsubsection{Parâmetros Técnicos}
%Para a realização das simulações no software PV*SOL, foi necessário definir os parâmetros técnicos que caracterizam o sistema fotovoltaico em cada cenário analisado. Esses aspectos incluem a localização geográfica, o modelo de módulo utilizado, a inclinação e orientação dos painéis, a configuração dos rastreadores solares quando aplicáveis, os inversores selecionados e a representação gráfica do arranjo do sistema, garantindo consistência metodológica em todas as análises realizadas.
%\begin{itemize}
    

   % \item Localização da Usina: As simulações foram realizadas para a cidade de Belo Horizonte – MG, situada na latitude 19,9°S, longitude 43,9°O. A escolha da localidade foi feita considerando sua alta incidência solar e relevância para estudos de viabilidade fotovoltaica.

  %  \item Módulo Fotovoltaico: Foi utilizado o módulo Canadian Solar Inc. – CS3Y-500MS, de 500 Wp. Esse modelo foi escolhido por sua ampla disponibilidade no mercado brasileiro e eficiência em torno de 21\%.

   %\item Inclinação e Orientação: Para os sistemas fixos, os módulos foram orientados para o norte geográfico (0°) e com inclinação de 20°, conforme recomendações do CRESESB para Belo Horizonte.

   % \item Sistemas com Rastreadores (Trackers): Nos cenários em que foram utilizados trackers, a angulação de movimento considerada foi de -60° a +60°, contemplando o rastreamento solar ao longo do dia.

   % \item Inversores: Para o sistema de 800 kW, foram utilizados quatro inversores SIW500H ST200 H3 da empresa fabricante WEG..

   % \item adotou-se o modelo simplificado de perdas com percentual global fixo de 14\% aplicado a todos os cenários, assegurando padronização metodológica.

   % \item Representação Gráfica: Para complementar a descrição técnica, cada cenário contou com a elaboração de um diagrama do sistema no PV*SOL, possibilitando a visualização da configuração adotada em termos de módulos, inversores e conexões principais.
    

%\end{itemize}

%\subsubsection{Parâmetos Financeiros}

  %  A Tabela  apresenta um resumo de todos os parâmetros financeiros empregados como dados de entrada nas simulações realizadas durante os testes.

%\begin{table}[h!]
%\centering
%\begin{tabular}{|c|c|c|c|c|}
%\hline
%\textbf{Tipo de Sistema} & \textbf{CAPEX} & \textbf{OPEX}& \textbf{Custo de Energia}& \textbf{Vida Útil}  \\ \hline
%Fixo & R\$ 3.120.000,00 & 1,5\% & 0,54 & 25 anos \\ \hline
%Tracker & R\$ 3.744.000,00 & 2,0\%  & 0,54 & 25 anos \\  \hline

%\end{tabular}
%\caption{Parametros Financeitos das Usinas de 800kW}
%\label{tab:exemplo}
%\end{table}


%\subsection{Usina de 1MW}
%\subsubsection{Parâmetros Técnicos}
%Para a realização das simulações no software PV*SOL, foi necessário definir os parâmetros técnicos que caracterizam o sistema fotovoltaico em cada cenário analisado. Esses aspectos incluem a localização geográfica, o modelo de módulo utilizado, a inclinação e orientação dos painéis, a configuração dos rastreadores solares quando aplicáveis, os inversores selecionados e a representação gráfica do arranjo do sistema, garantindo consistência metodológica em todas as análises realizadas.
%\begin{itemize}
    

 %  \item Localização da Usina: As simulações foram realizadas para a cidade de Belo Horizonte – MG, situada na latitude 19,9°S, longitude 43,9°O. A escolha da localidade foi feita considerando sua alta incidência solar e relevância para estudos de viabilidade fotovoltaica.

   % \item Módulo Fotovoltaico: Foi utilizado o módulo Canadian Solar Inc. – CS3Y-500MS, de 500 Wp. Esse modelo foi escolhido por sua ampla disponibilidade no mercado brasileiro e eficiência em torno de 21\%.

  %  \item Inclinação e Orientação: Para os sistemas fixos, os módulos foram orientados para o norte geográfico (0°) e com inclinação de 20°, conforme recomendações do CRESESB para Belo Horizonte.

  %  \item Sistemas com Rastreadores (Trackers): Nos cenários em que foram utilizados trackers, a angulação de movimento considerada foi de -60° a +60°, contemplando o rastreamento solar ao longo do dia.

  %  \item Inversores: Para o sistema de 1MW, foram utilizados cinco inversores SIW500H ST200 H3 da empresa fabricante WEG..

  %  \item adotou-se o modelo simplificado de perdas com percentual global fixo de 14\% aplicado a todos os cenários, assegurando padronização metodológica.

   % \item Representação Gráfica: Para complementar a descrição técnica, cada cenário contou com a elaboração de um diagrama do sistema no PV*SOL, possibilitando a visualização da configuração adotada em termos de módulos, inversores e conexões principais.
    

%\end{itemize}

%\subsubsection{Parâmetos Financeiros}

 %   A Tabela  apresenta um resumo de todos os parâmetros financeiros empregados como dados de entrada nas simulações realizadas durante os testes.

%\begin{table}[h!]
%\centering
%\begin{tabular}{|c|c|c|c|c|}
%\hline
%\textbf{Tipo de Sistema} & \textbf{CAPEX} & \textbf{OPEX}& \textbf{Custo de Energia}& \textbf{Vida Útil}  \\ \hline
%Fixo & R\$ 3.500.000,00 & 1,5\% & 0,54 & 25 anos \\ \hline
%Tracker & R\$ 4.200.000,00 & 2,0\%  & 0,54 & 25 anos \\  \hline

%\end{tabular}
%\caption{Parâmetros Financeiros das Usinas de 1MW}
%\label{tab:exemplo}
%\end{table}



%\section{Conclusões parciais}




\documentclass[
  % -- opções da classe memoir --
  12pt,       % tamanho da fonte
  openright,      % capítulos começam em pág ímpar (insere página vazia caso preciso)
  oneside,      % para impressão em verso e anverso. Oposto a oneside
  a4paper,      % tamanho do papel.
  % -- opções da classe abntex2 --
  %chapter=TITLE,   % títulos de capítulos convertidos em letras maiúsculas
  %section=TITLE,   % títulos de seções convertidos em letras maiúsculas
  %subsection=TITLE,  % títulos de subseções convertidos em letras maiúsculas
  %subsubsection=TITLE,% títulos de subsubseções convertidos em letras maiúsculas
  % -- opções do pacote babel --
  english,      % idioma adicional para hifenização
  french,        % idioma adicional para hifenização
  spanish,     % idioma adicional para hifenização
  brazil        % o último idioma é o principal do documento
  ]{abntex2-deelt-v1}

% ---
% Pacotes básicos
% ---
\usepackage{lmodern}      % Usa a fonte Latin Modern
\usepackage[T1]{fontenc}    % Selecao de codigos de fonte.
\usepackage[utf8]{inputenc}   % Codificacao do documento (conversão automática dos acentos)
\usepackage{lastpage}     % Usado pela Ficha catalográfica
\usepackage{indentfirst}    % Indenta o primeiro parágrafo de cada seção.
\usepackage{color}        % Controle das cores
\usepackage{graphicx}     % Inclusão de gráficos
\usepackage{acronym} %para a gerências de siglas durante o texto
\usepackage{mathtools}
% ---
\usepackage{enumerate}
\usepackage{multirow}

\usepackage{subcaption}  
\usepackage{diagbox}
% ---
% Pacotes de citações
% ---
%\usepackage{epsfig,subfigure}

\usepackage[brazilian,hyperpageref]{backref}	 % Paginas com as citações na bibliografia
\usepackage[alf,bibjustif]{abntex2cite}	% Citações padrão ABNT 6023

% ---
% Informações de dados para CAPA e FOLHA DE ROSTO
% ---
\titulo{ %Dimensionamento e Estudo de Viabilidade Técnica e Econômica de Instalação de Sistemas Fotovoltaicos % 
 Análise da viabilidade técnica e financeira sobre a utilização de sistemas fotovoltaicos fixos ou com rastreador solar  }
\autor{Samuel Breno de Souza}
\local{João Monlevade}
%\dia{20}
\ano{2026}
%\ano{2021} % Indicar apenas o ano da monografia


% ---
% Dados da Universidade e Curso
% ---
\instituicao{Universidade Federal de Ouro Preto\\Instituto de Ciências Exatas e Aplicadas}
\departamento{Departamento de Engenharia Elétrica}
%\colegiado{Colegiado de Engenharia Elétrica}

\curso{Engenharia Elétrica}

\grau{Bacharel em Engenharia Elétrica}

% ---
% Dados de Orientação e Banca de Defesa
% ---
\orientador{Prof. Thainan Santos Theodoro}
\orientadorTitulacao{Doutor em Engenharia Elétrica}
\orientadorDepartamento{DEELT - UFOP}


% Tamanho da linha de assinatura segundo o tamanho do maior nome
% Sugestão: 8cm
\setlength{\ABNTEXsignwidth}{10cm}

% ---
% Tipo de Trabalho
% ---
\tipotrabalho{Monografia (graduação)}

% informações do PDF
\makeatletter
\hypersetup{
     	%pagebackref=true,
		pdftitle={\@title}, 
		pdfauthor={\@author},
    	pdfsubject={\imprimirpreambulo},
	    %pdfcreator={LaTeX with abnTeX2},
		%pdfkeywords={abnt}{latex}{abntex}{abntex2}{trabalho acadêmico}, 
		colorlinks=true,       		% false: boxed links; true: colored links
    	linkcolor=black,          	% color of internal links
    	citecolor=black,        		% color of links to bibliography
    	filecolor=blue,      		% color of file links
		urlcolor=blue,
		bookmarksdepth=4
}

\makeatother
% --- 
% Espaçamentos entre linhas e parágrafos 
% --- 
% O tamanho do parágrafo é dado por:
\setlength{\parindent}{1.3cm}
% Controle do espaçamento entre um parágrafo e outro:
%\setlength{\parskip}{0.2cm}  % tente também \onelineskip


%%%%%%%%%%%%%%%%%%%%%%%%%%%%%%%%%
%                               %
%         Comandos              %
%                               %
%%%%%%%%%%%%%%%%%%%%%%%%%%%%%%%%%
\newcommand{\fonteretirado}[1]{\begin{center}\makebox[\width]{Fonte: Retirado de \citeonline{#1}.}\end{center}}
\newcommand{\fonteadaptado}[1]{\begin{center}\makebox[\width]{Fonte: Adaptado de \citeonline{#1}.}\end{center}}
\newcommand{\fonteautor}{\begin{center}\makebox[\width]{Fonte: Elaborado pelo autor (2026).}\end{center}}
% Como usar
% Retirado: \fonteretirado{inserir_ref}
% Adaptado: \fonteadaptado{inserir_ref}
% Autor:    \fonteautor

% Foo
\usepackage{mdframed}
\newcommand{\startfoo}{%
    \par\medskip
    \begin{mdframed}[linewidth=1pt]%
    \let\figure\figurehere
    \let\endfigure\endfigurehere
    \ignorespaces
}
\newcommand{\stopfoo}{%
    \unskip
    \end{mdframed}%
    \par\medskip
}

%%%%  CONTADORES
\newcounter{contador}
\setcounter{contador}{1} 
\newcommand{\itemsecao}[1]{\vspace{0.2cm} \noindent \textbf{\Alph{contador}) #1} \addtocounter{contador}{1} \par \vspace{0.2cm}}
\newcommand{\itempassos}[1]{\vspace{0.2cm} \noindent \textbf{Passo \arabic{contador}: #1} \addtocounter{contador}{1} \par \vspace{0.2cm}}



% ----
% Início do documento
% ----
\begin{document}

% Seleciona o idioma do documento (conforme pacotes do babel)
%\selectlanguage{english}
\selectlanguage{brazil}

% Retira espaço extra obsoleto entre as frases.
\frenchspacing 

%%%%%%%%%%%%%%%%%%%%%%%%%%%%%%%%%
%                               %
%         Pré textuais          %
%                               %
%%%%%%%%%%%%%%%%%%%%%%%%%%%%%%%%%
\pretextual

\imprimircapa

\imprimirfolhaderosto

\begin{dedicatoria}
	\vspace*{\fill}
	\centering
	\noindent
	\textit{Dedico este trabalho a Deus primeiramente, que me deu condições de poder chegar até aqui, a todos os meus Familiares que me acompanharam nessa trajetória e a Ufop pela infra estrutura e professores de qualidade.  } \vspace*{\fill}
\end{dedicatoria}     %% Dedicatoria.

\begin{agradecimentos}
	



A Deus, pela minha vida, e por me ajudar a ultrapassar todos os obstáculos encontrados ao longo do curso.
Aos meus pais e irmãos, que me incentivaram nos momentos difíceis e compreenderam a minha ausência enquanto eu me dedicava ao curso.
Aos professores pelas correções e ensinamentos que me permitiram apresentar um melhor desempenho no meu processo de formação profissional
Agradeço também a minha avó por ter me acolhido durante toda essa trajetória.
Agradeço a minha namorada, pelo amor e companheirismo




\end{agradecimentos}  %% Agradecimentos.

\begin{epigrafe}
	\vspace*{\fill}
	\begin{flushright}
		\textit{"Be the change you want to see in the world"\\ -- Mahatma Gandhi}
	\end{flushright}
\end{epigrafe}  %% epigrafe.

\setlength{\absparsep}{18pt} % ajusta o espaçamento dos parágrafos do resumo
\begin{resumo}
%O resumo deve apresentar: 
%a)	contexto, 
%b)	lacuna, 
%c)	objetivo, 
%d)	metodologia, 
%e)	resultados e 
%f)	Conclusões e implicações para academia/sociedade. 
% PARAGRAFO UNICO E +- 250 PALAVRAS

%%%%%%%%%%%%%%%%%%%%%
% CONTEXTO
% REDUZIR PARA DOIS PERÍODOS CLAROS
%Nos últimos anos, os sistemas fotovoltaicos ganharam bastante espaço no Brasil, acompanhando uma tendência mundial de busca por fontes renováveis e sustentáveis de energia. Essa tecnologia vem se tornando cada vez mais acessível, eficiente e presente em diferentes escalas, desde pequenas residências até grandes usinas. Além de ser uma fonte limpa, que reduz a emissão de gases de efeito estufa, a energia solar também ajuda a diminuir a dependência de fontes convencionais e a trazer mais segurança energética para o país.
Sistemas fotovoltaicos ganharam bastante espaço no Brasil, acompanhando uma tendência mundial de busca por fontes renováveis e sustentáveis de energia.
%%%%%%%%%%%%%%%%%%%%%
% LACUNA OU PROBLEMA DE PESQUISA
% AUMENTAR PARA DOIS PERIODOS CLAROS
%Com toda essa expansão, surge uma dúvida prática e importante: quando faz sentido usar sistemas fixos e quando é melhor apostar em sistemas com rastreamento solar? Em outras palavras, até que ponto o uso de \textit{trackers} se mostra realmente viável e justificável, levando em conta não apenas o aumento de geração de energia, mas também os custos envolvidos e o retorno do investimento? 
No entanto, não existe consenso sobre o quando é mais vantajoso usar sistemas fixos ou com \textit{trackers} para rastreamento solar e melhor aproveitamento da energia solar.
%%%%%%%%%%%%%%%%%%%%%
% OBJETIVO
% COLOCAR EM UM PERIODO (MÁXIMO DOIS) CLAROS
% DEVE FICAR PARECIDO COM OBJETIVO DA INTRODUÇÃO
Neste sentido, o objetivo deste trabalho é avaliar a viabilidade econômica do uso de \textit{trackers} em sistemas fotovoltaicos, buscando entender a partir de qual faixa de potência eles passam a ser economicamente e tecnicamente vantajosos. Para dar confiabilidade aos resultados, foi utilizada a ferramenta PV*SOL, que é amplamente reconhecida no setor e permite simular diferentes cenários de forma detalhada e próxima da realidade.
%%%%%%%%%%%%%%%%%%%%%
% METODOLOGIA - COMO SERÁ FEITO
% FALAR DAS TÉCNICAS UTILIZADAS E DA FERRAMENTA PRODUZIDA
% DOIS PERIODOS OU MAIS
A metodologia consistiu em simular diversos cenários no PV*SOL, variando a potência dos sistemas entre $10$ kW e $1$ MW para configurações fixas e com rastreamento de um eixo. A partir dessas simulações, foram analisados aspectos técnicos (como geração de energia, eficiência e rendimento) e financeiros (como \textit{payback}, \ac{TIR}, \ac{ROI} e custo de geração). % Essa combinação de análises possibilitou avaliar não só a diferença de desempenho entre os dois tipos de sistema, mas também sua viabilidade no longo prazo.
%\textcolor{red}{(RETIRAR) Este trabalho monográfico teve por objetivo estudar a viabilidade econômica e realizar o dimensionamento de sistemas fotovoltaicos residenciais.}
%%%%%%%%%%%%%%%%%%%%%
% RESULTADOS
% RESUMIR PRINCIPAIS RESULTADOS QUAIS?
% Ex: Os casos de estudo mostraram que ...
Os resultados mostraram que os \textit{trackers} sempre entregam maior produção de energia em comparação aos sistemas fixos, com ganhos médios de $20$ a $25\% $. No entanto, em sistemas menores (até cerca de $300$ kW), esse ganho não compensa o maior custo inicial, fazendo com que os fixos se mostrem mais atrativos financeiramente. A partir de potências intermediárias, como $500$ kW, a diferença começa a diminuir, e acima de 800 kW o rastreamento se torna mais economicamente vantajoso.
%%%%%%%%%%%%%%%%%%%%%
% IMPACTOS - QUEM SE BENEFICIARÁ
% PARECIDO COM JUSTIFICATIVA
%Esse estudo pode beneficiar principalmente investidores do setor de energia solar, ajudando na tomada de decisão sobre qual tecnologia adotar em seus projetos. Também pode servir como referência para o meio acadêmico, contribuindo para novas pesquisas e estudos comparativos, além de ser útil para empresas e profissionais do setor fotovoltaico que buscam otimizar custos e maximizar a eficiência de seus empreendimentos.
Espera-se que este trabalho possa servir como referência para o meio acadêmico, contribuindo para novas pesquisas e estudos comparativos, além de ser útil para empresas e profissionais do setor fotovoltaico que buscam otimizar custos e maximizar a eficiência de seus empreendimentos.
	
	\textbf{Palavras-chave}: Sistemas Fotovoltaicos, Viabilidade, \textit{Trackers}, \textit{Payback}, \ac{TIR}, \ac{ROI}. 
\end{resumo}  %% Resumo.

\begin{resumo}[Abstract]
	\begin{otherlanguage*}{english}
		Photovoltaic systems have gained significant ground in Brazil, following a global trend toward the adoption of renewable and sustainable energy sources. However, there is no consensus on when it is more advantageous to use fixed systems or systems with trackers for solar tracking and improved utilization of solar energy. In this context, the objective of this work is to evaluate the economic feasibility of using trackers in photovoltaic systems, seeking to understand from which power range they become economically and technically advantageous. To ensure the reliability of the results, the PV*SOL tool was used, which is widely recognized in the sector and allows different scenarios to be simulated in a detailed and realistic manner. The methodology consisted of simulating various scenarios in PV*SOL, varying system power between 10 kW and 1 MW for fixed configurations and single-axis tracking. From these simulations, technical aspects (such as energy generation, efficiency, and performance) and financial aspects (such as payback, internal rate of return (IRR), Return on Investment (ROI), and cost of generation) were analyzed. The results showed that trackers always deliver higher energy production compared to fixed systems, with average gains of 20 to 25\%. However, in smaller systems (up to about 300 kW), this gain does not offset the higher initial cost, making fixed systems more financially attractive. From intermediate power levels, such as 500 kW, the difference begins to decrease, and above 800 kW tracking becomes more economically advantageous. It is expected that this work may serve as a reference for the academic community, contributing to new research and comparative studies, as well as being useful for companies and professionals in the photovoltaic sector seeking to optimize costs and maximize the efficiency of their projects.
		%In recent years, photovoltaic systems have gained significant traction in Brazil, following a global trend toward renewable and sustainable energy sources. This technology has become increasingly accessible, efficient, and available on a variety of scales, from small homes to large power plants. Besides being a clean source that reduces greenhouse gas emissions, solar energy also helps reduce dependence on conventional sources and bring greater energy security to the country. With all this expansion, a practical and important question arises: when does it make sense to use fixed systems and when is it better to opt for systems with solar tracking? In other words, to what extent is the use of trackers truly viable and justifiable, considering not only the increase in energy generation but also the costs involved and the return on investment? The objective of this study is to assess the feasibility of using trackers in photovoltaic systems, seeking to understand at what power range they become economically and technically advantageous. To ensure the reliability of the results, the PV*SOL tool was used. This tool is widely recognized in the industry and allows for detailed, realistic simulations of different scenarios. The methodology consisted of simulating various scenarios in PV*SOL, varying system power between 10 kW and 1 MW, and comparing fixed and single-axis tracking configurations. Based on these simulations, technical aspects (such as energy generation, efficiency, and yield) and financial aspects (such as payback, internal rate of return (IRR), ROI, and generation cost) were analyzed. This combination of analyses made it possible to assess not only the performance differences between the two types of systems but also their long-term viability. The results showed that trackers consistently deliver higher energy production compared to fixed systems, with average gains of 20 to 25\%. However, in smaller systems (up to approximately 300 kW), this gain does not offset the higher initial cost, making fixed systems more financially attractive. Starting at intermediate power levels, such as 500 kW, the difference begins to narrow, and above 800 kW, tracking becomes clearly more advantageous, both from a technical and economic perspective. This study can primarily benefit investors in the solar energy sector, helping them decide which technology to adopt for their projects. It can also serve as a reference for academia, contributing to new research and comparative studies, in addition to being useful for companies and professionals in the photovoltaic sector seeking to optimize costs and maximize the efficiency of their projects.
		
		\vspace{\onelineskip} 
		\noindent 
		\textbf{Keywords}: Photovoltaic Systems, Viability, Trackers, Payback, IRR, ROI..
	\end{otherlanguage*}
\end{resumo}  %% Abstract.

% inserir lista de ilustrações
% ---
\pdfbookmark[0]{\listfigurename}{lof}
\listoffigures*
\cleardoublepage
% ---

%% inserir lista de abreviaturas e siglas

%% inserir lista de abreviaturas e siglas
%% ---
%\begin{siglas}
%  \item[ABNT] Associação Brasileira de Normas Técnicas
%  \item[abnTeX] ABsurdas Normas para TeX
%\end{siglas}
%% ---

\chapter*{Lista de Siglas}
%Define as siglas utilizadas nesta dissertação
\begin{acronym}[HOSVD] %lembrar que o argumento opcional é o maior acrônimo utilizado %lembrar de manter a lista em ordem alfabética    

\acro{ANEEL}{Agência Nacional de Energia Elétrica}
\acro{CA}{Corrente Alternada}
\acro{CAPEX}{Despesas de Capital (do inglês, \textit{Capital Expenditure})}
\acro{CC}{Corrente Contínua}
\acro{CIP}{Contribuição para Iluminação Pública}
\acro{Cofins}{Contribuição para o Financiamento da Seguridade Social}
\acro{CRESESB}{Centro de Referência para as Energias Solar e Eólica Sérgio de Salvo Brito}
\acro{EPE}{Empresa de Pesquisa Energética}
\acro{FS}{sistema fotovoltaico (do inglês, \textit{photovoltaic system})}
\acrodefplural{FS}{sistemas fotovoltaicos (do inglês, \textit{photovoltaic systems})}
\acro{GD}{Geração Distribuída}
\acro{ICMS}{Imposto sobre Circulação de Mercadorias e Serviços}
\acro{LCOE}{Custo Nivelado de Energia (do inglês, \textit{Levelized Cost of Energy})}
\acro{OPEX}{Despesas Operacionais (do inglês, \textit{Operational Expenditure})}
\acro{PIS}{Programa de Integração Social}
\acro{ROI}{Retorno sobre o Investimento (do inglês, \textit{Return on Investment})}
\acro{SFCR}{Sistemas Fotovoltaicos Conectados à Rede}
\acro{TIR}{Taxa Interna de Retorno}
\acro{TMA}{Taxa Mínima de Atratividade}
\acro{VPL}{Valor Presente Líquido}
	
	
\end{acronym}
\newpage  %% Acronimos.

%% inserir lista de tabelas
%% ---
\pdfbookmark[0]{\listtablename}{lot}
\listoftables*
\cleardoublepage
%% ---



%% inserir lista de símbolos
\include{./pretextuais/simbolos}  %% Simbolos.

% inserir o sumario
% ---
\pdfbookmark[0]{\contentsname}{toc}
\tableofcontents*
\thispagestyle{empty}
%\cleardoublepage
% ---

%%%%%%%%%%%%%%%%%%%%%%%%%%%%%%%%%
%                               %
%     Corpo da TEXTO            %
%                               %
%%%%%%%%%%%%%%%%%%%%%%%%%%%%%%%%%
\textual % nao remover

\chapter{Introdução}

Este capítulo faz uma introdução sobre conceitos e assuntos introdutórios que envolvem energia solar, mais especificamente \ac{FS}. É feita uma breve contextualização, análise do cenário atual no Brasil. Além disso, é feita uma revisão bibliográfica, apresentação do problema e contribuições do trabalho.

A crise climática tem se tornado uma das maiores preocupações da sociedade contemporânea, impulsionando a busca por soluções energéticas sustentáveis que reduzam as emissões de gases de efeito estufa \cite{ipcc2021}. O aumento da temperatura global, eventos climáticos extremos e a crescente pressão por políticas ambientais rigorosas têm levado governos e setores produtivos a investirem em fontes renováveis de energia, reduzindo a dependência de combustíveis fósseis \cite{cresesbe2025}. Neste contexto, a energia solar fotovoltaica surge como uma alternativa promissora para mitigar os impactos ambientais da geração elétrica tradicional.

Além dos desafios ambientais, o custo da eletricidade tem sido um fator crítico para consumidores e indústrias em todo o mundo. A volatilidade nos preços da energia elétrica, impulsionada por fatores como a escassez hídrica e oscilações nos preços dos combustíveis fósseis, tem aumentado a demanda por soluções que garantam maior previsibilidade e eficiência econômica na geração elétrica \cite{epe2022}. No Brasil, onde grande parte da matriz energética é baseada em hidrelétricas, períodos de seca podem resultar em aumentos tarifários significativos, tornando a diversificação da matriz uma necessidade estratégica \cite{camargos2016analise}.

A energia solar fotovoltaica se destaca nesse cenário por sua abundância, baixo impacto ambiental e viabilidade técnica em diversas regiões do mundo. Nos últimos anos, os avanços tecnológicos e a redução dos custos dos módulos fotovoltaicos tornaram essa fonte cada vez mais competitiva, viabilizando sua adoção em larga escala \cite{lima2024evoluccao}. Além disso, políticas de incentivos e mecanismos regulatórios têm impulsionado investimentos no setor, consolidando a energia solar como uma solução viável para suprir a crescente demanda energética de forma sustentável \cite{lucas2024influencia}. A \autoref{fig:img001}, mostra o quão importante a energia solar vem se tornando na matriz energética brasileira.

\begin{figure}[h]
	\centering
	\caption{Crescimento da Energia Solar no Brasil de 2013 a 2024.}
	\includegraphics[width=0.6\linewidth]{./textuais/figs/img001}
    \fonteretirado{absolar2024}
	\label{fig:img001}
\end{figure}

O Brasil possui um dos maiores potenciais solares do mundo, com uma irradiação média superior a $5$ kWh/m²/dia na maior parte do território \cite{pereira2021energia}. Esse potencial, aliado a políticas de fomento e incentivos à geração distribuída, tem acelerado a expansão da energia solar no país, que já representa uma parcela significativa da matriz elétrica nacional \cite{resolucaoaneel}.A \autoref{fig:fig002} nos mostra o potencial de irradiação solar no Brasil.

\begin{figure}[h]
	\centering
	\caption{Radiação solar no Brasil, plano inclinado - média anual.}
	\includegraphics[width=0.6\linewidth]{./textuais/figs/img002}
    \fonteretirado{absolar2024}
	\label{fig:fig002}
\end{figure}

No entanto, para que o setor alcance sua máxima eficiência e competitividade, torna-se essencial a adoção de tecnologias que otimizem a captação da radiação solar, aumentando a geração de energia sem a necessidade de ampliação da área ocupada pelos sistemas.

Nesse contexto, os sistemas de rastreamento solar \textit{(trackers)} surgem como uma alternativa tecnológica capaz de maximizar a captação da luz solar ao longo do dia, aumentando a eficiência da geração fotovoltaica em comparação aos sistemas fixos. Estudos indicam que os \textit{trackers} podem elevar a geração de energia em até $25$\% para sistemas de eixo único e até $40$\% para sistemas de duplo eixo, dependendo da localização e das condições climáticas \cite{denholm2021challenges}. No entanto, essa tecnologia também implica custos adicionais de instalação, operação e manutenção, o que levanta um questionamento central sobre sua viabilidade financeira em diferentes escalas de potência.

Assim, este trabalho tem como objetivo analisar a viabilidade econômica do uso de \textit{trackers} em sistemas fotovoltaicos, buscando identificar a partir de qual potência instalada essa tecnologia se torna financeiramente vantajosa. Para isso, serão consideradas variáveis como custos de implementação, ganhos energéticos e métricas financeiras, como Retorno do Investimento \textit{(ROI)}, Taxa Interna de Retorno \ac{TIR} e \textit{Payback}. A análise contribuirá para a tomada de decisão no setor fotovoltaico, auxiliando investidores e projetistas a otimizar seus projetos com base em critérios técnicos e econômicos.

%\section{Considerações iniciais} \label{sec:introConsideracoesIniciais}

%\section{Problema de Pesquisa}


%\section{Estado da Arte}



%\textcolor{red}{Nesta seção você fará uma revisão dos principais trabalhos, principalmente artigos, relacionado ao seu. É necessário destacar os objetivos, contribuições, avanços e principalmente lacunas. Use um parágrafo para cada autor/trabalho. Faça uma revisão de pelo menos três.}

%\section{Objetivos}


%\section{Justificativa}





%A \autoref{tab:fatores_escala} mostra os dados \cite{dantas2018viabilidade}. O trabalho de \citeonline{dantas2018viabilidade} mostra...
%\begin{table}[h]
%	\centering
%	\caption{Fatores de escala do controlador %\textit{fuzzy}.}
%	\begin{tabular}{c|c|c}
%		\hline
%		Descrição & Valor                & Unidade\\ %\hline
%		$GE^{-1}$ & $1.5 \cdot 10^{3}$   & adm \\
%		$GC^{-1}$ & $1.0 \cdot 10^{-1}$  & adm\\
%		$GU$      & $1.0 \cdot 10^{6}$   & adm \\
%		$T_s$     & $50.0 \cdot 10^{-6}$ & adm \\ 
%		\hline
%	\end{tabular}
%	\label{tab:fatores_escala}
%	\fonteadaptado{lamberts2013eficiencia}
%\end{table}


\chapter{Fundamentos Teóricos} \label{cap:capitulo_2}
Neste capítulo serão abordados todos os conceitos técnicos e econômicos relevantes sobre micro-geração e mini-geração solar. Com base nesses critérios analisar a viabilidade econômica da utilização de sistemas de geração fixos ou móveis.

\section{Energia Solar Fotovoltaica} \label{sec:AnalisetecsobreSFVCR}
A energia solar fotovoltaica tem se destacado como uma das principais fontes de energia renovável no mundo. Seu crescimento se deve à redução de custos dos módulos fotovoltaicos, às políticas de incentivo governamentais e à necessidade de descarbonização da matriz energética \cite{santos2023crescimento}.

No Brasil, a geração fotovoltaica tem expandido rapidamente devido à alta incidência solar e ao avanço da tecnologia de geração distribuída e centralizada \cite{epe2022}. No entanto, para maximizar a eficiência da geração e reduzir o \ac{LCOE}, estratégias como o uso de sistemas de rastreamento solar têm sido estudadas e implementadas.

\section{Sistemas Fotovoltaicos Fixos e com \textit{Trackers}}
Os sistemas fotovoltaicos podem ser instalados de maneira fixa ou equipados com mecanismos de rastreamento solar, conhecidos como \textit{trackers}. A \autoref{fig:angulos_solar} mostra a definição dos ângulos para orientação e posicionamento de painéis solares fotovoltaicos, com $\gamma$ sendo o ângulo azimutal de superfície entre a normal do painel e o norte geográfico e $\beta$ a inclinação (\textit{tilt}) do painel em relação ao plano horizontal. Os sistemas fixos são aqueles em que os painéis permanecem em uma inclinação ($\beta$) e orientação ($\gamma$) estáticas ao longo do dia e do ano, geralmente otimizadas para maximizar a captação de energia em uma determinada região. Por outro lado, os \textit{trackers} são dispositivos mecânicos e eletrônicos projetados para ajustar continuamente a posição dos painéis fotovoltaicos ($\beta$ e/ou $\gamma$) de forma a manter sua orientação mais alinhada à direção da radiação solar incidente, aumentando a eficiência energética do sistema \cite{dhibi2020reduced}.
\begin{figure}[h]
	\centering
	\caption{Ângulos para orientação e posicionamento de painéis solares fotovoltaicos.}
	\includegraphics[
	width=0.9\linewidth,
	% [trim = <esquerda> <inferior> <direita> <superior>, clip]
	trim = 0.5cm 3.5cm 0.8cm 1.8cm, clip
	]{./textuais/figs/angulos_solar.pdf}
	\fonteautor
	\label{fig:angulos_solar}
\end{figure}

Embora existam diferentes tipos de rastreamento solar, este trabalho concentra sua análise exclusivamente no sistema de um eixo. Essa escolha se justifica pelo fato de ser a configuração mais utilizada em projetos de médio e grande porte no Brasil, apresentando maior acessibilidade e difusão no mercado. Em contraste, os sistemas de dois eixos, apesar de oferecerem ganhos adicionais de eficiência, possuem um custo de implantação elevado e ainda são pouco aplicados na realidade nacional, o que limita sua viabilidade prática \cite{lima2024evoluccao}.

\subsection{Sistemas Fixos}
Os sistemas fotovoltaicos fixos constituem a forma mais tradicional de implantação de usinas solares, tanto em pequena escala (residencial e comercial) quanto em projetos de maior porte. Nesses arranjos, os módulos fotovoltaicos são instalados em estruturas estáticas, com inclinação e azimute ($\beta$ e $\gamma$ respectivamente) definidos de acordo com a latitude e a orientação solar do local, sem movimentação ao longo do dia. A grande vantagem dessa configuração é a simplicidade construtiva, que implica em menor custo inicial (\ac{CAPEX}) e operações de manutenção mais simples e baratas (\ac{OPEX}) em comparação a sistemas com rastreamento solar (\textit{trackers})\cite{absolar2023}.

No Brasil, os sistemas fixos são amplamente utilizados em telhados residenciais e comerciais devido à limitação de espaço e ao fato de apresentarem boa relação custo-benefício. A \ac{ANEEL} destaca que essa tecnologia responde por grande parte da \ac{GD} conectada à rede, sendo o arranjo padrão para a maioria das unidades consumidoras com microgeração \cite{santos2023crescimento}.

Em termos de desempenho energético, a principal limitação dos sistemas fixos é que a captação da radiação solar não é maximizada ao longo do dia, já que a inclinação e o azimute permanecem constantes. Dessa forma, a geração tende a ser maior no período central do dia e menor nas primeiras horas da manhã e no final da tarde. Estudos apontam que, em comparação com sistemas com rastreamento de um eixo, o ganho energético anual dos fixos pode ser de $10$\% a $25$\% menor, a depender da localização geográfica e das condições de irradiância direta e difusa \cite{villalva2012energia}.

Apesar dessa limitação, sistemas fixos ainda se destacam pela maior confiabilidade mecânica e por apresentarem \ac{LCOE} competitivos, especialmente em projetos de menor porte. Segundo a \cite{santana2023analise}, o custo por watt instalado em sistemas fotovoltaicos fixos no Brasil é, em média, inferior ao de sistemas com \textit{trackers}, variando conforme a faixa de potência instalada. Isso faz com que, para sistemas residenciais e comerciais, o arranjo fixo continue sendo a opção mais difundida e economicamente atrativa.

\subsection{ Sistema de Rastreamento Solar (\textit{Tracker})}
Os sistemas de rastreamento solar (\textit{trackers}) são dispositivos que ajustam a inclinação dos painéis fotovoltaicos ao longo do dia para maximizar a captação da radiação solar direta. Diferentemente dos sistemas fixos, que operam com um ângulo de inclinação fixo, os \textit{trackers} acompanham a trajetória do sol, reduzindo as perdas de geração.

\subsubsection{\textit{Tracker} de Eixo Único}
%Os \textit{trackers} de eixo único são os mais amplamente utilizados devido ao seu equilíbrio entre custo e ganho energético. Esse sistema movimenta os módulos fotovoltaicos ao longo de um único eixo (variando $\gamma$), geralmente na direção norte-sul, permitindo que os painéis sigam a trajetória solar no sentido leste-oeste ao longo do dia. Dessa forma, o ângulo de incidência dos raios solares é reduzido em relação ao sistema fixo, proporcionando um aumento de eficiência entre $15$\% e $25$\% dependendo das condições climáticas e da latitude da instalação \cite{de2020analise}.
Os \textit{trackers} de eixo único são os mais amplamente utilizados devido ao equilíbrio favorável entre custo adicional e ganho energético. Nesse arranjo, os módulos fotovoltaicos giram em torno de um único eixo, usualmente orientado na direção norte–sul (variação de $\gamma$ na \autoref{fig:angulos_solar}), permitindo que os painéis acompanhem o movimento aparente do Sol no sentido leste–oeste ao longo do dia. Como consequência, o ângulo de incidência da radiação solar sobre o plano do módulo é reduzido em comparação aos sistemas fixos, resultando em incrementos de geração tipicamente entre $15$\% e $25$\%, a depender da latitude e das condições climáticas locais \cite{de2020analise}.

%Existem dois principais subtipos de \textit{trackers} de eixo único: os horizontais e os verticais. No primeiro caso, o eixo de rotação é paralelo ao solo, sendo amplamente empregado em usinas de grande porte devido à simplicidade mecânica e ao menor consumo de energia dos atuadores. Já no segundo, o eixo é perpendicular ao solo, o que pode ser vantajoso em determinadas latitudes mais altas, onde a variação angular do Sol ao longo do dia é maior \cite{mansouri2020}.
Os \textit{trackers} de eixo único podem ser classificados principalmente em horizontais e verticais, conforme a orientação do eixo de rotação. Nos sistemas horizontais, o eixo é aproximadamente paralelo ao solo, configuração amplamente empregada em usinas de grande porte devido à maior simplicidade mecânica e ao menor consumo energético dos atuadores. Nos sistemas verticais, o eixo é perpendicular ao solo, o que pode apresentar vantagens em latitudes mais elevadas, onde a trajetória solar diária apresenta maior variação azimutal \cite{mansouri2020}.

\begin{comment}
A \autoref{fig:img003} ilustra o princípio de funcionamento de um \textit{tracker} de eixo único horizontal, destacando como os módulos acompanham a posição solar ao longo do dia, e mostrando a diferença entre o mesmo e um sistema fixo.
\begin{figure}[h]
	\centering
	\caption{Comparação do funcionamento de sistemas solares de eixo fixo e rastreamento solar.}
	\includegraphics[width=0.6\linewidth]{./textuais/figs/img003}
    \fonteretirado{AbSolar}
	\label{fig:img003}
\end{figure}
\end{comment}

Além da vantagem de aumentar a produção de energia sem um incremento expressivo nos custos, os \textit{trackers} de eixo único apresentam desafios como a necessidade de manutenção periódica dos motores e sistemas de controle, além de um consumo energético próprio que, embora pequeno, pode impactar a eficiência líquida do sistema. Estudos apontam que esse consumo varia entre $1$\% e $3$\% da produção total da usina, sendo um fator relevante na análise econômica \cite{castro2021construccion}.

\subsubsection{\textit{Trackers} de Eixo Duplo}
%Os \textit{trackers} de eixo duplo, por sua vez, oferecem um grau de otimização ainda maior, pois ajustam a posição dos módulos tanto no eixo horizontal (leste-oeste) quanto no eixo vertical (norte-sul). Isso permite que os painéis estejam constantemente perpendiculares aos raios solares, maximizando a captação da radiação direta. Esse sistema pode proporcionar um incremento de eficiência entre $30\% $ e $40\% $ em relação aos sistemas fixos, sendo especialmente vantajoso em regiões onde a radiação direta tem uma participação elevada no total da radiação incidente\cite{karpic2019comparison}.
Os \textit{trackers} de eixo duplo, por sua vez, proporcionam um grau de otimização superior, pois permitem o ajuste da orientação dos módulos em dois eixos ortogonais, acompanhando simultaneamente as variações diária (leste–oeste, $\gamma$) e sazonal (norte–sul, $\beta$) da posição solar. Dessa forma, a normal do plano do módulo pode ser mantida aproximadamente alinhada à direção da radiação solar direta ao longo do dia e do ano, maximizando a captação energética. Esse tipo de sistema pode proporcionar incrementos de geração tipicamente entre $30\%$ e $40\%$ em relação a arranjos fixos, sendo particularmente vantajoso em regiões onde a componente direta predomina na radiação global incidente \cite{karpic2019comparison}.

%Entretanto, o maior ganho energético dos \textit{tracker} de eixo duplo vem acompanhado de desafios técnicos e financeiros, pois o acréscimo na complexidade mecânica e eletrônica resulta em custos de instalação e manutenção significativamente superiores. Além disso, esses sistemas são mais vulneráveis a falhas mecânicas, dado o número maior de componentes móveis.
Entretanto, o maior ganho energético dos \textit{trackers} de eixo duplo é acompanhado por desafios técnicos e econômicos, uma vez que o aumento da complexidade mecânica e dos sistemas de controle eleva significativamente os custos de investimento, operação e manutenção. Adicionalmente, a presença de múltiplos graus de liberdade e maior número de componentes móveis aumenta a suscetibilidade a falhas e a necessidade de intervenções de manutenção.

% Apesar de sua alta eficiência, a utilização de \textit{trackers} de eixo duplo é menos comum em grandes usinas solares, uma vez que o custo adicional nem sempre é justificado pelo ganho energético. Normalmente, essa tecnologia é mais viável em aplicações de menor escala, como sistemas \textit{off-grid} e instalações onde o espaço disponível para os painéis é limitado \cite{lim2020industrial}. 
Apesar da elevada eficiência energética, a adoção de \textit{trackers} de eixo duplo é menos frequente em usinas fotovoltaicas de grande porte, nas quais o ganho adicional de geração nem sempre compensa o acréscimo de custo ao longo do ciclo de vida do sistema. Em geral, essa tecnologia mostra-se mais viável em aplicações de menor escala, como sistemas \textit{off-grid}, instalações com restrição de área disponível ou aplicações que demandam elevada densidade energética por unidade de superfície \cite{lim2020industrial}.

\begin{comment}
\section{Impacto do Rastreamento Solar na Eficiência Energética}
O uso de sistemas de rastreamento solar, conhecidos como \textit{trackers}, tem sido amplamente estudado como uma solução para maximizar a captação de energia fotovoltaica ao longo do dia. Enquanto sistemas fotovoltaicos fixos permanecem inclinados em um ângulo predeterminado, \textit{trackers} ajustam dinamicamente a posição dos módulos solares, acompanhando a trajetória aparente do sol no céu. Esse movimento permite um aumento na incidência da radiação solar direta sobre as células fotovoltaicas, reduzindo perdas por ângulos desfavoráveis e aumentando a eficiência da conversão energética \cite{javed2020comparison}.


\subsection{Comparação entre Sistemas Fixos e com \textit{trackers}}

Sistemas fotovoltaicos fixos são amplamente utilizados devido à sua simplicidade, menor custo inicial e menor necessidade de manutenção. No entanto, esses sistemas sofrem perdas devido à variação angular da radiação solar ao longo do dia e das estações do ano. Em contrapartida, sistemas com \textit{trackers} podem aumentar a geração de energia em $20$\% a $45$\%, dependendo da tecnologia utilizada e da localização geográfica \cite{mansouri2020}. \textit{Trackers} de eixo único, que acompanham o movimento solar ao longo do eixo leste-oeste, proporcionam ganhos médios de $25$\% a $35$\% em comparação com sistemas fixos. Já os \textit{trackers} de duplo eixo, que ajustam a inclinação tanto no eixo leste-oeste quanto no norte-sul, podem alcançar ganhos superiores a $40$\% em locais de elevada irradiância solar direta \cite{zgraggen2022physics}.

Embora o aumento na geração seja evidente, a decisão entre utilizar sistemas fixos ou \textit{trackers} depende de uma série de fatores, incluindo o custo do equipamento, consumo energético do rastreamento, manutenção e a disponibilidade de incentivos financeiros. Em regiões com alta proporção de radiação difusa, como áreas com clima predominantemente nublado, os benefícios do rastreamento podem ser reduzidos, uma vez que a incidência da luz não provém exclusivamente da posição direta do sol \cite{koussa2011measured}.
\end{comment}

\subsection{Influência da Latitude e do Clima nos Ganhos de Eficiência}
A eficiência do rastreamento solar é diretamente influenciada pela latitude e pelas condições climáticas da região em que o sistema está instalado. Em localidades próximas ao equador, a diferença na captação de energia entre sistemas fixos e móveis tende a ser menor, pois o sol percorre uma trajetória mais próxima ao zênite durante o ano. No entanto, em regiões de latitudes médias e altas, onde a variação da altura solar ao longo das estações é mais acentuada, o uso de \textit{trackers} se torna mais vantajoso \cite{agostinho2023orientaccao}.

Além da latitude, o clima local desempenha um papel essencial na viabilidade do rastreamento solar. Em áreas com alta incidência de céu claro e radiação solar direta, \textit{trackers} apresentam melhor desempenho. Por outro lado, em locais com elevada cobertura de nuvens, a fração de radiação difusa aumenta, reduzindo a eficácia do rastreamento. Estudos indicam que em regiões onde a radiação difusa representa mais de $50$\% da radiação total, a vantagem dos \textit{trackers} diminui consideravelmente \cite{mekhilef2012solar}.

\subsection{Aumento de Produção Energética}
%Diversos estudos têm analisado a eficiência dos \textit{trackers} em diferentes cenários. Em um estudo realizado por \cite{de2020desenvolvimento} no Brasil, foi observado que uma usina solar localizada no Nordeste apresentou um aumento de aproximadamente $32$\% na geração de energia ao adotar um sistema de rastreamento de eixo único, comparado a um sistema fixo. Em outra pesquisa conduzida por Hassan et al. (2018) na Espanha, foi verificado que o uso de \textit{trackers} de duplo eixo resultou em uma elevação da produção energética em até $42$\%, demonstrando a relevância dessa tecnologia em regiões com alto índice de radiação direta.

%Diversos estudos têm investigado o impacto do uso de sistemas de rastreamento solar no desempenho energético de usinas fotovoltaicas, evidenciando ganhos significativos em relação aos sistemas de estrutura fixa. No contexto brasileiro, \cite{de2020desenvolvimento} analisaram uma usina fotovoltaica instalada na região Nordeste do Brasil e constataram que a adoção de um sistema de rastreamento solar de eixo único proporcionou um aumento aproximado de 32\% na geração anual de energia, quando comparado a um sistema fixo de mesma potência instalada. Os autores associam esse ganho à elevada incidência de radiação solar direta característica da região, que favorece a eficiência dos sistemas com rastreamento.
Diversos estudos têm investigado o impacto da adoção de sistemas de rastreamento solar no desempenho energético de usinas fotovoltaicas, evidenciando ganhos expressivos em comparação a arranjos de inclinação fixa. No contexto brasileiro, \citeonline{de2020desenvolvimento} analisaram uma usina fotovoltaica instalada na região Nordeste e verificaram que a utilização de um rastreador de eixo único proporcionou um aumento aproximado de 32\% na geração anual de energia em relação a um sistema fixo de mesma potência instalada. Os autores atribuem esse ganho à elevada fração de radiação direta característica da região, condição que favorece o desempenho de sistemas com rastreamento.

%Resultados semelhantes foram observados em um estudo experimental conduzido por \cite{lima2020analise}, que comparou o desempenho de um sistema fotovoltaico equipado com rastreador solar de um eixo a um sistema fixo operando sob as mesmas condições ambientais. O estudo demonstrou que o sistema com rastreamento apresentou um ganho de geração energética de até $23$\% em relação ao sistema fixo, reforçando a relevância dessa tecnologia para maximizar a captação da radiação solar ao longo do dia, especialmente em locais com elevada disponibilidade de irradiância direta.
Resultados consistentes foram reportados por \citeonline{lima2020analise}, em um estudo experimental que comparou o desempenho de um sistema fotovoltaico com rastreamento de eixo único a um arranjo fixo operando sob as mesmas condições ambientais. Observou-se que o sistema com rastreamento apresentou incremento de geração de até $23\%$ em relação ao sistema fixo, reforçando o potencial dessa tecnologia para maximizar a captação de energia solar ao longo do dia, especialmente em locais com elevada disponibilidade de irradiância direta.

% Esses resultados indicam que, embora os ganhos energéticos proporcionados pelos \textit{trackers} variem de acordo com fatores climáticos, geográficos e tecnológicos, a utilização de sistemas de rastreamento solar pode representar uma estratégia eficaz para o aumento da produção energética em usinas fotovoltaicas, particularmente em regiões de alta insolação.
De modo geral, esses resultados indicam que, embora os ganhos energéticos associados aos \textit{trackers} dependam de fatores climáticos, geográficos e de configuração tecnológica, a adoção de sistemas de rastreamento solar constitui uma estratégia eficaz para ampliar a produção energética de usinas fotovoltaicas, particularmente em regiões de alta insolação.

\section{Custos Associados ao Uso de \textit{trackers} em Usinas Fotovoltaicas}
%A implementação de sistemas com \textit{trackers} em usinas fotovoltaicas traz benefícios significativos em termos de eficiência energética, mas também implica custos adicionais que devem ser analisados para determinar sua viabilidade financeira. 
Os principais componentes de custo envolvem a aquisição e instalação do sistema de rastreamento, os custos operacionais e de manutenção, além do consumo energético necessário para o funcionamento dos motores e mecanismos de ajuste. A análise desses fatores permite avaliar em que cenários a adoção do rastreamento se torna economicamente justificável em comparação com sistemas fixos.

\subsection{Custo de Aquisição e Instalação de \textit{trackers}}
Os custos iniciais de um sistema fotovoltaico com rastreamento são superiores aos de um sistema fixo, devido à necessidade de estruturas mecânicas e componentes eletrônicos adicionais. De acordo com \citeonline{wissmann2024desenvolvimento}, a inclusão de um \textit{trackers} de eixo único pode elevar o custo do investimento inicial em $10$\% a $25$\%, enquanto um sistema de duplo eixo pode representar um aumento de $30$\% a $50$\% no custo total da instalação.

Os principais fatores que influenciam o custo de aquisição e instalação incluem:
\begin{itemize}
	\item Tipo de \textit{trackers}: Sistemas de eixo único são mais baratos e apresentam menor complexidade mecânica, enquanto \textit{trackers} de duplo eixo requerem mecanismos mais sofisticados para permitir ajustes em ambas as direções.
	
	\item Dimensão e capacidade da usina: Grandes usinas fotovoltaicas podem obter economias de escala, reduzindo o custo unitário por megawatt instalado.

	\item Condições do solo e infraestrutura: A instalação de \textit{trackers} requer uma base estrutural mais robusta para suportar os movimentos e garantir estabilidade, o que pode aumentar os custos em terrenos irregulares ou com baixa resistência mecânica.
	
	\item Mão de obra e logística: O transporte e a instalação dos \textit{trackers} envolvem um maior grau de especialização, resultando em custos adicionais em comparação aos suportes fixos tradicionais.
\end{itemize}

%Estudos de caso mostram que o custo de instalação pode variar significativamente de acordo com a localização geográfica e disponibilidade de tecnologia. Segundo um relatório da \textit{International Renewable Energy Agency}  \citeonline{statistics2022international}, o custo médio de instalação de \textit{trackers} de eixo único foi estimado entre US 150 a US 250 dólares por kW, enquanto para sistemas de duplo eixo os valores podem atingir US $300$ a US $450$ dólares por kW.
Estudos de caso indicam que o custo de instalação de sistemas de rastreamento solar pode variar significativamente em função da localização geográfica e da disponibilidade tecnológica. Segundo relatório da \textit{International Renewable Energy Agency} \citeonline{statistics2022international}, o custo médio de instalação de \textit{trackers} de eixo único situa-se tipicamente entre US\$\,150/kW e US\$\,250/kW, enquanto sistemas de eixo duplo podem atingir valores entre US\$\,300/kW e US\$\,450/kW.

\subsection{Custos Operacionais e de Manutenção}
Os sistemas de rastreamento solar demandam custos operacionais e de manutenção mais elevados em comparação com sistemas fixos. Isso ocorre devido à presença de componentes móveis, motores e sensores que requerem inspeção e substituição periódicas. A manutenção corretiva e preventiva é essencial para evitar falhas mecânicas que possam comprometer o desempenho da usina.

\begin{comment}
Os principais custos associados à manutenção incluem:
\begin{itemize}
	\item Lubrificação e ajustes mecânicos: \textit{trackers} possuem peças móveis que sofrem desgaste ao longo do tempo, exigindo lubrificação e reposição de componentes para garantir um funcionamento eficiente.
	
	\item Em sistemas de rastreamento solar, a presença de componentes mecânicos e eletrônicos como motores elétricos, sensores de luminosidade e atuadores adiciona complexidade às operações de manutenção em comparação aos sistemas de estrutura fixa. Esses componentes móveis estão continuamente expostos a condições ambientais adversas, o que aumenta a necessidade de inspeção, manutenção preventiva e, eventualmente, substituição ao longo da vida útil do projeto. Estudos técnicos indicam que \textit{trackers} dependem fortemente de sensores e mecanismos de controle para manter o alinhamento dos módulos com o Sol, o que requer rotinas específicas de manutenção e verificação de funcionamento desses dispositivos para garantir a performance energética e reduzir falhas operacionais \cite{barboza2025tecnologias}\cite{paliyal2024automatic}.
	
	\item Consumo energético: Os sistemas de rastreamento necessitam de energia elétrica para operar os mecanismos de movimentação. Esse consumo representa um custo adicional, geralmente variando entre $1$\% e $5$\% da energia gerada pelo sistema \cite{chang2009output}. Em grandes usinas, essa parcela pode ter impacto significativo na viabilidade econômica.
	
	\item Monitoramento e controle: Sistemas modernos de rastreamento utilizam softwares de controle e monitoramento remoto, os quais demandam infraestrutura tecnológica e mão de obra qualificada para operação contínua.

\end{itemize}

\end{comment}

Os principais custos associados à manutenção de sistemas com \textit{trackers} incluem:

\begin{itemize}
	\item \textbf{Lubrificação e ajustes mecânicos:} Componentes móveis sujeitos a desgaste, como mancais, engrenagens e articulações, requerem lubrificação periódica, ajustes e substituição ao longo da vida útil para garantir o correto posicionamento e a confiabilidade operacional.
	
	\item \textbf{Componentes eletromecânicos e de controle:} Sistemas de rastreamento incorporam motores, atuadores, sensores de posição ou irradiância e controladores eletrônicos, o que aumenta a complexidade de manutenção em relação a estruturas fixas. A exposição contínua a condições ambientais adversas eleva a necessidade de inspeções, manutenção preventiva e eventual substituição. O desempenho dos \textit{trackers} depende da integridade dos sistemas de sensoriamento e controle responsáveis pelo alinhamento dos módulos com a trajetória solar \cite{barboza2025tecnologias, paliyal2024automatic}.
	
	\item \textbf{Consumo energético próprio:} Os mecanismos de acionamento requerem energia elétrica para operação, representando uma perda parasítica tipicamente entre $1\%$ e $5\%$ da energia gerada \cite{chang2009output}. Em usinas de grande porte, essa parcela pode afetar a viabilidade econômica.
	
	\item \textbf{Monitoramento e controle operacional:} Sistemas modernos utilizam plataformas de supervisão e controle remoto que demandam infraestrutura de comunicação, software especializado e mão de obra qualificada para operação e diagnóstico contínuos.
\end{itemize}

De acordo com dados levantados pela Empresa de Pesquisa Energética \cite{epe2017overview}, os custos anuais de operação e manutenção (O\&M) de projetos fotovoltaicos no Brasil variam conforme a tecnologia adotada. No conjunto de projetos analisados no 2ª Leilão de Energia de Reserva (LER/2016), verificou-se que os custos de O\&M anuais representaram, em média, cerca de 0,8\% do investimento total em projetos com estrutura fixa, enquanto os mesmos custos em projetos com rastreadores de eixo único corresponderam a cerca de 1,2\% do investimento total. Essa diferença indica que sistemas com rastreamento solar podem implicar custos de O\&M mais elevados do que sistemas com estrutura fixa, principalmente em função da maior complexidade mecânica e dos componentes adicionais associados aos \textit{trackers}.

\subsection{Impacto do Custo dos \textit{trackers} na Viabilidade Econômica}
A viabilidade financeira do uso de \textit{trackers} depende do equilíbrio entre o aumento da geração de energia e os custos adicionais envolvidos. O retorno sobre o investimento (\ac{ROI}) e o período de \textit{payback} são métricas fundamentais para determinar se o sistema de rastreamento é economicamente atrativo para um determinado projeto.

Para avaliar essa viabilidade, diversos estudos utilizam indicadores como:

\begin{itemize}
	\item \acf{VPL}: Mede a rentabilidade do projeto ao longo do tempo, considerando a diferença entre receitas e custos descontados ao valor presente.
	
	\item \acf{TIR}: Representa a taxa de retorno do investimento e permite compará-lo com outras alternativas de aplicação de capital.
	
	\item Período de \textit{Payback}: Indica o tempo necessário para que o investimento inicial seja recuperado através da economia ou aumento na geração de energia.

\end{itemize}

Estudos empíricos de desempenho econômico têm demonstrado que a adoção de sistemas de rastreamento solar pode reduzir o tempo de retorno do investimento (\textit{payback}) em comparação com arranjos fixos quando há ganhos energéticos relevantes. Em uma análise comparativa realizada por \cite{silva2024tcc}, um sistema fotovoltaico com rastreador apresentou \textit{payback} menor do que um sistema fixo, devido ao aumento de produção energética proporcionado pelo rastreamento, mesmo após considerar custos adicionais de operação e manutenção. Isso sugere que, em regiões com alta irradiância e tarifas de energia elevadas, a utilização de \textit{trackers} pode melhorar a competitividade econômica dos projetos fotovoltaicos em médio e grande porte \cite{silva2024tcc}.

Além disso, relatórios internacionais sobre custos de geração renovável da \textit{International Renewable Energy Agency} \cite{irena2024renewable} indicam uma declinação contínua nos custos de energia solar fotovoltaica, refletida na redução do custo nivelado médio ponderado de eletricidade em projetos solares instalados globalmente, o que favorece a competitividade de tecnologias mais avançadas, como rastreadores solares, em comparação com sistemas convencionais fixos.% \cite{irena2024renewable}.

%Estudos indicam que, em regiões com alta irradiação solar e elevados preços da eletricidade, o uso de \textit{trackers} pode reduzir o payback de uma usina solar em 2 a 4 anos em comparação com um sistema fixo, tornando-se economicamente viável, especialmente para instalações de médio e grande porte \cite{}(Hassan et al., 2018).

%Além disso, o avanço da tecnologia e a redução de custos dos \textit{trackers} nos últimos anos têm contribuído para ampliar sua adoção. Relatórios do setor indicam que o custo médio dos \textit{trackers} tem caído anualmente, favorecendo sua competitividade em relação a sistemas fixos \cite{} (IRENA, 2022).





%%%%%%%%%%%%%%%%%%%%%
%% NÃO MEXER - INICIO
\setcounter{contador}{1} 
%% NÃO MEXER - FIM
%%%%%%%%%%%%%%%%%%%%%

\section{Métricas de Avaliação Econômica e Financeira}
\citeonline{newman2000engineering} alegam que diferentes técnicas de engenharia pode ser usadas na tomada de decisão para investimentos em projetos, mas os aspectos econômicos dominam o problema, sendo, portando, preponderantes na determinação da melhor solução. Conforme \citeonline{budel2017viabilidade}, a analise e a avaliação de projetos são feitas com base nos fluxos de caixa gerados pelos mesmos. Os critérios de análise mais atuais são:
\begin{itemize}
    \item Valor Presente Liquido; 
    \item Taxa interna de retorno; 
    \item \textit{Payback} Simples;
    \item \textit{Payback}Descontado; 
    \item Taxa mínima atrativa
\end{itemize}

\subsection{Valor Presente Liquido}
O método do \ac{VPL} tem como objetivo determinar o valor presente de um investimento a partir de um fluxo de caixa composto por receitas e despesas ao longo do tempo \cite{fanti2015uso}. Segundo \citeonline{silva2015analise}, o \ac{VPL} permite comparar alternativas de investimento por meio do cálculo de seus valores equivalentes no instante inicial, sendo a alternativa com maior valor positivo aquela economicamente mais atrativa. O \ac{VPL} pode ser expresso por \citeonline{bron2007balian}:
\begin{equation}
	V_{PL} = \sum_{t=1}^{n}\frac{FC_t}{(1+i)^t} - FC_0    
    \label{eq:VPL_equation}
\end{equation}
\noindent em que $VPL$ é o valor presente líquido do investimento, em R\$; $FC_t$ é o fluxo de caixa no período $t$, em R\$; $i$ é a taxa mínima de atratividade (\ac{TMA}) adotada pelo investidor, em \% ao período; e $n$ é o horizonte de análise do projeto.

Conforme \citeonline{hess1992spatial}, a \ac{TMA} apresenta caráter subjetivo, pois depende do retorno mínimo exigido pelo investidor para compensar o risco e o custo de oportunidade do capital. Segundo \citeonline{camloffski2014analise}, investidores com perfil conservador ou moderado tendem a adotar taxas próximas às oferecidas por aplicações financeiras de baixo risco, frequentemente associadas à taxa básica de juros. Já investidores com perfil mais agressivo exigem retornos superiores, compatíveis com investimentos de maior risco.

\subsection{Taxa Interna de Retorno}
A \ac{TIR} é definida como a taxa de desconto que torna o valor presente líquido do fluxo de caixa igual a zero, representando a rentabilidade implícita do investimento ao longo de sua vida útil \citeonline{camloffski2014analise}. Um projeto é considerado economicamente viável quando a \ac{TIR} é superior à \ac{TMA}, critério amplamente utilizado na análise de investimentos \cite{casarotto2010analise}. A \ac{TIR} é obtida pela solução da seguinte equação \citeonline{camargos2016analise}:
\begin{equation}
	0 = \sum_{t=1}^{n}\frac{FC_t}{(1+K)^t} - FC_0    
    \label{eq:VPL_equation_1}
\end{equation}

\noindent em que $FC_0$ é o investimento inicial, em R\$; $FC_t$ é o fluxo de caixa no período $t$, em R\$; $\text{TIR}$ é a taxa interna de retorno, em \% ao período; e $n$ é o horizonte de análise.

Em problemas de engenharia econômica, fluxos de caixa ocorrem em diferentes instantes e com magnitudes distintas, exigindo sua equivalência temporal por meio de taxas de juros \cite{hess1992spatial}. Conforme \citeonline{newman2000engineering}, valores monetários em datas distintas podem ser comparados quando convertidos por taxas equivalentes. Taxas equivalentes são aquelas que, mesmo associadas a diferentes períodos de capitalização, produzem o mesmo valor acumulado ao final de um horizonte comum \cite{ferreira2014estimativas}. A relação entre taxas equivalentes é dada por:
\begin{equation}
	i_{eq} = (1 + i_n)^{-n} -1   
    \label{eq:VPL_equation_2}
\end{equation}

\noindent em que $i_{eq}$ é a taxa equivalente no período de referência, em \% ao período; $i_n$ é a taxa no período base, em \% ao período; e $n$ é o número de períodos de capitalização.

\subsection{Payback}
O período de recuperação do capital, ou \textit{Payback} simples, corresponde ao tempo necessário para que o fluxo de caixa acumulado iguale o investimento inicial ao longo da vida útil do projeto \citeonline{de2011research}. Segundo \citeonline{vilela2012analise}, a aceitação de um projeto com base no \textit{Payback} simples é admissível quando o período de retorno do capital é inferior ao limite máximo previamente estabelecido pelo investidor.

Entretanto, conforme destacam \citeonline{blank2009engenharia}, o período de recuperação não deve ser utilizado como principal indicador de decisão, mas sim como uma ferramenta de triagem inicial ou como informação complementar em conjunto com métodos mais robustos, como o \ac{VPL} ou a \ac{TIR}.

Embora constitua um parâmetro prático de avaliação, o \textit{Payback} simples apresenta limitações financeiras relevantes, pois desconsidera o valor do dinheiro no tempo e ignora os fluxos de caixa após o período de recuperação \cite{camloffski2014analise}.

\section{Conclusões parciais}

Este capítulo apresentou o conceito de viabilidade técnica e financeira, destacando as principais vantagens da utilização da energia solar para geração elétrica. Foram abordados o dimensionamento de usinas de geração distribuída do tipo \ac{SFCR}, bem como as principais variáveis de entrada necessárias e os procedimentos para sua obtenção. Também foram discutidos os principais métodos de análise de viabilidade econômica aplicáveis a sistemas fotovoltaicos, incluindo \ac{VPL}, \ac{TIR}, \ac{TMA} e \textit{Payback}. Por fim, descreveram-se as estruturas tarifárias que compõem a fatura de energia elétrica, incluindo \ac{ICMS}, \ac{PIS}, \ac{CIP} e \ac{Cofins}, bem como os acréscimos associados às bandeiras tarifárias vigentes.









\chapter{Metodologia}\label{cap:capitulo_3}
Este capítulo descreve a metodologia empregada para a coleta de dados, a definição dos cenários e a simulação dos sistemas fotovoltaicos, com o objetivo de obter resultados técnicos e econômico-financeiros comparáveis. Inicialmente, apresenta-se a ferramenta utilizada nas simulações. Em seguida, são definidos os cenários e os parâmetros adotados, bem como o procedimento aplicado para extração e análise dos resultados.

\section{PV*SOL}
%Para o desenvolvimento deste trabalho, foi utilizada a ferramenta PVSOL, um software especializado no dimensionamento, simulação e análise de sistemas fotovoltaicos conectados à rede e isolados. O PVSOL foi criado pela empresa alemã Valentin Software GmbH, reconhecida mundialmente pelo desenvolvimento de soluções computacionais voltadas para a modelagem de sistemas de energia renovável, especialmente solar fotovoltaica e térmica.
Para o desenvolvimento deste trabalho, utilizou-se o software PV*SOL, ferramenta voltada ao dimensionamento, simulação e análise de sistemas fotovoltaicos conectados à rede e isolados. O PV*SOL é desenvolvido pela empresa alemã Valentin Software GmbH e é amplamente empregado em aplicações acadêmicas e profissionais, em função da capacidade de simular o desempenho energético considerando dados meteorológicos locais, orientação e inclinação dos módulos, sombreamentos, eficiência de inversores e perdas elétricas, além de incorporar recursos para avaliação econômico-financeira do investimento.

%O PVSOL é amplamente empregado em estudos acadêmicos e profissionais devido à sua capacidade de realizar simulações dinâmicas que consideram fatores como radiação solar, orientação e inclinação dos módulos, sombreamentos parciais, eficiência dos inversores, perdas elétricas, além de aspectos econômicos do investimento. Essas funcionalidades permitem uma análise precisa da geração de energia ao longo do tempo, possibilitando estimativas realistas do desempenho energético e financeiro do sistema projetado.

Dentre suas principais funções, destacam-se:
\begin{itemize}
	\item Modelagem detalhada do arranjo fotovoltaico: escolha do tipo de módulo, inversores, número de strings e configuração elétrica;
	
	\item Análise de sombreamento em 3D: criação de cenários com edificações, árvores e obstáculos que possam impactar a produção de energia;
	
	\item Simulação de geração horária, mensal e anual: possibilitando o cálculo da curva de geração de acordo com dados climáticos locais;
	
	\item Avaliação financeira: análise de custos de investimento, operação e manutenção, fluxo de caixa, tempo de retorno e viabilidade econômica;
	
	\item Comparação de diferentes configurações de sistemas: como sistemas fixos, \textit{trackers} de um eixo e dois eixos, permitindo identificar a opção mais vantajosa tecnicamente e economicamente.
\end{itemize}

Neste contexto, a utilização do PV*SOL neste trabalho se justifica por sua confiabilidade e precisão, sendo uma ferramenta consolidada no mercado internacional e já validada em diversas pesquisas científicas e projetos de engenharia. A aplicação do software possibilita a obtenção de resultados consistentes, fundamentais para a análise comparativa proposta neste estudo. A \autoref{fig:img004}, mostra a tela inicial a ferramenta PV*SOL utilizada neste trabalho. 

\begin{figure}[h]
	\centering
	\caption{Tela Inicial do software PV*SOL.}
	\includegraphics[width=0.99\linewidth]{./textuais/figs/pv001}
    \fonteautor
	\label{fig:img004}
\end{figure}

\section{Cenários propostos}
%Neste estudo, serão realizadas simulações para diferentes potências de sistemas fotovoltaicos ($10$kW, $50$kW, $100$kW, $300$kW, $500$kW e $800$kW e $1$MW ), comparando duas configurações distintas: sistema fixo, sistema com rastreamento de um eixo (\textit{single-axis tracker}). As análises incluirão a geração de energia anual (kWh/ano), os custos de implantação e manutenção, e o tempo de retorno do investimento (\textit{payback}).
Neste estudo, foram realizadas simulações para sistemas fotovoltaicos com diferentes potências instaladas ($10$ kW, $50$ kW, $100$ kW, $300$ kW, $500$ kW, $800$ kW e $1$ MW), comparando duas configurações: (\textit{i}) estrutura fixa e (\textit{ii}) rastreamento solar de um eixo (\textit{single-axis tracker}). Para cada cenário, analisaram-se a geração anual de energia (kWh/ano), os custos de implantação e operação, e o tempo de retorno do investimento (\textit{payback}).

\begin{comment}
A metodologia adotada consistirá nos seguintes passos:
\begin{itemize}
    \item Definição dos parâmetros de simulação: Seleção da localização geográfica, escolha dos módulos fotovoltaicos e inversores, e configuração dos tipos de montagem dos painéis.

\item Simulação no PVSOL: Entrada dos dados técnicos e financeiros no software e execução das simulações para cada cenário.

\item Coleta e organização dos resultados: Extração dos dados gerados pelo PVSOL para análise comparativa entre os sistemas fixo e com rastreamento.

\item Análise dos resultados: Avaliação da viabilidade técnica e econômica do uso de trackers em comparação com sistemas fixos, considerando geração de energia, custos e payback.

\item Elaboração de gráficos e tabelas: Representação visual dos resultados para facilitar a interpretação e a comparação entre os diferentes cenários simulados.
\end{itemize}
\end{comment}

A metodologia adotada consistiu nas seguintes etapas:
\begin{itemize}
	\item Definição dos parâmetros de simulação: localização geográfica, seleção de módulos e inversores e configuração do tipo de montagem;
	\item Simulação no PV*SOL: inserção dos dados técnicos e financeiros e execução das simulações para cada cenário;
	\item Coleta e organização dos resultados: extração dos relatórios do software e consolidação dos dados;
	\item Análise comparativa: avaliação técnica e econômica entre sistemas fixos e com rastreamento, considerando geração, custos e \textit{payback};
	\item Elaboração de gráficos e tabelas: apresentação visual dos resultados para facilitar a interpretação e comparação.
\end{itemize}

%Com essa abordagem, buscamos garantir que os resultados apresentados sejam tecnicamente embasados e aplicáveis a estudos de viabilidade econômica de sistemas fotovoltaicos, contribuindo para a análise do impacto do uso de \textit{trackers} na eficiência e no retorno financeiro dos investimentos em energia solar.
Essa abordagem assegura comparabilidade entre os cenários e fornece base consistente para a análise do impacto do uso de \textit{trackers} no desempenho e no retorno financeiro de sistemas fotovoltaicos.

\section{Fluxograma}
%Com o objetivo de organizar e sistematizar as etapas metodológicas adotadas neste trabalho, a \autoref{fig:img005} apresenta o fluxograma que resume o processo de desenvolvimento do estudo. O fluxograma descreve, de forma sequencial, as principais etapas realizadas, desde a definição da ferramenta de simulação e do local do projeto até a análise e comparação dos resultados obtidos.
Com o objetivo de organizar e sistematizar as etapas metodológicas adotadas, a \autoref{fig:img005} apresenta o fluxograma que resume o desenvolvimento do estudo. O diagrama descreve, de forma sequencial, desde a definição da ferramenta de simulação e dos parâmetros do local do projeto até a extração, análise e comparação dos resultados obtidos.
\begin{figure}[h]
	\centering
	\caption{Fluxograma geral do estudo proposto.}
	\includegraphics[width=0.7\linewidth]{./textuais/figs/Fluxograma.drawio}
    \fonteautor
	\label{fig:img005}
\end{figure}

%As simulações foram conduzidas utilizando o software PV*SOL, contemplando diferentes configurações de sistemas fotovoltaicos, incluindo estruturas fixas e com rastreamento solar. Dessa forma, o fluxograma permite uma visualização clara da metodologia empregada, facilitando a compreensão do encadeamento lógico das atividades e garantindo a reprodutibilidade dos resultados apresentados.
As simulações foram conduzidas no PV*SOL, contemplando configurações com estrutura fixa e com rastreamento solar, permitindo a visualização clara do encadeamento lógico das atividades e favorecendo a reprodutibilidade do procedimento metodológico.

\section{Dimensionamento}
% O dimensionamento de sistemas fotovoltaicos consiste em definir, de maneira técnica e criteriosa, a configuração adequada de todos os componentes do sistema, desde os módulos e inversores até os cabeamentos, arranjos físicos e eventuais transformadores necessários para a conexão à rede elétrica. Esse processo é essencial para garantir a eficiência energética, a segurança operacional e a viabilidade econômica do empreendimento. No presente trabalho, o dimensionamento foi realizado por meio do software PV*SOL, que possibilita a simulação de diferentes cenários, considerando as condições climáticas locais, a orientação dos módulos, as características elétricas dos equipamentos e as normas aplicáveis ao setor. Nas subseções a seguir, são apresentados os critérios adotados para a padronização dos módulos, a escolha dos inversores, o dimensionamento de cabos, a utilização de transformadores e a definição dos arranjos dos painéis fotovoltaicos, justificando tecnicamente cada decisão tomada no processo. A \autoref{fig:img006} mostra a tela inicial do PV*SOL, que é de onde partimos para o dimensionamento dos sistemas fotovoltaico. 
O dimensionamento de sistemas fotovoltaicos consiste em definir, de forma técnica, a configuração dos componentes do sistema, incluindo módulos, inversores, cabeamentos, arranjos físicos e, quando aplicável, transformadores para conexão à rede elétrica. Esse processo é essencial para garantir eficiência energética, segurança operacional e viabilidade econômica. Neste trabalho, o dimensionamento foi realizado por meio do PV*SOL, permitindo simular diferentes cenários com base em condições climáticas locais, orientação dos módulos, características elétricas dos equipamentos e premissas de perdas.

Nas subseções a seguir, apresentam-se os critérios adotados para a padronização dos módulos, seleção dos inversores e definição das perdas, justificando tecnicamente as decisões empregadas. A \autoref{fig:img006} apresenta a tela do PV*SOL utilizada como ponto de partida para o dimensionamento.
\begin{figure}[h]
	\centering
	\caption{Entrada de dados do projeto no PV*SOL.}
	\includegraphics[width=0.99\linewidth]{./textuais/figs/pv002}
    \fonteautor
	\label{fig:img006}
\end{figure}

\subsection{Posicionamento Geográfico dos Módulos}
% As simulações foram realizadas considerando a cidade de Belo Horizonte MG, situada na latitude $19,9$° Sul. Em localidades do hemisfério sul, a orientação mais favorável dos módulos fotovoltaicos é voltada para o norte geográfico, de modo a maximizar a captação da radiação solar ao longo do ano.
As simulações foram realizadas para a cidade de Belo Horizonte (MG), situada na latitude $19{,}9^\circ$ S. No hemisfério sul, a orientação recomendada para maximizar a captação anual é voltada para o norte geográfico, de modo a maximizar a captação da radiação solar ao longo do ano.

% De acordo com as recomendações do CRESESB (Centro de Referência para Energia Solar e Eólica Sérgio Brito), a inclinação ótima dos painéis é de aproximadamente $20$°. Essa configuração possibilita um melhor aproveitamento da irradiação solar global, garantindo o equilíbrio entre a geração de energia nos diferentes períodos sazonais. Dessa forma, a adoção da inclinação de $20$° e orientação para o norte atende ao projeto. A \autoref{fig:img007}, mostra onde é definida a localização geográfica da usina fotovoltaica na ferramenta PV*SOL. 
De acordo com recomendações do \citeonline{cresesbe2025}, a inclinação ótima para Belo Horizonte é próxima de $20^\circ$, o que favorece o aproveitamento da irradiação ao longo do ano. Assim, adotou-se orientação norte geográfico e inclinação de $20^\circ$. A \autoref{fig:img007} mostra a definição da localização no PV*SOL.
\begin{figure}[h]
	\centering
	\caption{Ajuste da localização da usina solar no PV*SOL.}
	\includegraphics[width=0.99\linewidth]{./textuais/figs/pv003}
    \fonteautor
	\label{fig:img007}
\end{figure}

\subsection{Módulos Fotovoltaicos}
%Para garantir uniformidade na análise dos diferentes sistemas simulados, adotou-se a padronização dos módulos fotovoltaicos no software PV*SOL. A utilização de um mesmo modelo de painel, com potência nominal, eficiência e características técnicas bem definidas, permite uma comparação justa entre diferentes cenários de potência instalada e configurações de rastreamento solar. A escolha do módulo considerou critérios como eficiência média do mercado, confiabilidade do fabricante e disponibilidade comercial no Brasil. Segundo \cite{vilela2012analise}, a seleção de módulos deve priorizar não apenas a potência unitária, mas também a eficiência de conversão e o coeficiente de temperatura, que influenciam diretamente no desempenho energético do sistema.
Para assegurar uniformidade e comparabilidade entre os cenários simulados, adotou-se a padronização dos módulos fotovoltaicos no PV*SOL. A utilização de um único modelo, com potência nominal e características técnicas definidas, permite isolar o efeito da potência instalada e da configuração do sistema (estrutura fixa ou com rastreamento) sobre os resultados obtidos.

A escolha do módulo considerou critérios como eficiência representativa do mercado, confiabilidade do fabricante e disponibilidade comercial no Brasil. Conforme \cite{vilela2012analise}, a seleção de módulos deve levar em conta não apenas a potência unitária, mas também a eficiência de conversão e o coeficiente de temperatura, parâmetros que influenciam diretamente o desempenho energético do sistema.

% Para a realização das simulações no software PV*SOL, optou-se pela utilização do módulo \textit{Canadian Solar CS3Y-500MS}, com potência nominal de $500$Wp e tecnologia de células monocristalinas. Esse modelo apresenta eficiência de conversão próxima a $21$\%, garantindo maior aproveitamento da radiação solar incidente e consequente aumento da geração elétrica por área instalada.
Nas simulações, utilizou-se o módulo \textit{Canadian Solar CS3Y-500MS}, de $500$ Wp, com tecnologia monocristalina e eficiência próxima de $21\%$. Esse desempenho contribui para maior aproveitamento da radiação solar incidente e, consequentemente, para maior geração elétrica por unidade de área instalada.

% Além da alta eficiência, o módulo incorpora a tecnologia \textit{half-cell}, que reduz perdas resistivas e melhora a performance em condições de sombreamento parcial, característica relevante para aplicações em usinas de médio e grande porte. A escolha desse módulo justifica-se, ainda, por sua ampla disponibilidade comercial no mercado brasileiro e pela sua representatividade em projetos fotovoltaicos de maior escala.
Além disso, o módulo emprega tecnologia \textit{half-cell}, associada à redução de perdas resistivas e à melhoria do desempenho em condições de sombreamento parcial, aspecto relevante em usinas de médio e grande porte. A adoção de módulos com maior potência unitária também tende a reduzir custos indiretos (cabeamento, estruturas e mão de obra), pois diminui a quantidade de módulos necessária para atingir a mesma potência instalada.

% Outro aspecto relevante é que a adoção de módulos de maior potência unitária contribui para a redução de custos indiretos, como cabeamento, estruturas de suporte e mão de obra, uma vez que se necessita de menor número de painéis para atingir a mesma potência instalada. Dessa forma, a utilização do CS3Y-500MS garante não apenas a confiabilidade e o desempenho energético do sistema, mas também a viabilidade econômica e a consistência metodológica das análises realizadas neste trabalho.A \autoref{fig:img008}, mostra onde é feita a seleção dos módulos na ferramenta PV*SOL. 
Dessa forma, o uso do CS3Y-500MS garante consistência metodológica às simulações e contribui para uma avaliação técnica e econômica mais representativa. A \autoref{fig:img008} apresenta a etapa de seleção dos módulos no PV*SOL.
\begin{figure}[h]
	\centering
	\caption{Seleção do painel fotovoltaico no PV*SOL.}
	\includegraphics[width=0.99\linewidth]{./textuais/figs/pv004}
    \fonteautor
	\label{fig:img008}
\end{figure}

%%Para a simulação do sistema no software PV*SOL, foi necessário determinar parâmetros fundamentais que influenciam diretamente o desempenho e a confiabilidade da instalação. Entre esses aspectos, destacam-se o dimensionamento das strings fotovoltaicas no lado em corrente contínua (CC) e a escolha adequada dos condutores responsáveis pela condução da corrente alternada (CA). Esse processo foi conduzido de forma criteriosa, buscando reduzir perdas elétricas e assegurar que o sistema opere de maneira eficiente e segura.

%No caso do arranjo das strings, definiu-se a quantidade ideal de módulos conectados em série e em paralelo, observando limites de tensão suportados pelo inversor, a faixa operacional de trabalho e as variações climáticas da região de instalação. Esse cuidado garante que os módulos operem dentro de condições adequadas, prevenindo riscos de sobrecarga e mantendo a eficiência global do sistema.

%Quanto ao circuito em corrente alternada, a seleção dos cabos foi realizada considerando fatores como a queda de tensão admissível e a segurança elétrica, em conformidade com os critérios estabelecidos pela norma NBR 5410. A adoção de condutores adequados é essencial para evitar perdas excessivas de energia e para garantir a integridade da instalação ao longo de sua vida útil.

\subsection{Inversores}
%No que se refere aos inversores, sua seleção foi realizada de acordo com a potência de cada sistema simulado, observando-se a disponibilidade de modelos comerciais no Brasil e compatíveis com o banco de dados do PV*SOL. A \autoref{tab:inversores}, mostra os inversores adotados para as respectivas potências estudadas,
A seleção dos inversores foi realizada em função da potência de cada sistema fotovoltaico simulado, considerando a disponibilidade de modelos comerciais no mercado brasileiro e sua compatibilidade com o banco de dados do PV*SOL. A \autoref{tab:inversores} apresenta os inversores adotados para cada faixa de potência instalada analisada neste estudo.

\begin{table}[h!]
\centering
\caption{Inversores utilizados nas simulações.}
\begin{tabular}{C{2cm}cC{2cm}c}
\hline
\textbf{Potência instalada} & \textbf{Fabricante} & \textbf{Quantidade}   & \textbf{Modelo}  \\ \hline
$10$kW & SOLSYSTEMS GPE  & $1$ & SOL GPE $10$kW \\ 
$50$kW & SOLSYSTEMS GPE & $1$ & SOL GPE $50$kW  \\ 
$100$kW & SOLSYSTEMS GPE & $2$ & SOL GPE $50$kW  \\ 
$300$kW & CANADIAN SOLAR INC. & $3$ & CSI-100K-T4001B-E \\ 
$500$kW & CANADIAN SOLAR INC. & $4$ & CSI-125KTL-GI-E \\ 
$800$kW & WEG & $4$ & SIW500H ST200 H3 \\ 
$1$MW & WEG & $5$ & SIW500H ST200 H3 \\ 
\hline
\end{tabular}
\fonteautor
\label{tab:inversores}
\end{table}

%A adoção desses equipamentos não ocorreu de forma aleatória, mas baseou-se em critérios técnicos e de mercado, considerando a confiabilidade dos fabricantes, a disponibilidade no território nacional e a compatibilidade com projetos de diferentes portes. Dessa forma, a padronização estabelecida contribui para a consistência das análises, permitindo comparações precisas entre os diferentes cenários de potência e configuração avaliados neste trabalho. A \autoref{fig:img009}, mostra onde é feita a seleção dos inversores na ferramenta PV*SOL.
A escolha desses equipamentos baseou-se em critérios técnicos e de mercado, incluindo confiabilidade dos fabricantes, disponibilidade nacional e adequação às faixas de potência estudadas. A padronização dos inversores por porte de usina contribui para a consistência metodológica das simulações e permite comparações coerentes entre os diferentes cenários avaliados. A \autoref{fig:img009} mostra a etapa de seleção dos inversores no PV*SOL.
\begin{figure}[h]
	\centering
	\caption{Escolha dos inversores no PV*SOL.}
	\includegraphics[width=0.99\linewidth]{./textuais/figs/pv005}
    \fonteautor
	\label{fig:img009}
\end{figure}

\subsection{Perdas do Sistema}
%O software PV*SOL disponibiliza duas abordagens distintas para a definição das perdas em sistemas fotovoltaicos. A primeira delas é o modelo detalhado, no qual cada tipo de perda pode ser configurado individualmente, contemplando aspectos como perdas ôhmicas em cabos, eficiência dos inversores, perdas por \textit{mismatch}, sombreamento, sujeira, temperatura, entre outros fatores técnicos. Essa abordagem permite maior precisão no dimensionamento, sobretudo em projetos executivos, onde as condições reais da instalação já estão definidas com clareza.
% Por outro lado, o modelo simplificado possibilita a inserção de um único percentual global de perdas, que agrega todos os efeitos mencionados de forma consolidada. Esse método é frequentemente utilizado em estudos de pré-viabilidade ou em análises comparativas, uma vez que simplifica a modelagem sem comprometer a representatividade dos resultados.
O PV*SOL disponibiliza duas abordagens para a definição das perdas em sistemas fotovoltaicos. A primeira é o modelo detalhado, no qual cada parcela de perda pode ser configurada individualmente, incluindo perdas ôhmicas em cabos, eficiência dos inversores, \textit{mismatch}, sombreamento, sujeira e efeitos térmicos. Essa abordagem proporciona maior precisão e é mais indicada em projetos executivos, quando as condições reais da instalação estão plenamente definidas. A segunda abordagem é o modelo simplificado, que permite a adoção de um único percentual global de perdas, agregando os diferentes mecanismos em um valor consolidado. Esse procedimento é amplamente utilizado em estudos de pré-viabilidade e análises comparativas, pois reduz a complexidade da modelagem sem comprometer a representatividade dos resultados.

% Neste trabalho, optou-se pela utilização do modelo simplificado, aplicando um valor global de $14$\% de perdas em todas as simulações realizadas. O valor de $14$\% fundamenta-se em referências amplamente reconhecidas. O \textit{PVGIS (Photovoltaic Geographical Information System)}, ferramenta de referência internacional para estimativas solares, adota esse percentual como padrão para estudos simplificados de desempenho de sistemas fotovoltaicos. Além disso, pesquisas acadêmicas nacionais, como o trabalho de \cite{tonolo2019analise} na Universidade Tecnológica Federal do Paraná (UTFPR), também apresentam valores típicos de perdas globais em faixas de $12$\% a $16$\%, reforçando a adequação do percentual adotado.
Neste trabalho, adotou-se o modelo simplificado, com perdas globais fixadas em $14\%$ para todas as simulações. Esse valor está alinhado a referências reconhecidas na literatura e em ferramentas de estimativa solar. O \textit{PVGIS (Photovoltaic Geographical Information System)} utiliza $14\%$ como valor padrão em análises simplificadas de desempenho fotovoltaico. Estudos acadêmicos nacionais, como \cite{tonolo2019analise}, também reportam perdas globais típicas entre $12\%$ e $16\%$, corroborando a adequação do percentual adotado.

%Portanto, a utilização do modelo simplificado, com perdas fixadas em 14\%, garante a confiabilidade dos resultados, preserva a comparabilidade entre diferentes cenários e mantém o foco no objetivo central deste trabalho: analisar a viabilidade econômica  da utilização de \textit{trackers} em sistemas fotovoltaicos com distintas escalas de potência.A \autoref{fig:img010}, mostra onde é definida as perdas do sistema fotovoltaico na ferramenta PV*SOL. 
A utilização de um valor único de perdas assegura consistência metodológica e comparabilidade entre os cenários analisados, permitindo que as diferenças observadas nos resultados decorram exclusivamente das variáveis investigadas, em especial da configuração estrutural (fixa ou com \textit{tracker}). A \autoref{fig:img010} apresenta a definição das perdas no PV*SOL.
\begin{figure}[h]
	\centering
	\caption{Perdas do sistema no PV*SOL.}
	\includegraphics[width=0.99\linewidth]{./textuais/figs/pv006}
	\fonteautor
	\label{fig:img010}
\end{figure}

%Neste trabalho, optou-se pela utilização do modelo de perdas simplificado no PV*SOL, aplicando um percentual global fixo de 14\% em todas as simulações. Essa decisão está alinhada ao foco principal da pesquisa, que se concentra na avaliação da viabilidade técnica e econômica dos sistemas fotovoltaicos em diferentes escalas de potência e configurações de rastreamento. Assim, evita-se que a análise seja dispersa em um detalhamento minucioso das perdas individuais, permitindo que as comparações entre os cenários tenham como base apenas as variáveis de interesse.

%Embora seja sabido que as perdas técnicas podem variar de acordo com a escala da usina — influenciadas por fatores como distâncias de cabeamento, uso de transformadores e número de inversores —, a adoção de um valor único e padronizado assegura consistência metodológica. Dessa forma, qualquer diferença nos resultados das simulações pode ser atribuída exclusivamente às variáveis estudadas, e não a flutuações decorrentes de diferentes critérios de cálculo de perdas.

%A escolha do valor de 14\% fundamenta-se em referências amplamente reconhecidas. O PVGIS (Photovoltaic Geographical Information System), ferramenta de referência internacional para estimativas solares, adota esse percentual como padrão para estudos simplificados de desempenho de sistemas fotovoltaicos (PVGIS, 2023). No contexto brasileiro, a ANEEL reporta que perdas totais médias no setor elétrico (técnicas e não técnicas) oscilam entre 14\% e 15\%, valor próximo ao utilizado como parâmetro neste estudo (ANEEL, 2021). Além disso, pesquisas acadêmicas nacionais, como o trabalho de Tonolo (2019) na UTFPR, também apresentam valores típicos de perdas globais em faixas de 12\% a 16\%, reforçando a adequação do percentual adotado.

\section{Aspectos financeiros de uma Usina }
\subsection{CAPEX}
% O \ac{CAPEX}, ou despesa de capital, corresponde ao investimento inicial necessário para a implantação de um sistema fotovoltaico, englobando todos os custos relacionados à aquisição, transporte, instalação e comissionamento do empreendimento. Em projetos de geração solar, o \ac{CAPEX} é considerado o principal fator econômico a ser avaliado, uma vez que representa a maior parcela dos desembolsos financeiros do projeto, sendo determinante para a viabilidade econômico-financeira do mesmo \cite{epe2022}.
O \acf{CAPEX}, ou despesa de capital, corresponde ao investimento inicial necessário para a implantação de um sistema fotovoltaico, englobando custos de aquisição de equipamentos, transporte, instalação e comissionamento do empreendimento. Em projetos de geração solar, o \ac{CAPEX} constitui o principal componente econômico a ser avaliado, pois representa a maior parcela dos desembolsos financeiros e influencia diretamente a viabilidade econômico-financeira do projeto \cite{epe2022}.

% Para a determinação do \ac{CAPEX} neste trabalho, adotou-se como referência os valores apresentados por \cite{Sisquini2024}, apresentados na \autoref{tab:customedio}, no estudo os autores apresentam uma análise detalhada dos custos médios de implantação de usinas fotovoltaicas no Brasil, contemplando diferentes portes de sistemas, desde aplicações residenciais até grandes usinas solares.
Para a estimativa do \ac{CAPEX} neste trabalho, adotaram-se como referência os valores apresentados por \cite{Sisquini2024}, sintetizados na \autoref{tab:customedio}, considerando sistemas de eixo fixo. Nesse estudo, os autores apresentam uma análise dos custos médios de implantação de usinas fotovoltaicas no Brasil, considerando diferentes faixas de potência, desde aplicações residenciais até usinas de grande porte.
\begin{table}[htb!]
	\centering
	\caption{Faixas típicas de custo de implantação de sistemas fotovoltaicos de eixo fixo no Brasil, em R\$/Wp instalado.}
	%\caption{Custo médio por watt-pico (Wp) instalado.}
	\begin{tabular}{cc}
	\hline
	\textbf{Porte da Usina (kWp)} & \textbf{Custo Médio (RS/Wp)}  \\ \hline
	Residencial (até $10$ kWp) & $5,50$ - $7,50$  \\ 
	Comercial ($10$ kWp a $500$ kWp) & $4,50$ - $6,50$ \\ 
	Industrial (acima de $500$ kWp) & $3,80$ - $5,80$ \\ 
	Grandes usinas (acima de $1$ MWp) & $3,50$ -$ 5,50$  
	\\ \hline
	\end{tabular}
	\fonteautor
	\label{tab:customedio}
\end{table}

% Já os sistemas com \textit{trackers} exigem componentes mecânicos adicionais e maior complexidade de instalação, o que impacta diretamente no seu custo inicial. Os \textit{trackers} podem elevar o \ac{CAPEX} não apenas pela estrutura móvel em si, mas também por requererem manutenção mais frequente, ocuparem mais área e demandarem cabeamentos e proteções especiais, fatores que não são necessários em sistemas fixos.
Sistemas fotovoltaicos com rastreamento solar (\textit{trackers}) requerem estruturas mecânicas móveis e maior complexidade de instalação, o que eleva o investimento inicial em relação a sistemas fixos. Esse acréscimo decorre não apenas da estrutura de rastreamento, mas também de maior ocupação de área, necessidade de cabeamentos adicionais e requisitos de proteção específicos.

Com o objetivo de embasar a diferença de custos de investimento entre sistemas fotovoltaicos de estrutura fixa e sistemas com rastreamento solar, foi realizada uma revisão de trabalhos acadêmicos nacionais e internacionais que abordam comparações econômicas entre essas configurações para diferentes faixas de potência instalada. A \autoref{tab:capextracker} apresenta a relação de custo entre sistemas com \textit{tracker} e sistemas fixos, obtida a partir de artigos científicos, trabalhos de conclusão de curso, dissertações e publicações em periódicos técnicos. Observa-se que a razão de custo \textit{tracker}/fixo varia conforme a potência do sistema, o escopo do estudo e as premissas adotadas por cada autor, apresentando valores típicos entre aproximadamente $1,15$ e $1,80$. Dessa forma, os dados consolidados permitem estimar um acréscimo percentual no \ac{CAPEX} associado à utilização de rastreadores solares, servindo como base técnica e bibliográfica para a definição dos custos de investimento utilizados nas simulações e análises financeiras desenvolvidas neste trabalho.
\begin{table}[h!]
\centering
\caption{Relação de custo de investimento entre sistemas de eixo fixo/\textit{tracker} por faixa de potência do sistema, compilada a partir de estudos da literatura.}
\begin{tabular}{C{2.2cm}C{2.2cm}C{2.5cm}C{1.5cm}c} \hline
\textbf{Potência do sistema (kW)} & \textbf{Potência do estudo (kW)} & \textbf{Custo relativo tracker/fixo} & \textbf{Média} & \textbf{Fonte} \\ \hline
\multirow{2}{*}{10} & $8,76$ & $1,19$ & $1,195$ & \citeonline{rodrigues2021rastreadores} \\ \cline{2-5}
                    & $3$ & $1,20$ &$ 1,195$ & \citeonline{silvaneto2024seguidor} \\ \hline
\multirow{2}{*}{50} & $75$ & $1,50$ & $1,345$ & \citeonline{nobrega2019comparaccao} \\ \cline{2-5}
					& $75$ & $1,19$ & $1,345$ & \citeonline{pinto2025analise} \\ \hline
\multirow{2}{*}{100}& $100$ & $1,50$ & $1,345$ & \citeonline{nobrega2019comparaccao} \\  \cline{2-5}
					& $75$ & $1,19$ & $1,345$ & \citeonline{pinto2025analise} \\ \hline
\multirow{2}{*}{300}& $300$ & $1,80$ & $1,485$ & \citeonline{araujo2023avaliaccao} \\  \cline{2-5}
					& $300$ & $1,17$ & $1,485$ & \citeonline{casotto2021avaliaccao} \\ \hline
$500$ & $500$ & $1,36$ & $1,36$ & \citeonline{pinto2025analise} \\ \hline
\multirow{2}{*}{800}& $781$ & $1,18$ & $1,19$ & \citeonline{elahi2023_ijred} \\  \cline{2-5}
					& $800$ & $1,20$ & $1,19$ & \citeonline{regia2022_tcc_ufsc} \\ \hline
\multirow{2}{*}{1000}& $1000$ & $1,18$ & $1,165$ & \citeonline{pinto2025analise} \\  \cline{2-5}
					 & $2000$ & $1,15$ &  $1,165$ & \citeonline{de2022analise} \\ \hline
\end{tabular}
\normalsize
\fonteautor
\label{tab:capextracker}
\end{table}

\begin{comment}
\begin{table}[h!]
	\centering
	\caption{Relação de custo de investimento entre sistemas fotovoltaicos com rastreamento solar e sistemas de eixo fixo (tracker/fixo) por faixa de potência instalada, compilada a partir de estudos da literatura.}
	\scriptsize
	\begin{tabular}{C{1.5cm}C{1.5cm}C{1.5cm}C{1.5cm}c}
		\hline
		& \multicolumn{2}{c}{\textbf{Custo relativo tracker/fixo}} & & \\
		\textbf{Potência (kW)} & \textbf{Estudo 1} & \textbf{Estudo 2} & \textbf{Média} & \textbf{Fonte} \\ \hline
		10   & 1,19 & 1,20 & 1,195 & \cite{rodrigues2021rastreadores}; \cite{silvaneto2024seguidor} \\
		50   & 1,50 & 1,19 & 1,345 & \cite{nobrega2019comparaccao}; \cite{pinto2025analise} \\
		100  & 1,50 & 1,19 & 1,345 & \cite{nobrega2019comparaccao}; \cite{pinto2025analise} \\
		300  & 1,80 & 1,17 & 1,485 & \cite{araujo2023avaliaccao}; \cite{casotto2021avaliaccao} \\
		500  & 1,36 & --   & 1,36  & \cite{pinto2025analise} \\
		800  & 1,18 & 1,20 & 1,19  & \cite{elahi2023_ijred}; \cite{regia2022_tcc_ufsc} \\
		1000 & 1,18 & 1,15 & 1,165 & \cite{pinto2025analise}; \cite{de2022analise} \\ \hline
	\end{tabular}
	\normalsize
	\fonteautor
	\label{tab:capextracker}
\end{table}
\end{comment}

\subsection{OPEX}
O \acf{OPEX} corresponde aos custos operacionais recorrentes associados à operação e manutenção de sistemas fotovoltaicos ao longo de sua vida útil, incluindo atividades de manutenção preventiva e corretiva, monitoramento, limpeza, seguros e eventuais substituições de componentes. Em sistemas fotovoltaicos com rastreamento solar, o \ac{OPEX} tende a ser superior ao de estruturas fixas, devido à presença de componentes eletromecânicos adicionais, como motores, sensores e sistemas de controle, que demandam inspeções periódicas, lubrificação e maior frequência de intervenções de manutenção.

Com base no \textit{Caderno de Preços da Geração} (2022/2023) da \citeonline{epe2022}, adotaram-se neste trabalho valores de \ac{OPEX} equivalentes a $1{,}5\%$ do \ac{CAPEX} para sistemas fixos e $2{,}0\%$ do \ac{CAPEX} para sistemas com rastreadores solares. Esses percentuais foram aplicados uniformemente em todos os cenários simulados, garantindo consistência metodológica na comparação econômico-financeira entre as configurações analisadas.

A \autoref{fig:img011} apresenta a etapa de definição dos parâmetros de \ac{CAPEX} e \ac{OPEX} no ambiente de simulação do PV*SOL.
% O \ac{OPEX} (Operational Expenditure) refere-se aos custos operacionais recorrentes, incluindo manutenção, monitoramento, limpeza, seguros e eventuais falhas técnicas. Em sistemas fotovoltaicos com rastreadores solares (\textit{trackers}), os custos de manutenção apresentam um incremento relevante em relação às estruturas fixas, devido à presença de componentes móveis como motores, sensores e sistemas de controle que requerem inspeções periódicas, lubrificação e eventuais reparos. Assim como no Caderno de Preços da Geração (2022/2023)\cite{epe2022} neste trabalho sera utilizado \ac{OPEX} de 1,5\% para sistemas fixos e 2\% para sistemas com (\textit{trackers}). A \autoref{fig:img011}, mostra onde é definido \ac{CAPEX} E \ac{OPEX} do sistema fotovoltaico na ferramenta PV*SOL. 
\begin{figure}[h]
	\centering
	\caption{Definição dos parâmetros econômicos de investimento (\ac{CAPEX}) e custos operacionais (\ac{OPEX}) no PV*SOL, utilizados como dados de entrada nas simulações econômico-financeiras dos sistemas fotovoltaicos analisados.}
	\includegraphics[width=0.99\linewidth]{./textuais/figs/pv007}
    \fonteautor
	\label{fig:img011}
\end{figure}


\section{Estruturação dos Cenários}

Nesta etapa, são apresentados os diferentes cenários de simulação desenvolvidos no software PV*SOL, contemplando variadas potências de sistemas fotovoltaicos e distintas configurações de instalação. Foram considerados arranjos com estrutura fixa, bem como sistemas equipados com rastreadores solares de um eixo, de modo a possibilitar uma análise comparativa tanto em termos de desempenho energético quanto de viabilidade econômica. Essa estruturação permite avaliar de forma consistente o impacto da escala de potência e da tecnologia empregada sobre os indicadores técnicos e financeiros do projeto.

\subsection{Parâmetros Técnicos}
Para a realização das simulações no software PV*SOL, foi necessário definir os parâmetros técnicos que caracterizam o sistema fotovoltaico em cada cenário analisado. Esses aspectos incluem a localização geográfica, o modelo de módulo utilizado, a inclinação e orientação dos painéis, a configuração dos rastreadores solares quando aplicáveis, os inversores selecionados e a representação gráfica do arranjo do sistema, garantindo consistência metodológica em todas as análises realizadas.

\begin{itemize}  
    \item Localização da Usina: As simulações foram realizadas para a cidade de Belo Horizonte – MG, situada na latitude $19,9$°S, longitude $43,9$°O. A escolha da localidade foi feita considerando sua alta incidência solar e relevância para estudos de viabilidade fotovoltaica.

    \item Módulo Fotovoltaico: Foi utilizado o módulo \textit{Canadian Solar Inc.} – CS3Y-500MS, de $500$ Wp. Esse modelo foi escolhido por sua ampla disponibilidade no mercado brasileiro e eficiência em torno de $21$\%.

    \item Inclinação e Orientação: Para os sistemas fixos, os módulos foram orientados para o norte geográfico ($0$°) e com inclinação de $20$°, conforme recomendações do \citeonline{cresesbe2025} para Belo Horizonte.

    \item Sistemas com Rastreadores (\textit{trackers}): Nos cenários em que foram utilizados \textit{trackers}, a angulação de movimento considerada foi de -$60$° a +$60$°, contemplando o rastreamento solar ao longo do dia.
    
    \item Adotou-se o modelo simplificado de perdas com percentual global fixo de $14$\% aplicado a todos os cenários, assegurando padronização metodológica.
\end{itemize}

\subsection{Parâmetros Financeiros}
%Para garantir a consistência e a comparabilidade entre os cenários analisados, adotou-se como premissa que o custo operacional e de manutenção \ac{OPEX}, o custo da energia elétrica e a vida útil dos sistemas serão mantidos iguais para todas as configurações, variando apenas em função das respectivas potências instaladas. Dessa forma, as diferenças observadas nos resultados financeiros refletem exclusivamente o impacto da adoção de estruturas fixas ou com rastreador solar.
Para garantir consistência e comparabilidade entre os cenários analisados, adotou-se como premissa que o \ac{OPEX}, a tarifa de energia elétrica e a vida útil dos sistemas permanecem constantes para todas as configurações, diferenciando-se apenas conforme o tipo de estrutura (fixa ou com rastreador solar). Dessa forma, as variações observadas nos indicadores econômico-financeiros refletem exclusivamente o impacto da configuração estrutural adotada.

%A \autoref{tab:parametrosfinanceiros} apresenta um resumo dos parâmetros financeiros empregados como dados de entrada nas simulações realizadas durante os testes.
A \autoref{tab:parametrosfinanceiros} apresenta os parâmetros financeiros utilizados como dados de entrada nas simulações econômico-financeiras.

\begin{table}[h!]
\centering
\caption{Parâmetros financeiros das simulações.}
\begin{tabular}{ccC{2cm}c}
\hline
\textbf{Tipo de Sistema} &  \textbf{\ac{OPEX}}& \textbf{Custo de Energia}& \textbf{Vida Útil}  \\ \hline
Fixo &  $1,5$\% & $0,54$ & $25$ anos \\ 
Tracker  & $2,0$\%  & $0,54$ & $25$ anos \\  
\hline
\end{tabular}
\fonteautor
\label{tab:parametrosfinanceiros}
\end{table}

% Na sequência,  a \autoref{tab:capex} apresentam os custos de investimento inicial \ac{CAPEX} referentes a cada sistema fotovoltaico analisado, contemplando as configurações de estrutura fixa e com rastreador solar. Esses custos englobam os principais componentes do sistema, como módulos fotovoltaicos, inversores, estruturas de suporte, equipamentos elétricos, instalação e demais despesas associadas à implantação. A apresentação do \ac{CAPEX} por configuração e por potência instalada permite uma comparação direta entre os sistemas, evidenciando o impacto econômico da adoção de estruturas fixas ou com rastreamento solar no investimento inicial do empreendimento.
Na sequência, a \autoref{tab:capex} apresenta os custos de investimento inicial (\ac{CAPEX}) estimados para cada potência instalada e configuração estrutural analisada (fixa e com rastreador). Esses valores incluem os principais componentes do sistema fotovoltaico, como módulos, inversores, estruturas de suporte, equipamentos elétricos e custos de instalação. A apresentação do \ac{CAPEX} por potência e por tipo de estrutura permite avaliar diretamente o impacto econômico da adoção de rastreadores solares no investimento inicial do empreendimento.

\begin{comment}
\begin{table}[h!]
\centering
\caption{Custo de investimento inicial (\ac{CAPEX}) estimado para sistemas fotovoltaicos de diferentes potências instaladas e configurações estruturais (fixa e com rastreador solar), utilizado como entrada nas simulações econômico-financeiras.}
\begin{tabular}{C{3cm}cccc} \hline
\textbf{Potência do Sistema (kW)} &  \textbf{Tipo de Sistema} & \textbf{CAPEX (R\$)}  \\ \hline
\multirow{2}{*}{10}   & Fixo    & 55.000  \\
& Tracker & 66.000  \\ \hline
\multirow{2}{*}{50}   & Fixo    & 255.000 \\
& Tracker & 318.750 \\ \hline
\multirow{2}{*}{100}  & Fixo    & 480.000 \\
& Tracker & 600.000 \\ \hline
\multirow{2}{*}{300}  & Fixo    & 1.350.000 \\
& Tracker & 1.687.000 \\ \hline
\multirow{2}{*}{500}  & Fixo    & 2.100.000 \\
& Tracker & 2.520.000 \\ \hline
\multirow{2}{*}{800}  & Fixo    & 3.120.000 \\
& Tracker & 3.740.000 \\ \hline
\multirow{2}{*}{1000} & Fixo    & 3.500.000 \\
& Tracker & 4.200.000 \\ \hline
\end{tabular}
\fonteautor
\label{tab:capex}
\end{table}
\end{comment}

\begin{table}[h!]
	\centering
	\caption{Custo de investimento inicial (\ac{CAPEX}) estimado para sistemas fotovoltaicos de diferentes potências instaladas e configurações estruturais. A relação Tracker/Fixo indica o acréscimo relativo de investimento associado ao uso de rastreadores solares.}
	\begin{tabular}{cccc}
		\hline
		& \multicolumn{2}{c}{\textbf{CAPEX (R\$)}} & \\
		\textbf{Potência (kW)} & \textbf{Fixo} & \textbf{Tracker} & \textbf{Tracker/Fixo} \\ \hline
		10   & 55.000   & 66.000   & 1,20 \\
		50   & 255.000  & 318.750  & 1,25 \\
		100  & 480.000  & 600.000  & 1,25 \\
		300  & 1.350.000 & 1.687.000 & 1,25 \\
		500  & 2.100.000 & 2.520.000 & 1,20 \\
		800  & 3.120.000 & 3.740.000 & 1,20 \\
		1000 & 3.500.000 & 4.200.000 & 1,20 \\ \hline
	\end{tabular}
\fonteautor
	\label{tab:capex}
\end{table}

Observa-se que o acréscimo de \ac{CAPEX} associado ao uso de rastreadores situa-se aproximadamente entre 20\% e 25\% em todas as faixas de potência analisadas, em concordância com os valores reportados na literatura.

\section{Conclusões parciais}
Neste capítulo foram apresentados os procedimentos metodológicos adotados para a modelagem, simulação e avaliação técnico-econômica de sistemas fotovoltaicos com diferentes potências instaladas e configurações estruturais. A utilização do software PV*SOL permitiu a padronização dos parâmetros técnicos e financeiros, assegurando consistência e reprodutibilidade das simulações realizadas.

No dimensionamento dos sistemas, foram definidos critérios uniformes quanto à localização geográfica, orientação e inclinação dos módulos, seleção de equipamentos e modelagem de perdas, de modo a garantir que as diferenças de desempenho observadas entre os cenários resultem exclusivamente das variáveis de interesse, em especial do uso de estruturas fixas ou com rastreamento solar. A adoção de perdas globais fixas e de um mesmo modelo de módulo fotovoltaico e inversores compatíveis contribuiu para a comparabilidade técnica entre as configurações analisadas.

Do ponto de vista econômico, foram estabelecidos parâmetros financeiros consistentes para todos os cenários, incluindo vida útil do sistema, tarifa de energia e custos operacionais. O \ac{CAPEX} foi estimado a partir de referências nacionais de custos de implantação de sistemas fotovoltaicos, ajustado conforme a potência instalada e acrescido do diferencial associado ao uso de rastreadores solares, conforme valores consolidados na literatura. O \ac{OPEX} foi definido como percentual do investimento inicial, com valores distintos para sistemas fixos e com rastreamento, refletindo a maior complexidade operacional destes últimos.

A estruturação dos cenários contemplou potências representativas de aplicações fotovoltaicas no contexto brasileiro, permitindo analisar de forma abrangente o impacto da escala do sistema e da configuração estrutural nos indicadores técnico-econômicos. Dessa forma, a metodologia adotada estabelece uma base consistente para a análise comparativa de desempenho energético e viabilidade econômica dos sistemas fotovoltaicos estudados, cujos resultados são apresentados e discutidos no capítulo seguinte.



\chapter{Resultados}\label{cap:capitulo_4}
Este capítulo apresenta e discute os resultados obtidos a partir das simulações descritas no capítulo anterior, considerando sistemas fotovoltaicos de diferentes portes e duas configurações de instalação: estrutura fixa e rastreamento solar de um eixo (\textit{single-axis tracker}). As análises contemplam potências de $10$ kW, $50$ kW, $100$ kW, $300$ kW, $500$ kW, $800$ kW e $1$ MW, permitindo avaliar o desempenho técnico e econômico em diferentes escalas. Para cada cenário, foram avaliados a geração anual de energia, o investimento inicial (\ac{CAPEX}), os custos operacionais (\ac{OPEX}), a \ac{TIR} e o tempo de retorno do investimento (\textit{payback}). Em seguida, são apresentadas análises comparativas visando identificar em quais faixas de potência a adoção do rastreamento solar se torna economicamente mais atrativa, fornecendo subsídios técnicos e financeiros para a tomada de decisão em projetos fotovoltaicos.

A fim de evitar repetições e tornar a apresentação dos resultados mais objetiva, as subseções seguintes são organizadas por faixa de potência e apresentam diretamente
os principais resultados técnicos e econômicos de cada cenário, sem a repetição de introduções padronizadas. A discussão é conduzida de forma comparativa sempre que pertinente, com foco na interpretação dos indicadores e nas diferenças entre as configurações analisadas.

%Neste capítulo são apresentados e discutidos os resultados obtidos a partir das simulações realizadas no capitulo anterior, considerando sistemas fotovoltaicos de diferentes portes e suas configurações distintas: sistemas fixos e sistemas com rastreamento solar de um eixo. As análises contemplam potências de $10$ kW, $50$ kW, $100$ kW, $300$ kW, $500$ kW, $800$ kW e $1$MW, permitindo avaliar o desempenho técnico e econômico em diferentes escalas de projeto. Para cada cenário, foram observados parâmetros como a geração anual de energia, o investimento inicial \ac{CAPEX}, os custos operacionais \ac{OPEX}, o \ac{TIR}, o Tempo de Retorno do Investimento \textit{(Payback)} . Após a apresentação dos resultados, são realizadas análises comparativas com o objetivo de identificar em quais faixas de potência a adoção do rastreamento solar se torna mais vantajosa em relação ao sistema fixo, fornecendo subsídios técnicos e financeiros para a tomada de decisão em projetos de usinas fotovoltaicas.

\section{Usina de $10$kW}
\subsection{Resultados técnicos}
%Nesta seção será feita um resumo com os principais resultados técnicos obtidos através da simulação das usinas de $10$kW de eixo fixo e móvel. A \autoref{tab:rt10} mostra os resultados da simulação:

\begin{table}[h!]
	\centering
	\caption{Resultados técnicos obtidos no PV*SOL para a usina de $10$ kW em Belo Horizonte--MG, comparando estrutura fixa e rastreamento solar de um eixo, com energia anual gerada, rendimento específico e eficiência global do sistema.}
	\begin{tabular}{C{3cm}C{3cm}C{3cm}C{3cm}}
	\hline
	\textbf{Sistema} & \textbf{Energia anual [kWh]} & \textbf{Rendimento [kWh/kWp]} & \textbf{Eficiência [\%]} \\ \hline
	Fixo & $14.523$ &  $1.376,15$ & $67,60$\% \\
	\textit{Tracker} & $17.746$ &  $1.698,48$ & $69,73$\% \\ \hline
	\end{tabular}
	\fonteautor
	\label{tab:rt10}
\end{table}

Nas usinas de $10$ kW, observa-se que a utilização de \textit{tracker} proporciona um aumento significativo na produção de energia anual, passando de $14.523$ kWh no sistema fixo para $17.746$ kWh no sistema rastreador. Esse incremento representa aproximadamente $22$\% a mais de geração. O rendimento específico também acompanha essa evolução, variando de $1.376$ kWh/kWp no fixo para $1.698$ kWh/kWp no \textit{tracker}. Em termos de eficiência global do sistema, a diferença é modesta, mas perceptível: o fixo apresenta $67,60$\%, enquanto o \textit{tracker} alcança $69,73$\%.

\subsection{Resultados Financeiros}
%Nesta seção será feita um resumo com os principais resultados financeiros obtidos através da simulação das usinas de $10$kW de eixo fixo e móvel. A \autoref{tab:rf10} mostra os resultados da simulação:

\begin{table}[h!]
	\centering
	\caption{Indicadores econômico-financeiros estimados no PV*SOL para a usina de $10$ kW em Belo Horizonte--MG, comparando estrutura fixa e rastreamento solar de um eixo: custo de geração, \textit{payback}, \ac{TIR} e \ac{ROI}.}
	\begin{tabular}{C{3cm}C{3cm}C{3cm}C{2.5cm}C{2.5cm}} \hline
	\textbf{Sistema} & \textbf{Custo de geração [R\$/kWh]} & \textbf{\textit{PayBack} [anos]} & \textbf{\ac{TIR} [\%]} & \textbf{\ac{ROI} [\%]} \\ \hline
	Fixo & $0,2198$ & $7$ anos, $10$ meses & $11,91$ & $14,26$ \\
	\textit{Tracker} & $0,2321$ & $8$ anos  & $11,71$ & $14,52$ \\ \hline
	\end{tabular}
	\fonteautor
	\label{tab:rf10}
\end{table}

Nas usinas de $10$ kW, o sistema de estrutura fixa apresenta um menor \textit{payback}, de aproximadamente $7$ anos e $10$ meses, enquanto o sistema com \textit{tracker} atinge o retorno em cerca de $8$ anos, indicando uma recuperação do investimento mais rápida para a configuração fixa. Em termos de custo de geração de energia, o sistema fixo também se mostra mais vantajoso, com valor médio de R\$ $0,2198$/kWh, frente a R\$$ 0,2331$/kWh no sistema com \textit{tracker}.

Por outro lado, os indicadores de rentabilidade apresentam valores próximos entre as duas configurações. O sistema fixo registra uma \ac{TIR} de $11,91$\% e \ac{ROI} de $14,26$\%, enquanto o sistema com \textit{tracker} apresenta \ac{TIR}  de $11,71$\%, e \ac{ROI} de $14,52$\%. Apesar de o sistema com apresentar um \ac{ROI} ligeiramente superior, a menor \ac{TIR} e o \textit{payback} mais elevado reforçam que, para a escala de $10$kW, o sistema de estrutura fixa permanece como a alternativa economicamente mais atrativa. 

\section{Usina de $50$kW}
\subsection{Resultados técnicos}
%Nesta seção será feita um resumo com os principais resultados técnicos obtidos através da simulação das usinas de $50$kW de eixo fixo e móvel. A \autoref{tab:rt50} mostra os resultados da simulação:

\begin{table}[h!]
	\centering
	\caption{Resultados técnicos obtidos no PV*SOL para a usina de $50$ kW em Belo Horizonte--MG, comparando estrutura fixa e rastreamento solar de um eixo, com energia anual gerada, rendimento específico e eficiência global do sistema.}
	\begin{tabular}{C{3cm}C{3cm}C{3cm}C{3cm}} \hline
	\textbf{Sistema} & \textbf{Energia anual [kWh]} & \textbf{Rendimento [kWh/kWp]} & \textbf{Eficiência [\%]} \\ \hline
	Fixo & $72.251$ & $1.430,18$ &  $70,25$ \\
	\textbf{Tracker} & $88.295$ &  $1.751,08$ & $71,89$  \\ \hline
	\end{tabular}
	\fonteautor
	\label{tab:rt50}
\end{table}

Nos \ac{SFCR} de $50$ kW, a diferença técnica mantém-se proporcional. O sistema fixo injeta cerca de $72.251$ kWh/ano, ao passo que o \textit{tracker} atinge $88.295$ kWh/ano, representando também um ganho de aproximadamente $22$\%. O rendimento específico sobe de $1.430$ kWh/kWp para $1.751$ kWh/kWp, evidenciando melhor aproveitamento do recurso solar. Já a eficiência, embora próxima, é ligeiramente superior no \textit{tracker} ($71,9$\%) em comparação ao fixo (70,3\%).

\subsection{Resultados Financeiros}
%Nesta seção será feita um resumo com os principais resultados financeiros obtidos através da simulação das usinas de $50$kW de eixo fixo e móvel. A \autoref{tab:rf50} mostra os resultados da simulação:
\begin{table}[h!]
	\centering
	\caption{Indicadores econômico-financeiros estimados no PV*SOL para a usina de $50$ kW em Belo Horizonte--MG, comparando estrutura fixa e rastreamento solar de um eixo: custo de geração, \textit{payback}, \ac{TIR} e \ac{ROI}.}
	\begin{tabular}{C{3cm}C{3cm}C{3cm}C{2.5cm}C{2.5cm}} \hline
		\textbf{Sistema} & \textbf{Custo de geração [R\$/kWh]} & \textbf{\textit{PayBack} [anos]} & \textbf{\ac{TIR} [\%]} & \textbf{\ac{ROI} [\%]} \\ \hline
	Fixo & $0,1961$ & $7$ anos, $3$ meses & $13,07$ & $15,30$ \\ 
	\textbf{Tracker} & $0,234$ &  $8$ anos, $4$ meses & $11,00$ & $13,95$ \\ \hline
	\end{tabular}
	\fonteautor
	\label{tab:rf50}
\end{table}

Nas usinas de $50$ kW, o sistema de estrutura fixa apresenta desempenho econômico superior em relação ao sistema com \textit{tracker}. O \textit{payback} do sistema fixo é de aproximadamente $7$ anos e $3$ meses, enquanto o sistema com \textit{tracker} apresenta um retorno mais longo, de cerca de $8$ anos e $4$ meses. O custo de geração de energia também é significativamente menor no sistema fixo, com valor médio de R\$ $0,1961$/kWh, frente a R\$ $0,234$/kWh no sistema com \textit{tracker}.

No que se refere aos indicadores de rentabilidade, o sistema fixo apresenta \ac{TIR} de $13,07$\% e \ac{ROI} de $15,30$\%, ambos superiores aos obtidos pelo sistema com \textit{tracker}, que registra \ac{TIR} de $11,00$\% e \ac{ROI} de $13,95$\%. Dessa forma, para a escala de $50$ kW, os resultados indicam que o sistema de estrutura fixa é economicamente mais atrativo, apresentando menor custo de geração, retorno do investimento mais rápido e melhores indicadores financeiros globais.

\section{Usina de $100$kW}
\subsection{Resultados técnicos}
%Nesta seção será feita um resumo com os principais resultados técnicos obtidos através da simulação das usinas de $100$kW de eixo fixo e móvel. A \autoref{tab:rt100} mostra os resultados da simulação:
\begin{table}[h!]
	\centering
	\caption{Resultados técnicos obtidos no PV*SOL para a usina de $100$ kW em Belo Horizonte--MG, comparando estrutura fixa e rastreamento solar de um eixo, com energia anual gerada, rendimento específico e eficiência global do sistema.}
	\begin{tabular}{C{3cm}C{3cm}C{3cm}C{3cm}} \hline
		\textbf{Sistema} & \textbf{Energia anual [kWh]} & \textbf{Rendimento [kWh/kWp]} & \textbf{Eficiência [\%]} \\ \hline
Fixo & $144.501$ & $1.430,18$ & $70,25$ \\ 
\textit{Tracker} & $176.590$ & $1.751,08$  & $71,89$  \\ \hline
	\end{tabular}
	\fonteautor
	\label{tab:rt100}
\end{table}

Para as usinas $100$ kW, o cenário se repete: o sistema fixo gera 144.501 kWh/ano, enquanto o \textit{tracker} alcança $176.590$ kWh/ano, mantendo a tendência de 22\% de ganho. O rendimento específico passa de $1.430$ para $1.751$ kWh/kWp, e a eficiência do sistema se mantém superior no rastreador ($71,89$\% contra $70,3$\%).

\subsection{Resultados Financeiros}
%Nesta seção será feita um resumo com os principais resultados financeiros obtidos através da simulação das usinas de $100$kW de eixo fixo e móvel. A \autoref{tab:rf100} mostra os resultados da simulação:
\begin{table}[h!]
	\centering
	\caption{Indicadores econômico-financeiros estimados no PV*SOL para a usina de $100$ kW em Belo Horizonte--MG, comparando estrutura fixa e rastreamento solar de um eixo: custo de geração, \textit{payback}, \ac{TIR} e \ac{ROI}.}
	\begin{tabular}{C{3cm}C{3cm}C{3cm}C{2.5cm}C{2.5cm}} \hline
		\textbf{Sistema} & \textbf{Custo de geração [R\$/kWh]} & \textbf{\textit{PayBack} [anos]} & \textbf{\ac{TIR} [\%]} & \textbf{\ac{ROI} [\%]} \\ \hline
Fixo & $0,1846$ & $6$ anos, $9$ meses & $14,13$ & $16,26$ \\ \hline
\textit{Tracker} & $0,22$ & $7$ anos, $10$ meses  & $11,99$ & $14,83$ \\ \hline
	\end{tabular}
	\fonteautor
	\label{tab:rf100}
\end{table}

Nas usinas de $100$ kW, o sistema de estrutura fixa mantém desempenho econômico superior quando comparado ao sistema com \textit{tracker}. O \textit{payback} do sistema fixo é de aproximadamente $6$ anos e $9$ meses, enquanto o sistema com \textit{tracker} apresenta um retorno mais longo, de cerca de $7$ anos e $10$ meses. O custo de geração de energia do sistema fixo também é inferior, com valor médio de R\$ $0,1846$/kWh, frente a R\$ $0,22$/kWh no sistema com \textit{tracker}.

Em relação aos indicadores de rentabilidade, o sistema fixo apresenta \ac{TIR} de $14,13$\% e \ac{ROI} de $16,26$\%, valores superiores aos observados no sistema com \textit{tracker}, que registra \ac{TIR} de $11,99$\% e \ac{ROI} de $14,83$\%. Dessa forma, para a escala de $100$ kW, os resultados indicam que a configuração com estrutura fixa é economicamente mais atrativa, combinando menor custo de geração, retorno mais rápido do investimento e melhores indicadores financeiros.

\section{Usina de $300$kW}

\subsection{Resultados técnicos}
%Nesta seção será feita um resumo com os principais resultados técnicos obtidos através da simulação das usinas de $300$kW de eixo fixo e móvel. A \autoref{tab:rt300} mostra os resultados da simulação:
\begin{table}[h!]
	\centering
	\caption{Resultados técnicos obtidos no PV*SOL para a usina de $300$ kW em Belo Horizonte--MG, comparando estrutura fixa e rastreamento solar de um eixo, com energia anual gerada, rendimento específico e eficiência global do sistema.}
	\begin{tabular}{C{3cm}C{3cm}C{3cm}C{3cm}} \hline
		\textbf{Sistema} & \textbf{Energia anual [kWh]} & \textbf{Rendimento [kWh/kWp]} & \textbf{Eficiência [\%]} \\ \hline
Fixo & $457.813$ & $1.525,68$ & $74,90$ \\
\textit{Tracker} & $557.047$ & $1.856,46$  & $76,20$  \\ \hline
	\end{tabular}
	\fonteautor
	\label{tab:rt300}
\end{table}

Nas usinas de $300$ kW, a produção sobe de 457.813 kWh/ano no fixo para $557.047$ kWh/ano no \textit{tracker}. O ganho percentual em relação à geração é de aproximadamente $22$\%, mas aqui a melhoria no rendimento específico é ainda mais destacada: $1.525$ kWh/kWp no fixo contra $1.856$ kWh/kWp no \textit{tracker}. Além disso, a eficiência do sistema aumenta, passando de $74,9$\% para $76,2$\%.

\subsection{Resultados Financeiros}
%Nesta seção será feita um resumo com os principais resultados financeiros obtidos através da simulação das usinas de $300$kW de eixo fixo e móvel. A \autoref{tab:rf300} mostra os resultados da simulação:
\begin{table}[h!]
	\centering
	\caption{Indicadores econômico-financeiros estimados no PV*SOL para a usina de $300$ kW em Belo Horizonte--MG, comparando estrutura fixa e rastreamento solar de um eixo: custo de geração, \textit{payback}, \ac{TIR} e \ac{ROI}.}
	\begin{tabular}{C{3cm}C{3cm}C{3cm}C{2.5cm}C{2.5cm}} \hline
		\textbf{Sistema} & \textbf{Custo de geração [R\$/kWh]} & \textbf{\textit{PayBack} [anos]} & \textbf{\ac{TIR} [\%]} & \textbf{\ac{ROI} [\%]} \\ \hline
Fixo & $0,1622$ & $5$ anos, $11$ meses & $16,34$ & $18,31$ \\
\textit{Tracker} & $0,2133$ & $7$ anos, $8$ meses & $12,24$ & $15,06$ \\ \hline
	\end{tabular}
	\fonteautor
	\label{tab:rf300}
\end{table}

Nas usinas de $300$ kW, o sistema de estrutura fixa apresenta vantagem econômica significativa em relação ao sistema com \textit{tracker}. O \textit{payback} do sistema fixo é de aproximadamente $5$ anos e $11$ meses, enquanto o sistema com \textit{tracker} apresenta um retorno mais prolongado, de cerca de $7$ anos e $8$ meses. O custo de geração de energia também se mostra inferior no sistema fixo, com valor médio de R\$ $0,1622$/kWh, frente a R\$ $0,2133$/kWh no sistema com \textit{tracker}.

Quanto aos indicadores de rentabilidade, o sistema fixo registra \ac{TIR} de $16,34$\% e \ac{ROI} de $18,31$\%, ambos substancialmente superiores aos obtidos pelo sistema com \textit{tracker}, que apresenta \ac{TIR} de $12,24$\% e \ac{ROI} de $15,06$\%. Assim, para a escala de $300$ kW, os resultados indicam de forma clara que o sistema de estrutura fixa é economicamente mais atrativo, apresentando menor custo de geração, retorno do investimento mais rápido e melhores indicadores financeiros globais.

\section{Usina de $500$kW}
\subsection{Resultados técnicos}
%Nesta seção será feita um resumo com os principais resultados técnicos obtidos através da simulação das usinas de $500$kW de eixo fixo e móvel. A \autoref{tab:rt500} mostra os resultados da simulação:
\begin{table}[h!]
	\centering
	\caption{Resultados técnicos obtidos no PV*SOL para a usina de $500$ kW em Belo Horizonte--MG, comparando estrutura fixa e rastreamento solar de um eixo, com energia anual gerada, rendimento específico e eficiência global do sistema.}
	\begin{tabular}{C{3cm}C{3cm}C{3cm}C{3cm}} \hline
		\textbf{Sistema} & \textbf{Energia anual [kWh]} & \textbf{Rendimento [kWh/kWp]} & \textbf{Eficiência [\%]} \\ \hline
Fixo & $769.662$ & $1.539,20$ & $75,61$ \\ 
\textit{Tracker} & $935.237$ & $1.870,36$  & $76,80$  \\ \hline
	\end{tabular}
	\fonteautor
	\label{tab:rt500}
\end{table}

Nas usinas de $500$ kW, observa-se que o sistema fixo atinge $769.662$ kWh/ano, enquanto o \textit{tracker} chega a $935.237$ kWh/ano, consolidando um incremento de $22$\%. O rendimento também cresce de $1.539$ kWh/kWp para $1.870$ kWh/kWp, com melhoria na eficiência, que passa de $75,61$\% para $76,80$\%.

\subsection{Resultados Financeiros}
%Nesta seção será feita um resumo com os principais resultados financeiros obtidos através da simulação das usinas de $500k$W de eixo fixo e móvel. A \autoref{tab:rf500} mostra os resultados da simulação:
\begin{table}[h!]
	\centering
	\caption{Indicadores econômico-financeiros estimados no PV*SOL para a usina de $500$ kW em Belo Horizonte--MG, comparando estrutura fixa e rastreamento solar de um eixo: custo de geração, \textit{payback}, \ac{TIR} e \ac{ROI}.}
	\begin{tabular}{C{3cm}C{3cm}C{3cm}C{2.5cm}C{2.5cm}} \hline
		\textbf{Sistema} & \textbf{Custo de geração [R\$/kWh]} & \textbf{\textit{PayBack} [anos]} & \textbf{\ac{TIR} [\%]} & \textbf{\ac{ROI} [\%]} \\ \hline
Fixo & $0,1637$ & $5$ anos, $7$ meses & $17,37$ & $19,79$ \\ 
\textit{Tracker} & $0,1832$ & $6$ anos, $4$ meses & $15,14$ & $17,68$ \\ \hline
	\end{tabular}
	\fonteautor
	\label{tab:rf500}
\end{table}

Nas usinas de $500$ kW, o sistema de estrutura fixa mantém desempenho econômico superior em relação ao sistema com \textit{tracker}. O \textit{payback} do sistema fixo é de aproximadamente $5$ anos e $7$ meses, enquanto o sistema com \textit{tracker} apresenta um retorno mais longo, de cerca de $6$ anos e $4$ meses. O custo de geração de energia do sistema fixo também é inferior, com valor médio de R\$ $0,1637$/kWh, frente a R\$ $0,1832$/kWh no sistema com \textit{tracker}.

No que se refere aos indicadores de rentabilidade, o sistema fixo apresenta \ac{TIR} de $17,37$\% e \ac{ROI} de $19,79$\%, ambos superiores aos obtidos pelo sistema com \textit{tracker}, que registra \ac{TIR} de $15,14$\% e \ac{ROI} de $17,68$\%. Dessa forma, para a escala de $500$ kW, os resultados indicam que o sistema de estrutura fixa é economicamente mais atrativo, apresentando menor custo de geração, retorno mais rápido do investimento e melhores indicadores financeiros globais.

\section{Usina de $800$kW}
\subsection{Resultados técnicos}
%Nesta seção será feita um resumo com os principais resultados técnicos obtidos através da simulação das usinas de $800$kW de eixo fixo e móvel. A \autoref{tab:rt800} mostra os resultados da simulação:
\begin{table}[h!]
	\centering
	\caption{Resultados técnicos obtidos no PV*SOL para a usina de $800$ kW em Belo Horizonte--MG, comparando estrutura fixa e rastreamento solar de um eixo, com energia anual gerada, rendimento específico e eficiência global do sistema.}
	\begin{tabular}{C{3cm}C{3cm}C{3cm}C{3cm}} \hline
		\textbf{Sistema} & \textbf{Energia anual [kWh]} & \textbf{Rendimento [kWh/kWp]} & \textbf{Eficiência [\%]} \\ \hline
Fixo & $1.202.618$ & $1.538,66$ & $73,84$ \\ 
\textit{Tracker} & $1.498.156$ & $1.872,61$  & $76,88$ \\ \hline
	\end{tabular}
	\fonteautor
	\label{tab:rt800}
\end{table}

Nas usinas de $800$ kW, a diferença é ainda mais expressiva: o sistema fixo gera $1.202.618$ kWh/ano, enquanto o \textit{tracker}) atinge $1.498.156$ kWh/ano, o que corresponde a cerca de $25$\% de aumento na produção. O rendimento específico cresce de $1.538$ kWh/kWp para $1.872$ kWh/kWp, acompanhado de uma elevação na eficiência global ($73,84$\% no fixo contra $76,88$\% no \textit{tracker}).

\subsection{Resultados Financeiros}
%Nesta seção será feita um resumo com os principais resultados financeiros obtidos através da simulação das usinas de $800$kW de eixo fixo e móvel. A \autoref{tab:rf800} mostra os resultados da simulação:
\begin{table}[h!]
	\centering
	\caption{Indicadores econômico-financeiros estimados no PV*SOL para a usina de $800$ kW em Belo Horizonte--MG, comparando estrutura fixa e rastreamento solar de um eixo: custo de geração, \textit{payback}, \ac{TIR} e \ac{ROI}.}
	\begin{tabular}{C{3cm}C{3cm}C{3cm}C{2.5cm}C{2.5cm}} \hline
		\textbf{Sistema} & \textbf{Custo de geração [R\$/kWh]} & \textbf{\textit{PayBack} [anos]} & \textbf{\ac{TIR} [\%]} & \textbf{\ac{ROI} [\%]} \\ \hline
Fixo & $0,1427$ & $5$ anos, $2$ meses & $18,97$ & $20,81$ \\ 
\textit{Tracker} & $0,1487$ & $5$ anos, $1$ mês  & $19,47$ & $21,79$ \\ \hline
	\end{tabular}
	\fonteautor
	\label{tab:rf800}
\end{table}

Nas usinas de $800$ kW, observa-se uma maior proximidade entre os resultados econômicos das duas configurações analisadas. O \textit{payback} do sistema de estrutura fixa é de aproximadamente $5$ anos e $2$ meses, enquanto o sistema com \textit{tracker} apresenta um retorno ligeiramente mais rápido, de cerca de $5$ anos e $1$ mês. Em relação ao custo de geração de energia, o sistema fixo leva vantagem, com valor médio de R\$ $0,1427$/kWh, frente a R\$ $0,1487$/kWh no sistema com \textit{tracker}.

No entanto, ao analisar os indicadores de rentabilidade, o sistema com \textit{tracker} passa a apresentar resultados superiores, registrando \ac{TIR} de $19,47$\% e \ac{ROI} de $21,79$\%, enquanto o sistema fixo apresenta \ac{TIR} de $18,97$\% e \ac{ROI} de $20,81$\%. Dessa forma, para a escala de 800 kW, os resultados indicam um cenário de equilíbrio econômico entre as configurações, no qual o sistema fixo mantém menor custo de geração, enquanto o uso de \textit{tracker} proporciona maior rentabilidade percentual. Assim, a escolha da configuração mais adequada depende do critério econômico priorizado pelo investidor, caracterizando um cenário de equilíbrio econômico entre as alternativas.

\section{Usina de $1$MW}
\subsection{Resultados técnicos}
%Nesta seção será feita um resumo com os principais resultados técnicos obtidos através da simulação das usinas de $1$MW de eixo fixo e móvel.
A \autoref{tab:rt1} mostra os resultados da simulação:
\begin{table}[h!]
	\centering
	\caption{Resultados técnicos obtidos no PV*SOL para a usina de $1$ MW em Belo Horizonte--MG, comparando estrutura fixa e rastreamento solar de um eixo, com energia anual gerada, rendimento específico e eficiência global do sistema.}
	\begin{tabular}{C{3cm}C{3cm}C{3cm}C{3cm}} \hline
		\textbf{Sistema} & \textbf{Energia anual [kWh]} & \textbf{Rendimento [kWh/kWp]} & \textbf{Eficiência [\%]} \\ \hline
Fixo & $1.538.740$ & $1.538,66$ & $75,58$ \\ 
\textit{Tracker} & $1.872.695$ & $1.872,61$  & $76,90$  \\ \hline
	\end{tabular}
	\fonteautor
	\label{tab:rt1}
\end{table}

Nas usinas de $1$MW, a produção do sistema fixo alcança $1.538.740$ kWh/ano, enquanto o \textit{tracker} chega a $1.872.695$ kWh/ano, representando novamente um ganho de $22$\%. O rendimento específico cresce de $1.538$ para $1.872$ kWh/kWp, com a eficiência global também sendo superior no rastreador ($76,90$\% contra $75,58$\% no fixo).

\subsection{Resultados Financeiros}
%Nesta seção será feita um resumo com os principais resultados financeiros obtidos através da simulação das usinas de $1$MW de eixo fixo e móvel. A \autoref{tab:rf1} mostra os resultados da simulação:
\begin{table}[h!]
	\centering
	\caption{Indicadores econômico-financeiros estimados no PV*SOL para a usina de $1$ MW em Belo Horizonte--MG, comparando estrutura fixa e rastreamento solar de um eixo: custo de geração, \textit{payback}, \ac{TIR} e \ac{ROI}.}
	\begin{tabular}{C{3cm}C{3cm}C{3cm}C{2.5cm}C{2.5cm}} \hline
		\textbf{Sistema} & \textbf{Custo de geração [R\$/kWh]} & \textbf{\textit{PayBack} [anos]} & \textbf{\ac{TIR} [\%]} & \textbf{\ac{ROI} [\%]} \\ \hline
Fixo & $0,1251$ & $4$ anos, $6$ meses & $21,99$ & $23,74$ \\ 
\textit{Tracker} & $0,13$ & $4$ anos, $4$ meses  & $22,67$ & $24,91$ \\ \hline
	\end{tabular}
	\fonteautor
	\label{tab:rf1}
\end{table}

Nas usinas de $1$ MW, o sistema com estrutura fixa e o sistema com \textit{tracker} apresentam desempenho econômico bastante próximo, com leve vantagem para a configuração com rastreamento solar em alguns indicadores. O \textit{payback} do sistema fixo é de aproximadamente $4$ anos e $6$ meses, enquanto o sistema com \textit{tracker} apresenta um retorno ligeiramente mais rápido, de cerca de $4$ anos e $4$ meses. Quanto ao custo de geração de energia, o sistema fixo mantém valor inferior, com média de R$ $0,1251$/kWh, frente a R$ $0,13$/kWh no sistema com \textit{tracker}.

Em relação aos indicadores de rentabilidade, o sistema com \textit{tracker} apresenta \ac{TIR} de $22,67$\% e \ac{ROI} de $24,91$\%, superiores aos valores obtidos pelo sistema fixo, que registra \ac{TIR} de $21,99$\% e \ac{ROI} de $23,74$\%. Dessa forma, para a escala de $1$ MW, os resultados indicam que, apesar do ligeiro aumento no custo de geração, o uso de \textit{trackers} se torna economicamente mais atrativo, proporcionando maior retorno financeiro e menor tempo de recuperação do investimento.

\section{Resumo dos Resultados}
\subsection{Resumo Técnico das Usinas}

Com o intuito de facilitar a visualização e a consulta dos principais resultados técnicos obtidos nas simulações, a Tabela 20 apresenta um resumo dos parâmetros energéticos das usinas fotovoltaicas analisadas. Os dados estão organizados de forma a permitir uma comparação direta entre as diferentes potências instaladas e as configurações de sistema consideradas, reunindo informações referentes à energia anual gerada, rendimento específico e eficiência dos sistemas. Dessa forma, a \autoref{tab:resumot} atua como um instrumento de apoio às análises técnicas já realizadas e às discussões que serão aprofundadas nas seções subsequentes deste capítulo.

\begin{table}[h!]
\centering
\caption{Resumo dos resultados técnicos obtidos no PV*SOL para usinas fotovoltaicas de diferentes potências em Belo Horizonte--MG, comparando sistemas de estrutura fixa e com rastreamento solar de um eixo.}
\begin{tabular}{C{2cm}C{2cm}C{3cm}C{3cm}C{2cm}} \hline
\textbf{Potência} &\textbf{Sistema} & \textbf{Energia anual [kWh]} & \textbf{Rendimento [kWh/kWp]} & \textbf{Eficiência [\%]} \\ \hline
\multirow{2}{*}{$10$ kW} & Fixo & 14.523 & 1.376,15 & 67,60 \\ 
& \textit{Tracker} & 17.746 & 1.698,48 & 69,73 \\ \hline
\multirow{2}{*}{$50$ kW} & Fixo & 72.251 & 1.430,18 & 70,25 \\ 
& \textit{Tracker} & 88.295 & 1.751,08 & 71,89 \\ \hline
\multirow{2}{*}{$100$ kW} & Fixo & 144.501 & 1.430,18 & 70,25 \\ 
& \textit{Tracker} & 176.590 & 1.751,08 & 71,89 \\ \hline
\multirow{2}{*}{$300$ kW} & Fixo & 457.813 & 1.525,68 & 74,90 \\ 
& \textit{Tracker} & 557.047 & 1.856,46 & 76,20 \\ \hline
\multirow{2}{*}{$500$ kW} & Fixo & 769.662 & 1.539,20 & 75,61 \\ 
& \textit{Tracker} & 935.237 & 1.870,36 & 76,80 \\ \hline
\multirow{2}{*}{$800$ kW} & Fixo & 1.202.618 & 1.538,66 & 73,84 \\ 
& \textit{Tracker} & 1.498.156 & 1.872,61 & 76,88 \\ \hline
\multirow{2}{*}{$1$ MW} & Fixo & 1.538.740 & 1.538,66 & 75,58 \\ 
& \textit{Tracker} & 1.872.695 & 1.872,61 & 76,90 \\ \hline
\end{tabular}
\fonteautor
\label{tab:resumot}
\end{table}

\subsection{Resumo Financeiro das Usinas}
De maneira complementar, a Tabela 21 consolida os principais indicadores financeiros associados às usinas fotovoltaicas estudadas, com o objetivo de proporcionar uma visão geral e organizada dos resultados econômicos obtidos. A apresentação sintetizada dos custos, do tempo de retorno do investimento e dos indicadores de rentabilidade possibilita uma consulta mais ágil e eficiente aos dados, servindo como base para as análises comparativas e para a discussão dos resultados financeiros que serão desenvolvidas posteriormente. Assim a \autoref{tab:resumof} contribui para a clareza e a sistematização das informações econômicas do estudo.

\begin{table}[h!]
\centering
\caption{Resumo dos resultados econômico-financeiros estimados no PV*SOL para usinas fotovoltaicas de diferentes potências em Belo Horizonte--MG, comparando sistemas de estrutura fixa e com rastreamento solar de um eixo.}
\begin{tabular}{C{2cm}C{2cm}C{2cm}C{3cm}C{2cm}C{2cm}} \hline
\textbf{Potência} & \textbf{Sistema} & \textbf{Custo de geração [R\$/kWh]} & \textbf{\textit{PayBack} [anos]]} & \textbf{\ac{TIR} [\%]} & \textbf{\ac{ROI} [\%]} \\ \hline
\multirow{2}{*}{$10$ kW} & Fixo & $0,2198$ & $7$ anos, $10$ meses & 11,91 & 14,26 \\
& \textit{Tracker} & 0,2321 & $8$ anos & 11,71 & 14,52 \\ \hline
\multirow{2}{*}{$50$ kW} & Fixo & 0,1961 & $7$ anos, $3$ meses & 13,07 & 15,30 \\
& \textit{Tracker} & 0,2340 & $8$ anos, $4$ meses & 11,00 & 13,95 \\ \hline
\multirow{2}{*}{$100$ kW} & Fixo & 0,1846 & $6$ anos, $9$ meses & 14,13 & 16,26 \\
& \textit{Tracker} & 0,2200 & $7$ anos, $10$ meses & 11,99 & 14,83 \\ \hline
\multirow{2}{*}{$300$ kW} & Fixo & 0,1622 & $5$ anos, $11$ meses & 16,34 & 18,31 \\
& \textit{Tracker} & 0,2133 & $7$ anos, $8$ meses & 12,24 & 15,06 \\ \hline
\multirow{2}{*}{$500$ kW} & Fixo & 0,1637 & $5$ anos, $7$ meses & 17,37 & 19,79 \\
& \textit{Tracker} & 0,1832 & $6$ anos, $4$ meses & 15,14 & 17,68 \\ \hline
\multirow{2}{*}{$800$ kW} & Fixo & 0,1427 & $5$ anos, $2$ meses & 18,97 & 20,81 \\
& \textit{Tracker} & 0,1487 & $5$ anos, $1$ mês & 19,47 & 21,79 \\ \hline
\multirow{2}{*}{$1$ MW} & Fixo & 0,1251 & $4$ anos, $6$ meses & 21,99 & 23,74 \\
& \textit{Tracker} & 0,1300 & $4$ anos, $4$ meses & 22,67 & 24,91 \\ \hline
\end{tabular}
\fonteautor
\label{tab:resumof}
\end{table}

\section{Análise geral}
%Nesta seção são apresentados os principais resultados consolidados do estudo, abordando tanto os aspectos técnicos quanto os financeiros dos sistemas fotovoltaicos avaliados. A análise técnica contempla indicadores como a energia anual gerada, o rendimento específico e a eficiência global, discutindo o comportamento desses parâmetros em função da potência instalada e do tipo de tecnologia adotada (sistema fixo ou com rastreamento). Já a análise financeira foca em métricas como custo nivelado de geração (R\$/kWh), \textit{payback}, taxa interna de retorno (\ac{TIR}) e retorno sobre o investimento (\ac{ROI}), permitindo verificar a viabilidade econômica e comparar o desempenho entre as diferentes configurações estudadas. Dessa forma, busca-se oferecer uma visão integrada que relacione ganhos de desempenho energético com os impactos nos custos de implantação e retorno financeiro, apontando em que situações a utilização de \textit{trackers} passa a ser mais vantajosa.
Nesta seção realiza-se a síntese integrada dos resultados obtidos, articulando os desempenhos energético e econômico dos sistemas fotovoltaicos avaliados. A análise considera, de forma conjunta, a variação dos indicadores de geração e eficiência com a escala de potência e com a tecnologia empregada, bem como seus desdobramentos nos custos e na rentabilidade dos projetos. Essa abordagem permite estabelecer a relação entre o ganho energético proporcionado pelo rastreamento solar e o investimento adicional requerido, evidenciando como esse equilíbrio evolui ao longo das diferentes faixas de potência. Desse modo, busca-se identificar as condições em que o aumento de produção se traduz efetivamente em vantagem econômica, caracterizando o ponto a partir do qual o uso de \textit{trackers} passa a apresentar maior atratividade no contexto analisado.

\subsection{Aspectos técnicos}
A \autoref{fig:img012} apresenta a relação entre a energia anual injetada e o rendimento específico para sistemas fotovoltaicos com estrutura fixa e com rastreamento solar de um eixo em Belo Horizonte–MG. A análise técnica evidencia que os sistemas com rastreamento apresentam desempenho energético superior em todas as potências analisadas, tanto em
termos de energia anual gerada quanto de rendimento específico (kWh/kWp).

Esse comportamento está associado ao melhor alinhamento angular dos módulos em
relação à trajetória aparente do sol ao longo do dia, proporcionado pelo mecanismo de rastreamento. Como consequência, ocorre um maior aproveitamento da irradiação
incidente e uma redução das perdas associadas ao desalinhamento angular, resultando em maior utilização efetiva da potência instalada. Nos sistemas de estrutura fixa, o ângulo constante de inclinação limita esse aproveitamento, sobretudo nos períodos matutino e vespertino, refletindo-se em menores valores de geração e rendimento específico.
% A análise da energia anual gerada evidencia que os sistemas com rastreamento de um eixo apresentam desempenho superior em comparação aos sistemas fixos em todas as potências avaliadas. Esse comportamento se deve à maior captação da radiação solar proporcionada pelo movimento do \textit{tracker}, que acompanha o deslocamento aparente do sol ao longo do dia.
% A avaliação técnica evidencia que os sistemas com rastreamento de um eixo apresentam desempenho energético superior aos sistemas de estrutura fixa em todas as potências analisadas. Esse comportamento decorre da maior captação de radiação ao longo do dia, proporcionada pelo acompanhamento do movimento aparente do sol, o que amplia o período de operação em condições próximas ao ângulo ótimo de incidência.
% Quando se observa o rendimento específico (kWh/kWp), a mesma tendência é confirmada: o sistema com \textit{tracker} demonstra maior aproveitamento energético por unidade de potência instalada. Esse comportamento está diretamente associado à redução das perdas angulares, uma vez que o seguidor solar mantém os módulos mais próximos da inclinação ideal durante o dia. Em contrapartida, no sistema fixo, a limitação do ângulo de inclinação reduz o aproveitamento da radiação em horários de baixa incidência solar.  A \autoref{fig:img012} nos mostra na relação entre as tres grandezas mencionadas. 
\begin{figure}[h]
	\centering
	\caption{Relação entre energia anual injetada e rendimento específico para sistemas fotovoltaicos com estrutura fixa e com rastreamento solar de um eixo em Belo Horizonte--MG.}
	%\caption{Energia anual gerada x Potencia x Rendimento.}
	\includegraphics[width=0.8\linewidth]{./textuais/figs/graficoenergiaanualxpotenciaxrendimento}
    \fonteautor
	\label{fig:img012}
\end{figure}

% \color{red}
% CONTRADITÓRIO

O gráfico apresentado na \autoref{fig:img013} mostra o ganho percentual de geração proporcionado pelo rastreamento solar em função da potência instalada. Observa-se que o incremento relativo de energia permanece dentro da faixa esperada ao longo das diferentes faixas de potência analisadas, com variações pontuais associadas às condições de simulação e às configurações dos sistemas modelados. Esse comportamento pode indicar que o benefício técnico do rastreamento está predominantemente relacionado ao aumento do tempo de captação eficiente da radiação solar, e não a efeitos de escala intrínsecos à potência instalada.

Dessa forma, o rastreamento solar não deve ser interpretado como uma tecnologia cujo ganho percentual cresce com o porte do sistema, mas como uma solução que eleva de maneira semelhante o desempenho energético relativo em diferentes escalas. Consequentemente, embora o ganho percentual permaneça quase invariável, o ganho absoluto de energia torna-se mais expressivo em sistemas de maior potência, ampliando seu impacto prático.

Nesse contexto, a análise puramente técnica mostra-se insuficiente para determinar a viabilidade do uso de \textit{trackers}, tornando necessária a avaliação dos reflexos econômicos associados a esse ganho energético. Assim, na sequência, são analisados os aspectos financeiros dos sistemas estudados, visando identificar em que condições o aumento de geração proporcionado pelo rastreamento solar se traduz em vantagens econômicas efetivas.

No que se refere à eficiência global do sistema, observa-se desempenho superior nos projetos com rastreamento solar, associado principalmente à maior captação de irradiação ao longo do dia. Esse ganho decorre do melhor alinhamento angular dos módulos em relação à trajetória aparente do sol, o que resulta em maior produção energética anual. Cabe destacar que, nas simulações realizadas, esse desempenho foi obtido considerando um arranjo adequado do layout do sistema, com espaçamento entre fileiras e utilização de estratégias de controle que mitigam perdas por sombreamento entre módulos, como o \textit{backtracking}. Contudo, é importante destacar que esse ganho de eficiência não ocorre de forma isolada, pois está diretamente associado a um aumento no custo de capital (\ac{CAPEX}). Em outras palavras, a maior produção de energia proporcionada pelo rastreamento só é viável mediante um investimento inicial mais elevado, decorrente da necessidade de estruturas móveis, sistemas de acionamento, componentes adicionais e maior demanda de manutenção preventiva. 
\begin{figure}[htb!]
	\centering
	\caption{Ganho percentual de geração de energia anual de sistemas fotovoltaicos com rastreamento solar de um eixo em relação a sistemas fixos, em função da potência instalada, para as condições de Belo Horizonte--MG.}
	\includegraphics[width=0.65\linewidth]{./textuais/figs/ganhodegeracaoxpotencia}
	\fonteautor
	\label{fig:img013}
\end{figure}

Para ilustrar essa relação entre desempenho e investimento, a \autoref{fig:img014} apresenta a comparação entre o \ac{CAPEX} e a eficiência global dos sistemas fixos e com rastreamento ao longo das diferentes potências analisadas. No eixo horizontal, a potência instalada (kW) e, nos eixos verticais, o investimento inicial \ac{CAPEX} (à esquerda, em R\$) e a eficiência global do sistema (à direita, em \%).
\begin{figure}[htb!]
	\centering
	\caption{Relação entre o investimento inicial (\ac{CAPEX}) e a eficiência global de sistemas fotovoltaicos de estrutura fixa e com rastreamento solar de um eixo, em função da potência instalada, para as condições de Belo Horizonte--MG.}
	\includegraphics[width=0.99\linewidth]{./textuais/figs/capexeficiencia}
	\fonteautor
	\label{fig:img014}
\end{figure}

As curvas azuis (\ac{CAPEX} fixo) e laranjas (\ac{CAPEX} com rastreamento) mostram crescimento aproximadamente monotônico com a potência, evidenciando que o investimento aumenta com a escala do sistema, sendo sempre mais elevado para a configuração com \textit{tracker}. Observa-se também que a distância entre as curvas de \ac{CAPEX} se amplia à medida que a potência cresce, indicando que o custo adicional do rastreamento não é constante, mas tende a aumentar em termos absolutos em sistemas maiores. As curvas de eficiência, representadas em cinza (desempenho fixo) e amarelo (desempenho com \textit{tracker}), situam-se no eixo direito e mostram que o rastreamento mantém eficiência superior em toda a faixa de potência analisada, com diferença típica da ordem de 1 a 2 pontos percentuais. Nota-se ainda uma tendência de saturação da eficiência a partir de cerca de 300–500 kW para ambas as configurações, sugerindo que o ganho técnico do rastreamento é relativamente independente da escala. Um ponto particular do gráfico é a leve queda da eficiência do sistema fixo em torno de 800 kW, seguida de recuperação em 1 MW, comportamento que foge à tendência monotônica observada nas demais potências e que pode indicar variação de modelagem ou inconsistência pontual nos dados. De forma geral, as curvas evidenciam o \textit{trade-off} característico do uso de \textit{trackers}: ganho moderado de eficiência associado a aumento significativo de \ac{CAPEX}, relação que se mantém ao longo de toda a faixa de potência considerada.


\subsection{Aspectos Financeiros}
No âmbito econômico, os resultados mostram que o custo de geração (R\$/kWh) é diretamente influenciado pelo investimento inicial. Apesar de gerar mais energia, o sistema com rastreamento possui um custo de implantação mais elevado, o que tende a aumentar o custo nivelado de energia em sistemas de menor porte.

O comportamento ilustrado na \autoref{fig:img015} evidencia a relação entre o investimento inicial e o custo de geração de energia para sistemas fotovoltaicos fixos e com rastreamento solar. No eixo horizontal, a potência instalada (kW), no eixo vertical esquerdo o investimento inicial CAPEX (R\$) e no eixo vertical direito o custo de geração de energia (R\$/kWh). As curvas azuis (\ac{CAPEX} fixo) e laranjas (\ac{CAPEX} com rastreamento) já foram mostradas na \autoref{fig:img014} e estão aqui apenas para melhorar a visualização. 
\begin{figure}[h]
	\centering
	\caption{Relação entre o \ac{CAPEX} e o custo de geração de energia (R\$/kWh) de sistemas fotovoltaicos fixos e com rastreamento de um eixo em função da potência instalada, para Belo Horizonte--MG.}
	\includegraphics[width=0.8\linewidth]{./textuais/figs/capexxcusto}
	\fonteautor
	\label{fig:img015}
\end{figure}

% Ao analisar o custo de geração de energia, verifica-se que, embora os sistemas com rastreamento apresentem uma redução progressiva desse indicador à medida que a potência instalada aumenta, o custo de geração permanece superior ao dos sistemas fixos na faixa de potência considerada neste estudo. Isso indica que, nas condições avaliadas, o ganho energético proporcionado pelo \textit{tracker} não foi suficiente para compensar integralmente o maior investimento inicial.

% Ainda assim, a tendência de aproximação entre as curvas sugere que, em potências instaladas superiores às analisadas ou sob diferentes condições de custo e operação, pode ocorrer o cruzamento entre os custos de geração dos sistemas fixos e rastreados. Tal comportamento reforça o caráter dependente de escala da viabilidade dos sistemas com rastreamento solar, especialmente em projetos de maior porte.
As curvas de custo de geração na \autoref{fig:img015}, representadas em cinza (fixo) e amarelo (\textit{tracker}), apresentam tendência decrescente com o aumento da potência instalada, refletindo ganhos de escala que reduzem o custo nivelado de energia em sistemas maiores. Em toda a faixa de potência considerada, o custo de geração do sistema fixo permanece inferior ao do sistema com rastreamento, embora a diferença entre eles diminua progressivamente com a potência, sugerindo aproximação entre as tecnologias em projetos de maior porte. Nota-se ainda que a curva amarela (\textit{tracker}) apresenta valores iniciais mais elevados e queda mais acentuada até cerca de 500–800 kW, enquanto a curva cinza (fixo) decresce de forma mais suave, mantendo-se sempre abaixo. De forma geral, as curvas evidenciam que, embora o rastreamento proporcione maior geração de energia, o maior \ac{CAPEX} associado mantém o custo de geração superior ao do sistema fixo na faixa analisada, ainda que a diferença se reduza com o aumento da escala, caracterizando um \textit{trade-off} econômico dependente da potência instalada.
%% 

Já o Gráfico apresentado na \autoref{fig:img016} ilustra de forma integrada a evolução do \textit{payback} e da Taxa Interna de Retorno (\ac{TIR}) em função da potência instalada, comparando sistemas fotovoltaicos de estrutura fixa e com rastreamento solar. Inicialmente, observa-se que, em sistemas de menor porte ($10$ a $100$ kW), os sistemas fixos apresentam melhor desempenho financeiro. Nessa faixa, o \textit{payback} dos sistemas fixos é sistematicamente inferior ao dos sistemas com \textit{tracker}, enquanto a \ac{TIR} também se mostra superior, indicando que o aumento do \ac{CAPEX} associado ao rastreamento ainda não é compensado pelo ganho adicional de geração.
\begin{figure}[h]
	\centering
	\caption{Relação entre o \textit{payback} e a \ac{TIR} de sistemas fotovoltaicos fixos e com rastreamento de um eixo em função da potência instalada, para Belo Horizonte--MG.}
	\includegraphics[width=0.8\linewidth]{./textuais/figs/paybackxtirxpotencia}
    \fonteautor
	\label{fig:img016}
\end{figure}

À medida que a potência instalada aumenta, nota-se uma redução progressiva do \textit{payback} em ambas as configurações, porém com comportamento mais acentuado nos sistemas com rastreamento. Na faixa intermediária ($300$ a $500$ kW), embora o sistema fixo ainda apresenta menor tempo de retorno, a diferença entre as duas soluções diminui significativamente. Paralelamente, a \ac{TIR} dos sistemas com \textit{tracker} cresce de forma consistente, aproximando-se dos valores observados nos sistemas fixos, evidenciando um cenário de transição na viabilidade econômica.

%Em sistemas de maior porte, a partir das potências simuladas de $800$ kW, observa-se uma inversão do desempenho relativo entre as configurações analisadas. Nessa faixa, os sistemas com rastreamento passam a apresentar \textit{payback} inferior ao dos sistemas fixos, além de \ac{TIR} superior, demonstrando que o investimento adicional requerido pelos \textit{trackers} é compensado pela maior eficiência energética e pelo aumento da geração anual. Esse comportamento caracteriza uma mudança clara na atratividade financeira, consolidando o rastreamento como a alternativa economicamente mais vantajosa em empreendimentos de maior escala.
Em sistemas de maior porte, a partir das potências simuladas de $800$ kW, observa-se uma inversão do desempenho relativo entre as configurações analisadas. Nessa faixa, os sistemas com rastreamento passam a apresentar \textit{payback} inferior ao dos sistemas fixos, além de \ac{TIR} superior, sugerindo que o investimento adicional requerido pelos \textit{trackers} tende a ser compensado pela maior eficiência energética e pelo aumento da geração anual. Esse comportamento indica uma mudança na atratividade financeira com o aumento da escala, apontando o rastreamento como potencialmente mais vantajoso em empreendimentos de maior porte dentro das condições simuladas neste estudo.

%Dessa forma, os resultados evidenciam que a viabilidade do uso de \textit{trackers} está diretamente associada à potência instalada do sistema. Enquanto em projetos de pequeno porte a solução fixa se mostra mais racional sob o ponto de vista financeiro, em usinas de médio a grande porte o rastreamento solar passa a oferecer melhores indicadores econômicos, reforçando a importância da análise conjunta dos aspectos técnicos e financeiros, conforme discutido nas seções seguintes.
Entretanto, é importante ressaltar que essa inversão não deve ser interpretada como um resultado definitivo ou universal. Os valores adotados nas simulações possuem natureza teórica e refletem premissas específicas de custo, desempenho e operação, podendo variar em aplicações reais conforme condições locais de implantação, mercado e engenharia do projeto. Assim, embora os resultados indiquem uma tendência favorável ao uso de \textit{trackers} em potências elevadas, a consolidação dessa vantagem requer análises adicionais com dados empíricos e estudos de caso reais, o que não constitui escopo deste trabalho, mas se apresenta como um achado relevante que merece investigação futura.

Dessa forma, os resultados evidenciam que a viabilidade do uso de \textit{trackers} está associada à potência instalada do sistema, ainda que dependente das premissas adotadas. Enquanto em projetos de pequeno porte a solução fixa se mostra mais racional sob o ponto de vista financeiro, em usinas de médio a grande porte o rastreamento solar tende a apresentar indicadores econômicos mais favoráveis, reforçando a importância da análise conjunta dos aspectos técnicos e financeiros, conforme discutido nas seções seguintes.

%A \autoref{fig:img017} apresenta o gráfico de bolhas, que integra em uma única visualização três indicadores financeiros fundamentais para a avaliação dos sistemas fotovoltaicos: o \textit{payback}, representado no eixo horizontal, o \ac{TIR}, no eixo vertical, e o\ac{ROI}, indicado pelo tamanho das bolhas possibilitando a visualização conjunta da atratividade financeira e da rentabilidade percentual.
A \autoref{fig:img017} apresenta, em um gráfico de bolhas, a relação integrada entre o tempo de retorno do investimento (\textit{payback}), a taxa interna de retorno (\ac{TIR}) e o retorno sobre o investimento (\ac{ROI}) para sistemas fotovoltaicos de diferentes potências e configurações (fixa e com rastreamento). No plano cartesiano, o eixo horizontal representa o \textit{payback} (anos), o eixo vertical representa a \ac{TIR} (\%), enquanto o tamanho das bolhas é proporcional ao \ac{ROI}, permitindo a comparação simultânea dos principais indicadores financeiros.
\begin{figure}[h]
	\centering
	\caption{Relação integrada entre \textit{payback}, \ac{TIR} e \ac{ROI} de sistemas fotovoltaicos fixos e com rastreamento solar de um eixo em função da potência instalada, para as condições de Belo Horizonte–MG.}
	\includegraphics[
	width=0.99\linewidth,
	% [trim = <esquerda> <inferior> <direita> <superior>, clip]
	trim = 0.0cm 0.4cm 0.0cm 0.9cm, clip
	]{./textuais/figs/bolhas}
    \fonteautor
	\label{fig:img017}
\end{figure}

% De forma geral, o gráfico de bolhas confirma as tendências previamente discutidas no Gráfico \textit{Payback} × \ac{TIR}, evidenciando a relação entre escala do empreendimento e desempenho econômico das configurações analisadas. Observa-se a progressiva migração dos sistemas com rastreamento para regiões de menor \textit{payback} e maior \ac{TIR} à medida que a potência instalada aumenta, acompanhada por bolhas de maior dimensão, indicando elevação do \ac{ROI}.

% Assim, o gráfico atua como uma ferramenta de síntese visual, reforçando que o ganho de rentabilidade proporcionado pelo rastreamento solar torna-se mais expressivo em potências mais elevadas, enquanto, em sistemas de menor porte, os sistemas de estrutura fixa tendem a se concentrar em regiões de maior atratividade financeira. Dessa forma, a \autoref{fig:img017} complementa a análise anterior ao integrar, em uma única representação gráfica, os principais indicadores econômicos considerados neste estudo.

Observa-se que os sistemas de menor porte (10 a 100 kW) concentram-se na região de maior \textit{payback} e menor \ac{TIR}, indicando menor atratividade econômica relativa, independentemente da configuração. Nessa faixa, os sistemas fixos tendem a apresentar desempenho ligeiramente superior, com menor tempo de retorno e indicadores de rentabilidade mais elevados que os sistemas com rastreamento, refletindo o impacto proporcionalmente maior do \ac{CAPEX} adicional dos \textit{trackers} em projetos de pequena escala.

À medida que a potência instalada aumenta, nota-se um deslocamento progressivo dos pontos em direção à região de menor \textit{payback} e maior \ac{TIR}, evidenciando o efeito de escala na melhoria dos indicadores financeiros. Nas potências intermediárias (300 a 500 kW), os resultados das duas configurações tornam-se mais próximos, caracterizando uma zona de transição na viabilidade econômica do rastreamento solar.

Nos sistemas de maior porte (800 kW e 1 MW), os pontos associados aos \textit{trackers} passam a se posicionar na região de menor \textit{payback} e maior \ac{TIR}, com bolhas de maior dimensão, indicando \ac{ROI} superior. Esse comportamento sugere que, nas condições simuladas, o ganho adicional de geração proporcionado pelo rastreamento tende a compensar o maior investimento inicial em projetos de maior escala. Contudo, deve-se ressaltar que essa tendência decorre de premissas teóricas adotadas nas simulações e pode variar conforme condições reais de custo, implantação e operação. Assim, embora o gráfico indique maior atratividade econômica do rastreamento em potências elevadas, tal resultado deve ser interpretado como indicativo e não conclusivo, apontando a necessidade de investigações complementares com dados empíricos.

De forma geral, a distribuição dos pontos confirma que a atratividade financeira dos sistemas fotovoltaicos aumenta com a potência instalada e que o efeito de escala favorece progressivamente a adoção de \textit{trackers}, ainda que essa vantagem dependa das condições específicas consideradas no estudo.

\section{Conclusões Parciais}
Os sistemas com rastreamento apresentam melhor desempenho técnico em todas as potências analisadas, com maior geração, rendimento específico e eficiência global. Entretanto, sob o ponto de vista econômico, a estrutura fixa mostra-se mais vantajosa até aproximadamente $500$ kW, com menor custo de geração, menor \textit{payback} e indicadores de rentabilidade iguais ou superiores, com exceção pontual do \ac{ROI} em $10$ kW. A partir de $800$ kW observa-se uma transição, na qual o sistema com \textit{tracker} passa a apresentar \ac{TIR} e \ac{ROI} superiores e \textit{payback} equivalente ou ligeiramente menor, indicando maior competitividade econômica em sistemas de maior porte. Em $1$ MW, o rastreamento torna-se a alternativa mais atrativa dentro das condições simuladas, sugerindo que o ganho energético compensa o maior investimento inicial, ainda que esse resultado dependa das premissas adotadas e deva ser interpretado com cautela.

\begin{comment}
De maneira geral, os resultados mostraram que os sistemas com rastreamento solar sempre apresentam melhor desempenho técnico, com maior geração de energia, rendimento específico e eficiência global em todas as potências analisadas. Porém, do ponto de vista financeiro, o comportamento é diferente: nas potências menores (até $300$ kW), os sistemas fixos se mostraram mais vantajosos, com menor custo de geração, \textit{payback} mais curto e melhores índices de \ac{TIR} e com excessão da usina de $10$kW \textit{tracker} que possui um \ac{ROI} ligeiramente melhor que a estrutura fixa, todas as demais potências possuem vantagem nesse indicador sobre a estrutura móvel. A partir de $500$ kW, o cenário muda, e os trackers começam a apresentar resultados equivalentes, consolidando-se como a opção mais atrativa nas potências maiores ($800$ kW e $1$ MW), onde o ganho energético compensa plenamente o investimento adicional.


Nova 

Para usinas de pequeno porte, especificamente nas potências de $10$ kW, $50$ kW e $100$ kW, os resultados indicam que a estrutura fixa é a alternativa economicamente mais adequada. Nessas faixas, o sistema fixo apresenta, de forma consistente, menor custo de geração de energia, menor tempo de retorno do investimento e indicadores de rentabilidade iguais ou superiores aos do sistema com \textit{tracker}. O maior investimento inicial e os custos associados ao rastreamento não são compensados pelos ganhos de geração nessas escalas, tornando o uso de \textit{trackers} financeiramente menos atrativo.

Na faixa intermediária de potência, representada pela usina de $300$ kW, a estrutura fixa continua apresentando vantagem econômica clara. Os resultados demonstram diferenças expressivas nos indicadores financeiros, com destaque para o \textit{payback} significativamente menor e valores de \ac{TIR} e \ac{ROI} superiores no sistema fixo. Dessa forma, mesmo em um porte intermediário, o ganho adicional de geração proporcionado pelo \textit{tracker} ainda não é suficiente para superar os custos adicionais associados à sua implementação.

Para a potência de $500$ kW, observa-se a manutenção da vantagem econômica da estrutura fixa, embora a diferença entre as configurações seja menor quando comparada às potências inferiores. O sistema fixo ainda apresenta menor custo de geração, \textit{payback} mais curto e maiores indicadores de rentabilidade, indicando que, até esse nível de potência, a adoção de \textit{trackers} não se justifica financeiramente sob os parâmetros analisados.

A partir da usina de $800$ kW, os resultados apontam para um cenário de transição. Embora o sistema de estrutura fixa continue apresentando menor custo de geração, o sistema com \textit{tracker} passa a registrar valores superiores de \ac{TIR} e \ac{ROI}, além de um \textit{payback} muito próximo. Esse comportamento indica que, nessa faixa de potência, os ganhos adicionais de geração começam a compensar o maior investimento inicial, tornando o uso de \textit{trackers} financeiramente competitivo, ainda que não claramente dominante.

Por fim, na usina de $1$ MW, os resultados demonstram que o sistema com \textit{trackers} se torna economicamente mais atrativo, apresentando menor \textit{payback} e indicadores de rentabilidade superiores, apesar do custo de geração ligeiramente mais elevado. Nessa escala, os ganhos de produção energética proporcionados pelo rastreamento solar superam os custos adicionais de implantação, justificando financeiramente a adoção de \textit{trackers}.


\color{magenta}
\section*{Inconsistências numéricas}

\begin{itemize}
	\item \textbf{Ganho percentual (22\% versus 25\%) não consistente ao longo do capítulo.}\\
	Trechos: 
	``mantendo a tendência de 22\% de ganho'' (100 kW) 
	vs ``corresponde a cerca de 25\% de aumento na produção'' (800 kW) 
	vs ``o incremento relativo de energia permanece aproximadamente constante'' (análise da Figura do ganho percentual).\\
	Inconsistência: o texto sustenta ganho percentual aproximadamente constante (em torno de 22\%), mas reporta 25\% em 800 kW. O autor deve revisar os dados e/ou justificar tecnicamente a exceção.
	
	\item \textbf{(FEITO) Resumo financeiro diverge das tabelas individuais no caso 10 kW (TIR do tracker).}\\
	Trechos: Tabela 10 kW (individual): ``Tracker ... TIR = 11,94'' 
	vs Tabela resumo financeiro: ``Tracker ... TIR = 11,64'' (10 kW).\\
	Inconsistência: mesma grandeza (TIR do caso 10 kW com tracker) aparece com valores diferentes. O autor deve revisar a consolidação.
	
	\item \textbf{(FEITO)Resumo financeiro diverge das tabelas individuais no caso 300 kW (TIR do fixo).}\\
	Trechos: Tabela 300 kW (individual): ``Fixo ... TIR = 18,31'' 
	vs Tabela resumo financeiro: ``Fixo ... TIR = 16,34'' (300 kW).\\
	Inconsistência: TIR do sistema fixo em 300 kW muda entre tabela individual e resumo. O autor deve revisar a transcrição/exportação.
	
	\item \textbf{(FEITO)Resumo financeiro diverge do texto e da tabela individual no caso 1 MW (tracker: TIR e ROI).}\\
	Trechos: Tabela 1 MW (individual) e texto: ``TIR = 22,67'' e ``ROI = 24,91'' 
	vs Tabela resumo financeiro: ``TIR = 24,91'' e ``ROI = 24,08'' (1 MW, tracker).\\
	Inconsistência: no resumo, TIR e ROI do tracker em 1 MW não batem com a tabela individual e com o texto (há forte indício de troca/erro de lançamento). O autor deve revisar.
	
	\item \textbf{(FEITO)Resumo técnico diverge da tabela individual no caso 100 kW (eficiência do fixo).}\\
	Trechos: Tabela 100 kW (individual): ``Fixo ... Eficiência = 70,30'' 
	vs Tabela resumo técnico: ``Fixo ... Eficiência = 70,25'' (100 kW).\\
	Inconsistência: a eficiência do sistema fixo em 100 kW aparece com dois valores. O autor deve revisar e padronizar arredondamento.
	
	\item \textbf{(FEITO)Resumo técnico diverge da tabela individual no caso 10 kW (eficiência do tracker).}\\
	Trechos: Tabela 10 kW (individual): ``Tracker ... Eficiência = 69,73'' 
	vs Tabela resumo técnico: ``Tracker ... Eficiência = 69,70'' (10 kW).\\
	Inconsistência: divergência pequena, mas deve haver consistência de arredondamento/lançamento.

	
	\item \textbf{(FEITO)Resumo técnico diverge das tabelas individuais nos casos 500 kW e 1 MW (eficiência do fixo).}\\
	Trechos: 500 kW (individual): ``Fixo ... Eficiência = 75,60'' 
	vs resumo técnico: ``Fixo ... Eficiência = 75,61''.\\
	Trechos: 1 MW (individual): ``Fixo ... Eficiência = 75,60'' 
	vs resumo técnico: ``Fixo ... Eficiência = 75,58''.\\
	Inconsistência: diferenças pequenas, mas afetam rastreabilidade. O autor deve revisar os valores originais e o critério de arredondamento.
	
	\item \textbf{(FEITO)Legenda (caption) da tabela financeira de 50 kW está factualmente errada.}\\
	Trecho: na seção 50 kW, a tabela financeira está descrita como ``para a usina de 10 kW''.\\
	Inconsistência: a legenda não corresponde à seção/tabela (50 kW). O autor deve corrigir para evitar erro factual de documentação.
	
	\item \textbf{(FEITO)Inconsistência de notação numérica na tabela técnica de 500 kW (formatação).}\\
	Trecho: na tabela técnica de 500 kW, aparece ``Tracker ... 935.237'' sem o mesmo padrão de marcação usado nos demais valores (por exemplo, com delimitadores e símbolo de modo matemático).\\
	Inconsistência: pode ser apenas formatação, mas pode mascarar erro de transcrição. O autor deve revisar o valor e padronizar a apresentação.
\end{itemize}


\section*{Contradições técnicas}

\begin{itemize}
	\item \textbf{(FEITO)Conclusão parcial generaliza “melhores índices de TIR e ROI” para fixo até 300 kW, mas 10 kW contradiz.}\\
	Trecho (Conclusões Parciais): ``até 300 kW ... melhores índices de TIR e ROI''.\\
	Trecho (10 kW): ``Fixo ... TIR 11,91\% e ROI 14,26\%'' e ``Tracker ... TIR 11,94\% e ROI 14,52\%''.\\
	Contradição: em 10 kW, TIR e ROI do tracker são maiores. Sugestão: tornar a afirmação tecnicamente direta, especificando quais métricas favorecem cada configuração por faixa de potência, sem generalizar.
	
	\item \textbf{(FEITO)Afirmação de ganho percentual “quase invariável” conflita com o próprio caso de 800 kW.}\\
	Trechos: ``incremento relativo permanece aproximadamente constante'' 
	vs ``cerca de 25\%'' (800 kW).\\
	Incoerência: se 800 kW é exceção, isso precisa ser dito e tecnicamente sustentado; caso contrário, os dados devem ser revisados. Sugestão: explicitar objetivamente se há exceção e por quê.
	
	\item \textbf{(FEITO)Uso de “vantagem” e “mais atrativo” com critérios diferentes no mesmo bloco (métricas conflitantes).}\\
	Trechos (800 kW): ``o sistema fixo ainda apresenta vantagem'' (custo de geração menor) 
	e ``tracker ... maior rentabilidade percentual'' (TIR/ROI maiores) 
	e ``cenário de equilíbrio econômico''.\\
	Incoerência: “vantagem” muda de critério sem aviso (custo de geração versus TIR/ROI versus payback). Sugestão: ser tecnicamente direto e declarar que as métricas divergem e que a conclusão depende do critério adotado.
	
	\item \textbf{(FEITO)Causalidade potencialmente frágil: “tracker minimiza sombreamentos”.}\\
	Trecho: ``o tracker contribui para minimizar sombreamentos''.\\
	Incoerência técnica potencial: rastreamento pode reduzir ou aumentar sombreamento dependendo de layout, espaçamento, backtracking e geometria. Sem declarar as premissas do modelo (se houve backtracking e como foi o arranjo), a afirmação fica ambígua. Sugestão: explicitar o que foi configurado na simulação antes de atribuir causa.
\end{itemize}


\section*{Repetições}

\begin{itemize}
	\item \textbf{(FEITO)Aberturas de seção repetidas em todas as potências, com mudança apenas dos números.}\\
	Trechos recorrentes: ``Nesta seção será feita um resumo... A Tabela X mostra...'' aparece em todas as subseções técnicas e financeiras.\\
	Ideia repetida: introdução padronizada que não agrega informação nova. Sugestão: condensar com uma frase padrão no início do capítulo ou reduzir essas introduções.
	
	\item \textbf{(FEITO)Mesma narrativa por faixas de potência repetida em dois gráficos diferentes.}\\
	Trechos: descrição do gráfico payback e TIR (pequeno: fixo melhor; intermediário: aproximação; grande: tracker melhor) 
	e descrição do gráfico de bolhas repete a mesma segmentação e conclusão.\\
	Ideia repetida: mesma análise qualitativa em duplicidade. Sugestão: condensar e usar um dos gráficos como síntese, referenciando o outro apenas como reforço.
	
	\item \textbf{(FEITO)Explicação física do ganho do tracker repetida em sequência curta.}\\
	Trechos: ``maior captação ao longo do dia'' + ``reduz perdas por desalinhamento'' + ``amplia o período de operação próximo ao ângulo ótimo''.\\
	Ideia repetida: mesma causa física (melhor alinhamento angular) descrita várias vezes. Sugestão: condensar em um único bloco explicativo.
\end{itemize}


\section*{Problemas de clareza técnica}

\begin{itemize}
	\item \textbf{(FEITO metodologia) Métrica “Eficiência global do sistema” não é definida.}\\
	Trechos: coluna ``Eficiência [\%]'' em todas as tabelas técnicas e discussão de “ganho de eficiência”.\\
	Problema: não fica claro se é PR, eficiência global do software, ou outra métrica; isso afeta interpretação e comparabilidade entre potências.
	
	\item \textbf{“(Feito metodologia )Custo de geração (R\$/kWh)” não tem premissas explícitas.}\\
	Trechos: tabelas financeiras e análise de custo de geração.\\
	Problema: não está claro horizonte, taxa de desconto, vida útil, degradação, OPEX incluído, reposições (por exemplo, inversores), e se há atualização monetária. Sem isso, “custo de geração” fica pouco rastreável.
	
	\item \textbf{( não identifiquei )Figura CAPEX x Eficiência contém duas legendas.}\\
	Trecho: aparecem duas linhas de legenda para a mesma figura (uma curta e outra longa).\\
	Problema: pode causar erro de compilação/numeração e dificulta referência consistente.
	
	\item \textbf{Marca editorial “CONTRADITÓRIO” dentro do texto.}\\
	Trecho: aparece uma anotação em vermelho “CONTRADITÓRIO”.\\
	Problema: indica pendência não resolvida e quebra o fluxo científico do capítulo. Deve ser removida após revisão, ou substituída por discussão objetiva (com revisão dos dados).
	
	\item \textbf{(FEITO)Afirmação de “cruzamento” das curvas sem explicitar critério numérico.}\\
	Trecho: ``a partir de aproximadamente 800 kW, ocorre o cruzamento das curvas de payback e de TIR''.\\
	Problema: como os pontos são discretos (800 kW e 1 MW), não fica claro se houve interpolação ou se o “cruzamento” é apenas comparação ponto a ponto. Seria importante indicar explicitamente em quais potências cada métrica muda de sinal (payback e TIR) e se ocorre simultaneamente.
	
	\item \textbf{(FEITO)Notação monetária inconsistente (R\$, R\$).}\\
	Trechos: ``R\$ 0,2198/kWh'' em um ponto e ``R\$ 0,1251/kWh'' em outro.\\
	Problema: pode gerar erro tipográfico e confusão na leitura, além de dificultar padronização do documento.
\end{itemize}

\end{comment}
\color{black}

\chapter{Conclusão}\label{cap:capitulo_5}


Este trabalho teve como objetivo avaliar a viabilidade técnica e financeira da utilização de sistemas fotovoltaicos fixos e com rastreamento solar de um eixo em diferentes portes de usinas, variando de 10 kW até 1 MW. Para isso, foram realizadas simulações no software PV*SOL, ferramenta amplamente utilizada no setor fotovoltaico, o que confere confiabilidade e consistência aos resultados obtidos.

As análises mostraram que os sistemas com rastreamento apresentam maior geração de energia em todas as potências estudadas, com ganhos médios de $20$ a $25$\% em relação aos sistemas fixos. Contudo, o ponto-chave está na viabilidade econômica. Em sistemas de menor porte (até 300 kW), os fixos se mostraram mais atrativos financeiramente, já que possuem menor custo de geração, \textit{payback} mais curto e melhores indicadores de \ac{TIR} e \ac{ROI}.

A partir da faixa de 500 kW, os resultados passam a indicar equilíbrio entre as duas configurações, e em potências acima de 800 kW o uso de \textit{trackers} se torna claramente mais vantajoso, apresentando menor tempo de retorno e índices financeiros superiores, além de manter o ganho técnico de produção de energia.

Dessa forma, conclui-se que a adoção de rastreadores solares de um eixo é recomendada principalmente para usinas de grande porte (maior que $800$ kW), onde o investimento adicional é compensado pelo aumento de geração e pelo melhor desempenho financeiro. Já para projetos menores, a configuração fixa ainda permanece como a opção mais eficiente em termos de custo-benefício. 

\ac{TIR}, \ac{VPL} e \textit{Payback}, 

%\section{Trabalhos Futuros}
%%Para dar continuidade a ferramenta desenvolvida, são elencadas as seguintes propostas:

%\begin{enumerate}[]
    %\item \textbf{\underline{Migrar ferramenta}}: A ferramenta utilizada foi uma planilha eletrônica. A ideia migrar para uma abordagem de código aberto cooperação da comunidade. São duas vertentes possíveis atualmente: \textit{(i)} javascript ou (\textit{ii}) python; 
    
   % \item \textbf{\underline{Disponibilizar ferramenta online}}: Dependendo da migração feita, pretende-se disponibilizar a ferramenta gratuitamente em ambiente online. Existem alguns serviços de hospedagem gratuita que estão sendo estudados.
    
   % \item \textbf{\underline{Estudos de casos}}: Realizar estudos de casos reais com a ferramenta e mostrar tutoriais.
%\end{enumerate}

%a principal tarefa será a otimização, trabalhando com o máximo de variáveis possíveis, para que a resposta seja mais próxima possível da realidade. Outro aspecto importante será migrar essa ferramenta para o Python.

%O python é uma linguagem de programação de alto nível, dinâmica e modular,o que possibilita maior controle e estabilidade de códigos para projetos de grandes proporções.


%\textcolor{red}{Descrever como você planeja a execução do seu trabalho para o TCC2. O que você irá fazer e como. Essa é uma das partes mais importantes, pois sua banca pode te ajudar a melhorá-lo.}



%%%%%%%%%%%%%%%%%%%%%%%%%%%%%%%%%
%                               %
%         Pós textuais          %
%                               %
%%%%%%%%%%%%%%%%%%%%%%%%%%%%%%%%%

\postextual


% Referências bibliográficas
%\bibliography{Referencias}
\bibliography{./referencias}

%% Apendices
\begin{apendicesenv}
	%\include{./postextuais/1apendice}
	%\chapter{Meu titulo}

\section{Considerações iniciais}

\section{Estado da Arte}


\end{apendicesenv}

%% Anexos
\begin{anexosenv}
	 %\include{./postextuais/1anexo}
	 %\chapter{Meu titulo}

\section{Considerações iniciais}

Veja a \autoref{fig:internet2022pato} (p\'ag). \pageref{fig:internet2022pato})

\section{Estado da Arte}
\end{anexosenv}

\end{document}

%%%%%%%%%%%%%%%%%%%%%%%%%%%%%%%%%
%                               %
%         Comentários gerais    %
%                               %
%%%%%%%%%%%%%%%%%%%%%%%%%%%%%%%%%


%- Números com $  FEITO
%- Todas as figuras/tabelas devem ser chamadas no texto Feito 
%- Não economize nos captions figuras/tabelas FEITO
%- Referências: 
%	- autor não faz parte da frase - parafraseando
%	- cite         A vida é bela \cite{}.
%	
%	- Falando diretamente do autor, sujeito/objeto
%	- citeonline   Segundo citeonline a vida é bela.
%
% Datas: 
% - 
% - Mandar para banca 21/10/22
% - Apresentação      28/10/22
% 
% Resumo:     assistir como fazer
% Introdução: assistir aulas (objetivo) FEITO
% Conclusão:  assistir aulas 
% 
% Apresentação: assistir aulas
% Só na introdução: estado da arte, objetivo e justificativa FEito
% definições de equações FEito
% Samuel, todo o capítulo 2 deve ser feito de forma genérica. 
%o s equacionamentos devem ser coerentes em TODO o texto
%
%







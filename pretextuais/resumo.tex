\setlength{\absparsep}{18pt} % ajusta o espaçamento dos parágrafos do resumo
\begin{resumo}
%O resumo deve apresentar: 
%a)	contexto, 
%b)	lacuna, 
%c)	objetivo, 
%d)	metodologia, 
%e)	resultados e 
%f)	Conclusões e implicações para academia/sociedade. 
% PARAGRAFO UNICO E +- 250 PALAVRAS

%%%%%%%%%%%%%%%%%%%%%
% CONTEXTO
% REDUZIR PARA DOIS PERÍODOS CLAROS
%Nos últimos anos, os sistemas fotovoltaicos ganharam bastante espaço no Brasil, acompanhando uma tendência mundial de busca por fontes renováveis e sustentáveis de energia. Essa tecnologia vem se tornando cada vez mais acessível, eficiente e presente em diferentes escalas, desde pequenas residências até grandes usinas. Além de ser uma fonte limpa, que reduz a emissão de gases de efeito estufa, a energia solar também ajuda a diminuir a dependência de fontes convencionais e a trazer mais segurança energética para o país.
Sistemas fotovoltaicos ganharam bastante espaço no Brasil, acompanhando uma tendência mundial de busca por fontes renováveis e sustentáveis de energia.
%%%%%%%%%%%%%%%%%%%%%
% LACUNA OU PROBLEMA DE PESQUISA
% AUMENTAR PARA DOIS PERIODOS CLAROS
%Com toda essa expansão, surge uma dúvida prática e importante: quando faz sentido usar sistemas fixos e quando é melhor apostar em sistemas com rastreamento solar? Em outras palavras, até que ponto o uso de \textit{trackers} se mostra realmente viável e justificável, levando em conta não apenas o aumento de geração de energia, mas também os custos envolvidos e o retorno do investimento? 
No entanto, não existe consenso sobre o quando é mais vantajoso usar sistemas fixos ou com \textit{trackers} para rastreamento solar e melhor aproveitamento da energia solar.
%%%%%%%%%%%%%%%%%%%%%
% OBJETIVO
% COLOCAR EM UM PERIODO (MÁXIMO DOIS) CLAROS
% DEVE FICAR PARECIDO COM OBJETIVO DA INTRODUÇÃO
Neste sentido, o objetivo deste trabalho é avaliar a viabilidade econômica do uso de \textit{trackers} em sistemas fotovoltaicos, buscando entender a partir de qual faixa de potência eles passam a ser economicamente e tecnicamente vantajosos. Para dar confiabilidade aos resultados, foi utilizada a ferramenta PV*SOL, que é amplamente reconhecida no setor e permite simular diferentes cenários de forma detalhada e próxima da realidade.
%%%%%%%%%%%%%%%%%%%%%
% METODOLOGIA - COMO SERÁ FEITO
% FALAR DAS TÉCNICAS UTILIZADAS E DA FERRAMENTA PRODUZIDA
% DOIS PERIODOS OU MAIS
A metodologia consistiu em simular diversos cenários no PV*SOL, variando a potência dos sistemas entre $10$ kW e $1$ MW para configurações fixas e com rastreamento de um eixo. A partir dessas simulações, foram analisados aspectos técnicos (como geração de energia, eficiência e rendimento) e financeiros (como \textit{payback}, \ac{TIR}, \ac{ROI} e custo de geração). % Essa combinação de análises possibilitou avaliar não só a diferença de desempenho entre os dois tipos de sistema, mas também sua viabilidade no longo prazo.
%\textcolor{red}{(RETIRAR) Este trabalho monográfico teve por objetivo estudar a viabilidade econômica e realizar o dimensionamento de sistemas fotovoltaicos residenciais.}
%%%%%%%%%%%%%%%%%%%%%
% RESULTADOS
% RESUMIR PRINCIPAIS RESULTADOS QUAIS?
% Ex: Os casos de estudo mostraram que ...
Os resultados mostraram que os \textit{trackers} sempre entregam maior produção de energia em comparação aos sistemas fixos, com ganhos médios de $20$ a $25\% $. No entanto, em sistemas menores (até cerca de $300$ kW), esse ganho não compensa o maior custo inicial, fazendo com que os fixos se mostrem mais atrativos financeiramente. A partir de potências intermediárias, como $500$ kW, a diferença começa a diminuir, e acima de 800 kW o rastreamento se torna mais economicamente vantajoso.
%%%%%%%%%%%%%%%%%%%%%
% IMPACTOS - QUEM SE BENEFICIARÁ
% PARECIDO COM JUSTIFICATIVA
%Esse estudo pode beneficiar principalmente investidores do setor de energia solar, ajudando na tomada de decisão sobre qual tecnologia adotar em seus projetos. Também pode servir como referência para o meio acadêmico, contribuindo para novas pesquisas e estudos comparativos, além de ser útil para empresas e profissionais do setor fotovoltaico que buscam otimizar custos e maximizar a eficiência de seus empreendimentos.
Espera-se que este trabalho possa servir como referência para o meio acadêmico, contribuindo para novas pesquisas e estudos comparativos, além de ser útil para empresas e profissionais do setor fotovoltaico que buscam otimizar custos e maximizar a eficiência de seus empreendimentos.
	
	\textbf{Palavras-chave}: Sistemas Fotovoltaicos, Viabilidade, \textit{Trackers}, \textit{Payback}, \ac{TIR}, \ac{ROI}. 
\end{resumo}
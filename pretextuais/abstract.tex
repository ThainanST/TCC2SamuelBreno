\begin{resumo}[Abstract]
	\begin{otherlanguage*}{english}
		Photovoltaic systems have gained significant ground in Brazil, following a global trend toward the adoption of renewable and sustainable energy sources. However, there is no consensus on when it is more advantageous to use fixed systems or systems with trackers for solar tracking and improved utilization of solar energy. In this context, the objective of this work is to evaluate the economic feasibility of using trackers in photovoltaic systems, seeking to understand from which power range they become economically and technically advantageous. To ensure the reliability of the results, the PV*SOL tool was used, which is widely recognized in the sector and allows different scenarios to be simulated in a detailed and realistic manner. The methodology consisted of simulating various scenarios in PV*SOL, varying system power between 10 kW and 1 MW for fixed configurations and single-axis tracking. From these simulations, technical aspects (such as energy generation, efficiency, and performance) and financial aspects (such as payback, internal rate of return (IRR), Return on Investment (ROI), and cost of generation) were analyzed. The results showed that trackers always deliver higher energy production compared to fixed systems, with average gains of 20 to 25\%. However, in smaller systems (up to about 300 kW), this gain does not offset the higher initial cost, making fixed systems more financially attractive. From intermediate power levels, such as 500 kW, the difference begins to decrease, and above 800 kW tracking becomes more economically advantageous. It is expected that this work may serve as a reference for the academic community, contributing to new research and comparative studies, as well as being useful for companies and professionals in the photovoltaic sector seeking to optimize costs and maximize the efficiency of their projects.
		%In recent years, photovoltaic systems have gained significant traction in Brazil, following a global trend toward renewable and sustainable energy sources. This technology has become increasingly accessible, efficient, and available on a variety of scales, from small homes to large power plants. Besides being a clean source that reduces greenhouse gas emissions, solar energy also helps reduce dependence on conventional sources and bring greater energy security to the country. With all this expansion, a practical and important question arises: when does it make sense to use fixed systems and when is it better to opt for systems with solar tracking? In other words, to what extent is the use of trackers truly viable and justifiable, considering not only the increase in energy generation but also the costs involved and the return on investment? The objective of this study is to assess the feasibility of using trackers in photovoltaic systems, seeking to understand at what power range they become economically and technically advantageous. To ensure the reliability of the results, the PV*SOL tool was used. This tool is widely recognized in the industry and allows for detailed, realistic simulations of different scenarios. The methodology consisted of simulating various scenarios in PV*SOL, varying system power between 10 kW and 1 MW, and comparing fixed and single-axis tracking configurations. Based on these simulations, technical aspects (such as energy generation, efficiency, and yield) and financial aspects (such as payback, internal rate of return (IRR), ROI, and generation cost) were analyzed. This combination of analyses made it possible to assess not only the performance differences between the two types of systems but also their long-term viability. The results showed that trackers consistently deliver higher energy production compared to fixed systems, with average gains of 20 to 25\%. However, in smaller systems (up to approximately 300 kW), this gain does not offset the higher initial cost, making fixed systems more financially attractive. Starting at intermediate power levels, such as 500 kW, the difference begins to narrow, and above 800 kW, tracking becomes clearly more advantageous, both from a technical and economic perspective. This study can primarily benefit investors in the solar energy sector, helping them decide which technology to adopt for their projects. It can also serve as a reference for academia, contributing to new research and comparative studies, in addition to being useful for companies and professionals in the photovoltaic sector seeking to optimize costs and maximize the efficiency of their projects.
		
		\vspace{\onelineskip} 
		\noindent 
		\textbf{Keywords}: Photovoltaic Systems, Viability, Trackers, Payback, IRR, ROI..
	\end{otherlanguage*}
\end{resumo}

%% inserir lista de abreviaturas e siglas
%% ---
%\begin{siglas}
%  \item[ABNT] Associação Brasileira de Normas Técnicas
%  \item[abnTeX] ABsurdas Normas para TeX
%\end{siglas}
%% ---

\chapter*{Lista de Siglas}
%Define as siglas utilizadas nesta dissertação
\begin{acronym}[HOSVD] %lembrar que o argumento opcional é o maior acrônimo utilizado %lembrar de manter a lista em ordem alfabética
	%\acro{CA}{Corrente Alternada}
	%\acro{CC}{Corrente Contínua}
	%\acro{IGBT}{\textit{Insulated Gate Bipolar Transistor}}
	%\acro{LGR}{Lugar Geométrico das Raízes}
	%\acro{MIMO}{\textit{Multiple Input Multiple Output}}
	%\acro{P}{\textit{Proportional}}
	%\acro{PD}{\textit{Proportional-Derivative}}
	%\acro{PI}{\textit{Proportional-Integral}}
	%\acro{PID}{\textit{Proportional-Integral-Derivative}}
	%\acro{PWM}{\textit{Pulse Width Modulation}}
	

    \acro{ANEEL}{Agência Nacional de Energia Elétrica}
    \acro{GD}{Geração Distribuida}
    \acro{CC}{Corrente Continua}
    \acro{CA}{Corrente Alternada}
	\acro{CRESESB}{Centro de referência para as energias solar e eólica Sérgio de S. Brito}
	\acro{FS}{sistema fotovoltaico (do inglês, \textit{photovoltaic system})}
	\acrodefplural{FS}{sistemas fotovoltaicos (do inglês, \textit{photovoltaic systems})}
	\acro{SFCR}{Sistemas Fotovoltaicos Conectados a Rede}
	\acro{TIR}{taxa interna de retorno}
	\acro{TMA}{taxa mínima de atratividade}
	\acro{VPL}{valor presente liquido}
    \acro{ROI}{Retorno do Investimento (do inglês, \textit{Return of Investment})}
    \acro{EPÉ}{Empresa de Pesquisa Energética}
	\acro{LCOE}{Custo Nivelado de Energia (do inglês, \textit{levelized cost of energy})}
	\acro{CAPEX}{\textit{CAPITAL EXPENDITURE}}
    \acro{OPEX}{\textit{OPERATIONAL EXPENDITURE}}
\end{acronym}
\newpage